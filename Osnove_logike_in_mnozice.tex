\section{Osnove logike in teorije množice}

\begin{frame}
    \sectionpage
\end{frame}

\begin{frame}
    \tableofcontents[currentsection, hideothersubsections]
\end{frame}

    \subsection{Osnove logike}

        \begin{frame}
            \frametitle{Izjave}

            \begin{alertblock}{Matematična izjava}
                \textbf{Matematična izjava} je vsaka smiselna poved, za katero 
                lahko določimo resničnost oz. pravilnost.
            \end{alertblock}

            \begin{alertblock}{Logična vrednost matematične izjave}
                Matematična izjava lahko zavzame dve logični vrednosti:
                \begin{itemize}
                    \item izjava je \textbf{resnična}/\textbf{pravilna}, 
                        oznaka $\mathbf{R}$/$\mathbf{P}$/$\mathbf{1}$/$\mathbf{\top}$;
                    \item izjava je \textbf{neresnična}/\textbf{nepravilna}, 
                        oznaka $\mathbf{N}$/$\mathbf{0}$/$\mathbf{\bot }$.
                \end{itemize}                
            \end{alertblock}

            \begin{alertblock}{}
                Izjave označujemo z velikimi tiskanimi črkami ($A$, $B$, $C$ ...).
            \end{alertblock}
        \end{frame}

        \begin{frame}
            \begin{exampleblock}{Naloga ???}
                
            \end{exampleblock}
        \end{frame}

        \begin{frame}
            \begin{alertblock}{Enostavne in sestavjene izjave}
                Izjave delimo med:
                \begin{itemize}
                    \item \textbf{elementarne}/\textbf{enostavne izjave} -- ne moremo 
                        jih razstaviti na bolj enostavne;
                    \item \textbf{sestavljene izjave} -- sestavljene iz elementarnih izjav, 
                        ki jih med seboj povezujejo \textbf{izjavne povezave} oz. 
                        \textbf{logična vezja}.
                \end{itemize}
            \end{alertblock}

            \begin{alertblock}{}
                Vrednost sestavljene izjave izračunamo glede na vrednosti elementarnih 
                izjav in izjavnih povezav med njimi.
            \end{alertblock}
            \begin{alertblock}{}
                Pravilnost sestavljenih izjav nazorno prikazujejo 
                \textbf{resničnostne}/\textbf{pravilnostne tabele}.
            \end{alertblock}

        \end{frame}

        \begin{frame}
            \frametitle{Izjavne povezave}

            \begin{alertblock}{Negacija}
                \textbf{Negacija} izjave $A$ je izjava, ki \textbf{trdi nasprotno} 
                kot izjava $A$.
                Oznaka: $\mathbf{\lnot A}$.
                $$ \mathbf{\lnot A} \quad \quad \textmd{\textbf{Ni res}, da velja izjava A.}$$
            \end{alertblock}

            \begin{columns}
                \column{0.65\textwidth} 
                    \begin{alertblock}{}
                        Če je izjava $A$ pravilna, je $\lnot A$ nepravilna in obratno: 
                        če je $\lnot A$ pravilna, je $A$ nepravilna.
                    \end{alertblock}
                    \begin{alertblock}{}
                        Negacija negacije izjave je potrditev izjave. \quad $\lnot(\lnot A)=A$
                    \end{alertblock}

                \column{0.3\textwidth} 
                \begin{table}
                    \centering
                    \begin{tabular}{||c|c||} 
                    \hhline{|t:==:t|}
                    \rowcolor[rgb]{0.843,0.718,0.718} $A$ & $\lnot A$  \\ 
                    \hhline{|:==:|}
                    $P$                                   & $N$                       \\ 
                    \hline
                    $N$                                   & $P$                       \\
                    \hhline{|b:==:b|}
                    \end{tabular}                    
                \end{table}                

            \end{columns}
        \end{frame}

        \begin{frame}
            \begin{alertblock}{Konjunkcija}
                \textbf{Konjunkcija} izjav $A$ in $B$ nastane tako, da povežemo izjavi $A$ in $B$ 
                z \textbf{in hkrati}.
                $$ \mathbf{A\land B} \quad \quad \textmd{Velja izjava A \textbf{in hkrati} izjava B.}$$
            \end{alertblock}
            \begin{columns}
                \column{0.6\textwidth} 
                    \begin{alertblock}{}
                        Če sta izjavi $A$ in $B$ pravilni, je pravilna tudi njuna konjunkcija, 
                        če je pa ena od izjav nepravilna, je nepravilna tudi njuna konjunkcija.
                    \end{alertblock}

                \column{0.35\textwidth} 
                    \begin{table}
                        \centering
                        \begin{tabular}{||c|c|c||} 
                        \hhline{|t:===:t|}
                        \rowcolor[rgb]{0.843,0.718,0.718} $A$ & $B$ & $A\land B$  \\ 
                        \hhline{|:===:|}
                        $P$ & $P$ & $P$                         \\ 
                        \hline
                        $P$ & $N$ & $N$                         \\ 
                        \hline
                        $N$ & $P$ & $N$                         \\ 
                        \hline
                        $N$ & $N$ & $N$                         \\
                        \hhline{|b:===:b|}
                        \end{tabular}
                    \end{table}

            \end{columns}


        \end{frame}

        \begin{frame}
            \begin{alertblock}{Disjunkcija}
                \textbf{Disjunkcija} izjav $A$ in $B$ nastane s povezavo \textbf{ali}.
                $$ \mathbf{A\lor B} \quad \quad \textmd{Velja izjava A \textbf{ali} izjava B 
                (lahko tudi obe hkrati).}$$
            \end{alertblock}
            \begin{columns}
                \column{0.62\textwidth} 
                    \begin{alertblock}{}
                        Disjunkcija je nepravilna, če sta nepravilni obe izjavi, ki jo sestavljata,
                        v preostalih treh primerih je pravilna.
                    \end{alertblock}

                \column{0.35\textwidth} 
                    \begin{table}
                        \centering
                        \begin{tabular}{||c|c|c||} 
                        \hhline{|t:===:t|}
                        \rowcolor[rgb]{0.843,0.718,0.718} $A$ & $B$ & $A\lor B$  \\ 
                        \hhline{|:===:|}
                        $P$ & $P$ & $P$                         \\ 
                        \hline
                        $P$ & $N$ & $P$                         \\ 
                        \hline
                        $N$ & $P$ & $P$                         \\ 
                        \hline
                        $N$ & $N$ & $N$                         \\
                        \hhline{|b:===:b|}
                        \end{tabular}
                    \end{table}

            \end{columns}


        \end{frame}

        \begin{frame}
            \begin{block}{Komutativnost konjunkcije in disjunkcije}
                $$ A\land B = B\land A  \quad \quad \quad
                 A\lor B = B\lor A$$
            \end{block}      
            
            \begin{block}{Asociativnost konjunkcije in disjunkcije}
                $$(A\land B)\land C = A\land(b\land C) \quad \quad \quad
                (A\lor B)\lor C = A\lor (B\lor C) $$
            \end{block}

            \begin{block}{Distributivnost zakona za konjunkcijo in disjunkcijo}
                $$(A\lor B)\land C = (A\land C)\lor(B\land C) \quad \quad \quad
                (A\land B)\lor C = (A\lor C)\land(B\lor C) $$
            \end{block}

            \begin{block}{De Morganova zakona}
                \begin{itemize}
                    \item negacija konjunkcije je disjunkcija negacij: 
                        $\lnot(A\land B)=\lnot A\lor\lnot B$
                    \item negacija disjunkcije je konjunkcija negacij: 
                        $\lnot(A\lor B)=\lnot A\land\lnot B$
                \end{itemize}                
            \end{block}
        
        \end{frame}

        \begin{frame}
            \begin{alertblock}{Implikacija}
                \textbf{Implikacija} izjav $A$ in $B$ je sestavljena izjava, ki jo lahko beremo
                na različne načine.
                $$ \mathbf{A\Rightarrow B} \quad \quad \textmd{\textbf{Če} velja izjava A, 
                \textbf{potem} velja izjava B. / \textbf{Iz} A \textbf{sledi} B.}$$
                Izjava $A$ je \textbf{pogoj} ali \textbf{privzetek}, izjava $B$ pa 
                \textbf{(logična) posledica} izjave $A$.
            \end{alertblock}
            \begin{columns}
                \column{0.62\textwidth} 
                    \begin{alertblock}{}
                        Implikacija je nepravilna, ko je izjava $A$ pravilna, izjava $B$ pa 
                        nepravilna, v preostalih treh primerih je pravilna.
                    \end{alertblock}

                \column{0.35\textwidth} 
                    \begin{table}
                        \centering
                        \begin{tabular}{||c|c|c||} 
                        \hhline{|t:===:t|}
                        \rowcolor[rgb]{0.843,0.718,0.718} $A$ & $B$ & $A\Rightarrow B$  \\ 
                        \hhline{|:===:|}
                        $P$ & $P$ & $P$                         \\ 
                        \hline
                        $P$ & $N$ & $N$                         \\ 
                        \hline
                        $N$ & $P$ & $P$                         \\ 
                        \hline
                        $N$ & $N$ & $P$                         \\
                        \hhline{|b:===:b|}
                        \end{tabular}
                    \end{table}

            \end{columns}
        \end{frame}

        \begin{frame}
            \begin{alertblock}{Ekvivalenca}
                \textbf{Ekvivalenca} izjavi $A$ in $B$ poveže s \textbf{če in samo če} oz.
                \textbf{natanko tedaj, ko}.
                \begin{align*} 
                    \mathbf{A\Leftrightarrow B} \quad \quad &\textmd{Izjava A velja, \textbf{če in
                    samo če} velja izjava B.} / \\
                        &\textmd{Izjava A velja \textbf{natanko tedaj, ko} velja izjava B.}
                \end{align*}
            \end{alertblock}


            \begin{columns}
                \column{0.62\textwidth} 
                    \begin{alertblock}{}
                        Ekvivalenca dveh izjav je pravilna, če imata obe izjavi enako vrednost 
                        (ali sta obe pravilni ali obe nepravilni), in nepravilna, če imata izjavi
                        različno vrednost.
                    \end{alertblock}
                    \begin{block}{}
                        Ekvivalentni/enakovredni izjavi pomenita eno in isto, lahko ju nadomestimo 
                        drugo z drugo.
                    \end{block}

                \column{0.35\textwidth} 
                    \begin{table}
                        \centering
                        \begin{tabular}{||c|c|c||} 
                        \hhline{|t:===:t|}
                        \rowcolor[rgb]{0.843,0.718,0.718} $A$ & $B$ & $A\Leftrightarrow B$  \\ 
                        \hhline{|:===:|}
                        $P$ & $P$ & $P$                         \\ 
                        \hline
                        $P$ & $N$ & $N$                         \\ 
                        \hline
                        $N$ & $P$ & $N$                         \\ 
                        \hline
                        $N$ & $N$ & $P$                         \\
                        \hhline{|b:===:b|}
                        \end{tabular}
                    \end{table}

            \end{columns}
        \end{frame}


        \begin{frame}
            \begin{block}{Vrstni red operacij}
                Kadar so izjave povezane z več izjavnimi povezavami, pri določanju logične 
                vrednosti upoštevamo oklepaje in naslednji \textbf{vrstni red} oz.~ 
                \textbf{prioriteto izjavnih povezav}:
                \begin{enumerate}
                    \item negacija,
                    \item konjunkcija,
                    \item disjunkcija,
                    \item implikacija,
                    \item ekvivalenca.
                \end{enumerate}
            \end{block}
            \begin{block}{}
                Če moramo zapored izvesti več enakih izjavnih povezav, velja pravilo združevanja 
                od leve proti desni.
            \end{block}
        \end{frame}

        \begin{frame}
            \begin{alertblock}{Tavtologija}
                \textbf{Tavtologija} ali \textbf{logično pravilna izjava} je sestavljena izjava, 
                ki je pri vseh naborih vrednosti elementarnih izjav, iz katerih je sestavjena, pravilna.
            \end{alertblock}

            \begin{alertblock}{Protislovje}
                \textbf{Protislovje} je sestavljena izjava, ki ni nikoli pravilna.
            \end{alertblock}

            \begin{block}{Kvantifikatorja}
                \begin{itemize}
                    \item $\forall$ (beri 'vsak') -- izjava velja za vsak element dane množice
                    \item $\exists$ (beri 'obstaja' ali 'eksistira') -- izjava je pravilna za vsaj en element dane množice
                \end{itemize}
            \end{block}

        \end{frame}

        \begin{frame}
            \frametitle{Pomen izjav v matematiki}
            \begin{block}{}
                \textbf{Aksiomi} so najpreprostejše izjave, ki so očitno pravilne in zato njihove 
                pravilnosti ni treba dokazovati.
            \end{block}
            \begin{block}{}
                \textbf{Izreki} ali \textbf{teoremi} so izjave, ki so pravilne, vendar pa njihova 
                pravilnost ni očitna. 
                Pravilnost izreka (teorema) moramo potrditi z dokazom, ki temelji na aksiomih in na 
                preprostejših že prej dokazanih izrekih.
            \end{block}
            \begin{block}{}
                \textbf{Definicije} so izjave, s katerimi uvajamo nove pojme. Najpreprostejših pojmov 
                v matematiki ne opisujemo z definicijami (to so pojmi kot npr.: število, premica ipd.); 
                vsak nadaljnji pojem pa moramo definirati, zato da se nedvoumno ve, o čem govorimo.
            \end{block}

        \end{frame}
        
    \subsection{Množice}

        \begin{frame}
            \frametitle{Množice}
        \end{frame}