\documentclass[11pt,a4paper]{book}

\usepackage[T1]{fontenc}
\usepackage[utf8]{inputenc}
\usepackage[slovene]{babel}
\usepackage{lmodern}
% \usepackage[colorlinks]{hyperref}

\usepackage{xcolor}
\usepackage{amsfonts}
\usepackage{amssymb}
\usepackage{amsmath}
\usepackage{amsthm}
\usepackage{bbold}
\usepackage{fancyhdr}
\usepackage{mathpartir}
\usepackage{proof}
% \usepackage{xypic}
\usepackage[all,cmtip]{xy}
\usepackage{tikz}
\usepackage{tkz-graph}
\usepackage{circuitikz}
\usepackage{booktabs}
\usepackage{enumitem}
\setlist{nosep}

\usepackage{float}
% \usepackage{gensymb}
\usetikzlibrary{calc}
\usetikzlibrary{through}
\usepackage{ulem}
\usepackage{graphicx}
\usepackage{mathtools}
\usepackage{multicol}
\usepackage{mathrsfs}
\usepackage{mathabx}
% \usepackage{answers}
% \usepackage{xparse}
\usepackage{arcs}
% \usepackage{emptypage}
\usepackage{wrapfig}
\usepackage{colortbl}
\usepackage{hhline}
\usepackage{subcaption}

\setlength\parindent{0pt}

\usepackage[papersize={210mm,297mm}, % A4
            twoside,
            includehead,
            top=20mm, % margina na vrhu strani
            bottom=20mm, % margina na dnu strani
            inner=20mm, % margina na notranji strani strani
            outer=20mm, % margina na zunanji strani strani
            bindingoffset=10mm % dodatna margina na notranji strani
           ]{geometry}



% %-- Okolja

{
  % \theoremstyle{plain}
  % \theorembodyfont{\itshape}
  \newtheorem{izrek}{Izrek}[chapter]
  \newtheorem{lema}[izrek]{Lema}
  % \newtheorem{izjava}[izrek]{Izjava}
  \newtheorem{posledica}[izrek]{Posledica}
  % \newtheorem{hipoteza}[izrek]{Hipoteza}
  \newtheorem{aksiom}[izrek]{Aksiom}
  \newtheorem{trditev}[izrek]{Trditev}
% }

% {
%   \theorembodyfont{\rmfamily}
  \newtheorem{definicija}[izrek]{Definicija}
  % \newtheorem{primer}[izrek]{Primer}
  \newtheorem{opomba}[izrek]{Opomba}
  \newtheorem{naloga}[izrek]{Naloga}
}

\newcommand{\qedsign}{{\vrule width 1ex height 1ex depth 0ex}}
% % \newcommand{\qed}{\hfill\qedsign}

\newenvironment{dokaz}{
  \goodbreak\par
  \textit{Dokaz.}%
}{%
  \nopagebreak
  \qed
  \medskip
  \goodbreak
}

%--------------------------------------------------------------------
% \input{macros.tex}

\newcommand{\NN}{\mathbb{N}}
\newcommand{\ZZ}{\mathbb{Z}}
\newcommand{\QQ}{\mathbb{Q}}
\newcommand{\RR}{\mathbb{R}}
\newcommand{\CC}{\mathbb{C}}
\newcommand{\HH}{\mathbb{H}}
\newcommand{\OO}{\mathbb{O}}

%--------------------------------------------------------------------
\makeatletter
\DeclareFontFamily{U}{tipa}{}
\DeclareFontShape{U}{tipa}{m}{n}{<->tipa10}{}
\newcommand{\arc@char}{{\usefont{U}{tipa}{m}{n}\symbol{62}}}%

\newcommand{\arc}[1]{\mathpalette\arc@arc{#1}}

\newcommand{\arc@arc}[2]{%
  \sbox0{$\m@th#1#2$}%
  \vbox{
    \hbox{\resizebox{\wd0}{\height}{\arc@char}}
    \nointerlineskip
    \box0
  }%
}
\makeatother


%--------------------------------------------------------------------
%-- Glava in dno

\pagestyle{fancyplain}

\setlength{\headheight}{15pt}
\renewcommand{\chaptermark}[1]{\markboth{\textsc{\thechapter. #1}}{}}
\renewcommand{\sectionmark}[1]{\markright{\textsc{\thesection\ #1}}}

\lhead[\fancyplain{}{{\thepage}}]%
      {\fancyplain{}{\nouppercase{\rightmark}}}
\rhead[\fancyplain{}{\nouppercase{\leftmark}}]%
      {\fancyplain{}{\thepage}}
\lfoot[]{}
\cfoot[]{}
\rfoot[]{}

%%%%%%%%%%%%%%%%%%%%%%%%%%%%%%%%%%%%%%%%%


%--------------------------------------------------------------------
% NASLOV

\author{}
\title{\Huge{Matematika \\ Splošna gimnazija} \\ ~\\\Large{ZAPISKI}}

\begin{document}

\frontmatter
\maketitle

\newpage
\thispagestyle{empty}
~

\newpage

\mbox{}
\vfill
Pred vami so zapiski za predmet Matematika v splošnem gimnazijskem izobraževanju. 
Sproti bodo nastajali od šolskega leta 2024/2025 naprej.
V besedilu so mogoče prisotne še kake napake. Če kakšno opazite, mi javite
\bigskip
\begin{flushright}
  Jan Kastelic \qquad \qquad \qquad\qquad\qquad\hbox{} \\
\end{flushright}

\newpage
\thispagestyle{empty}
~

%--------------------------------------------------------------------
% KAZALO
\tableofcontents

\newpage
\thispagestyle{empty}
~

%--------------------------------------------------------------------
% VSEBINA
\mainmatter

\chapter{Osnove logike}

      
             \section{Izjave}

                \textbf{Matematična izjava} je vsaka smiselna poved, za katero 
                lahko določimo resničnost oziroma pravilnost.

                 
                Matematična izjava lahko zavzame dve logični vrednosti:
                \begin{itemize}
                    \item izjava je \textbf{resnična}/\textbf{pravilna}, 
                        oznaka $\mathbf{R}$/$\mathbf{P}$/$\mathbf{1}$/$\mathbf{\top}$;
                    \item izjava je \textbf{neresnična}/\textbf{nepravilna}, 
                        oznaka $\mathbf{N}$/$\mathbf{0}$/$\mathbf{\bot }$.
                \end{itemize}                

                 
                Izjave označujemo z velikimi tiskanimi črkami ($A$, $B$, $C$ ...).
             
         

         
             \begin{naloga}
                Ali so naslednje povedi izjave?
                \begin{itemize}   
                    \item Danes sije sonce.
                    \item Koliko je ura?
                    \item Piramida je geometrijski lik.
                    \item Daj mi jabolko.
                    \item Število $12$ deli število $3$.
                    \item Število $3$ deli število $10$.
                    \item Ali si pisal matematični test odlično?
                    \item Matematični test si pisal odlično.
                    \item Ali je $10~dl$ isto kot $1~l$?
                    \item Število $41$ je praštevilo.
                \end{itemize}
            \end{naloga}
             
         

         
             \begin{naloga}
                Spodnjim izjavam določite logične vrednosti.
                \begin{itemize}   
                    \item $A$: Najvišja gora v Evropi je Mont Blanc.
                    \item $B$: Število je deljivo s $4$ natanko takrat, ko je vsota števk deljiva s $4$.
                    \item $C$: Ostanek pri deljenju s $4$ je lahko $1$, $2$ ali $3$.
                    \item $D$: Mesec februar ima 28 dni.
                    \item $E$: Vsa praštevila so liha števila.
                    \item $F$: Število $1$ je naravno število.
                    \item $G$: Praštevil je neskončno mnogo.
                \end{itemize}
            \end{naloga}
            
         

         
              \subsection{Enostavne in sestavjene izjave}
                
                Izjave delimo med:
                \begin{itemize}
                    \item \textbf{elementarne}/\textbf{enostavne izjave} -- ne moremo 
                        jih razstaviti na bolj enostavne;
                    \item \textbf{sestavljene izjave} -- sestavljene iz elementarnih izjav, 
                        ki jih med seboj povezujejo \textbf{logične operacije} (imenovane 
                        tudi izjavne povezave oziroma~ logična vezja).
                \end{itemize}
             

               
                Vrednost sestavljene izjave izračunamo glede na vrednosti elementarnih 
                izjav in izjavnih povezav med njimi.
             
               
                Pravilnost sestavljenih izjav nazorno prikazujejo 
                \textbf{resničnostne}/\textbf{pravilnostne tabele}.
             

         

         
             \section{Logične operacije}

              \subsection{Negacija}
                \textbf{Negacija} izjave $A$ je izjava, ki \textbf{trdi nasprotno} 
                kot izjava $A$.
                Oznaka: $\mathbf{\lnot A}$.
                $$ \mathbf{\lnot A} \quad \quad \textmd{\textbf{Ni res}, da velja izjava A.}$$
             

            % \begin{columns}
            %     \column{0.65\textwidth} 
                      
                        Če je izjava $A$ pravilna, je $\lnot A$ nepravilna in obratno: 
                        če je $\lnot A$ pravilna, je $A$ nepravilna.
                     
                      
                        Negacija negacije izjave je potrditev izjave. \quad $\lnot(\lnot A)=A$
                     

                % \column{0.3\textwidth} 
                \begin{table}[H]
                    \centering
                    \begin{tabular}{||c|c||} 
                    \hhline{|t:==:t|}
                    \rowcolor[rgb]{0.843,0.718,0.718} $A$ & $\lnot A$  \\ 
                    \hhline{|:==:|}
                    $P$                                   & $N$                       \\ 
                    \hline
                    $N$                                   & $P$                       \\
                    \hhline{|b:==:b|}
                    \end{tabular}                    
                \end{table}                

            % \end{columns}
         

         
             \begin{naloga}
                Izjavam določite logično vrednost, potem jih zanikajte in določite logično vrednost negacij.
                \begin{itemize}
                    \item $A$: $5 \cdot 8 = 30$
                    \item $B$: Število $3$ je praštevilo.
                    \item $C$: Največje dvomestno število je $99$.
                    \item $D$: Število $62$ je večratnik števila $4$.
                    \item $E$: Praštevil je neskončno mnogo.
                    \item $F$: $7 \leq 5$
                    \item $G$: Naša pisava je cirilica.
                \end{itemize}
            \end{naloga}
         

         
             \subsection{Konjunkcija}
                \textbf{Konjunkcija} izjav $A$ in $B$ nastane tako, da povežemo izjavi $A$ in $B$ 
                z \textbf{in hkrati}.
                $$ \mathbf{A\land B} \quad \quad \textmd{Velja izjava A \textbf{in (hkrati)} izjava B.}$$
             
            % \begin{columns}
                % \column{0.6\textwidth} 
                    %   
                        Če sta izjavi $A$ in $B$ pravilni, je pravilna tudi njuna konjunkcija, 
                        če je pa ena od izjav nepravilna, je nepravilna tudi njuna konjunkcija.
                     

                % \column{0.35\textwidth} 
                    \begin{table}[H]
                        \centering
                        \begin{tabular}{||c|c|c||} 
                        \hhline{|t:===:t|}
                        \rowcolor[rgb]{0.843,0.718,0.718} $A$ & $B$ & $A\land B$  \\ 
                        \hhline{|:===:|}
                        $P$ & $P$ & $P$                         \\ 
                        \hline
                        $P$ & $N$ & $N$                         \\ 
                        \hline
                        $N$ & $P$ & $N$                         \\ 
                        \hline
                        $N$ & $N$ & $N$                         \\
                        \hhline{|b:===:b|}
                        \end{tabular}
                    \end{table}

            % \end{columns}


         

         
             \begin{naloga}
                Določite logično vrednost konjunkcijam.
                \begin{itemize}
                    \item Število $28$ je večratnik števila $3$ in večkratnik števila $8$.
                    \item Število $7$ je praštevilo in je deljivo s številom $1$.
                    \item Vsakemu celemu številu lahko pripišemo nasprotno število in obratno celo število.
                    \item Ostanki pri deljenju števila s $3$ so lahko $0$, $1$ ali $2$, 
                        pri deljenju s $5$ pa $0$, $1$, $2$, $3$ ali $4$.
                    \item Število je deljivo s $3$, če je vsota števk deljiva s $3$, in je 
                        deljivo z $9$, če je vsota števk deljiva z $9$.
                \end{itemize}
            \end{naloga}
         

         
             \subsection{Disjunkcija}
                \textbf{Disjunkcija} izjav $A$ in $B$ nastane s povezavo \textbf{ali}.
                $$ \mathbf{A\lor B} \quad \quad \textmd{Velja izjava A \textbf{ali} izjava B 
                (lahko tudi obe hkrati).}$$
             
            % \begin{columns}
                % \column{0.62\textwidth} 
                      
                        Disjunkcija je nepravilna, če sta nepravilni obe izjavi, ki jo sestavljata,
                        v preostalih treh primerih je pravilna.
                     

                % \column{0.35\textwidth} 
                    \begin{table}[H]
                        \centering
                        \begin{tabular}{||c|c|c||} 
                        \hhline{|t:===:t|}
                        \rowcolor[rgb]{0.843,0.718,0.718} $A$ & $B$ & $A\lor B$  \\ 
                        \hhline{|:===:|}
                        $P$ & $P$ & $P$                         \\ 
                        \hline
                        $P$ & $N$ & $P$                         \\ 
                        \hline
                        $N$ & $P$ & $P$                         \\ 
                        \hline
                        $N$ & $N$ & $N$                         \\
                        \hhline{|b:===:b|}
                        \end{tabular}
                    \end{table}

            % \end{columns}


         

         
             \begin{naloga}
                Določite logično vrednost disjunkcijam.
                \begin{itemize}
                    \item Število $24$ je večratnik števila $3$ ali $8$.
                    \item Število $35$ ni večratnik števila $7$ ali $6$.
                    \item Število $5$ deli število $16$ ali $18$.
                    \item Ploščina kvadrata s stranico $a$ je $a^2$ ali obseg kvadrata je $4a$.
                    \item Ni res, da je vsota notranjih kotov trikotnika $160^\circ$, ali ni res, 
                        da Pitagorov izrek velja v poljubnem trikotniku.
                \end{itemize}
            \end{naloga}
         


         
             \subsection{Komutativnost konjunkcije in disjunkcije}
                $$ A\land B = B\land A $$
                $$ A\lor B = B\lor A$$
                   
            
             \subsection{Asociativnost konjunkcije in disjunkcije}
                $$(A\land B)\land C = A\land(B\land C) $$
                $$ (A\lor B)\lor C = A\lor (B\lor C) $$
             

             \subsection{Distributivnost zakona za konjunkcijo in disjunkcijo}
                $$(A\lor B)\land C = (A\land C)\lor(B\land C) $$
                $$ (A\land B)\lor C = (A\lor C)\land(B\lor C) $$
             

             \subsection{De Morganova zakona}
                \begin{itemize}
                    \item negacija konjunkcije je disjunkcija negacij: 
                        $\lnot(A\land B)=\lnot A\lor\lnot B$
                    \item negacija disjunkcije je konjunkcija negacij: 
                        $\lnot(A\lor B)=\lnot A\land\lnot B$
                \end{itemize}                
             
        
         

         
             \begin{naloga}
                Katere od spodnjih izjav so pravilne in katere nepravilne?
                \begin{itemize}
                    \item $(3\cdot 4 = 12)\land(12:4=3)$
                    \item $(a^3\cdot a^5=a^{15})\lor (a^3\cdot a^5=a^8)$
                    \item $(3|30)\land(3|26)$
                    \item $(3|30)\lor(3|26)$
                    \item $(2^3=9)\lor(3^2=9)$
                    \item $((-2)^2=4)\land\lnot(-2^2=4)$
                \end{itemize}
            \end{naloga}
         

         
             \subsection{Implikacija}
                \textbf{Implikacija} izjav $A$ in $B$ je sestavljena izjava, ki jo lahko beremo
                na različne načine.
                $$ \mathbf{A\Rightarrow B} \quad \quad \textmd{\textbf{Če} velja izjava A, 
                \textbf{potem} velja izjava B. / \textbf{Iz} A \textbf{sledi} B.}$$
                Izjava $A$ je \textbf{pogoj} ali \textbf{privzetek}, izjava $B$ pa 
                \textbf{(logična) posledica} izjave $A$.
             
            % \begin{columns}
                % \column{0.62\textwidth} 
                      
                        Implikacija je nepravilna, ko je izjava $A$ pravilna, izjava $B$ pa 
                        nepravilna, v preostalih treh primerih je pravilna.
                     

                % \column{0.35\textwidth} 
                    \begin{table}[H]
                        \centering
                        \begin{tabular}{||c|c|c||} 
                        \hhline{|t:===:t|}
                        \rowcolor[rgb]{0.843,0.718,0.718} $A$ & $B$ & $A\Rightarrow B$  \\ 
                        \hhline{|:===:|}
                        $P$ & $P$ & $P$                         \\ 
                        \hline
                        $P$ & $N$ & $N$                         \\ 
                        \hline
                        $N$ & $P$ & $P$                         \\ 
                        \hline
                        $N$ & $N$ & $P$                         \\
                        \hhline{|b:===:b|}
                        \end{tabular}
                    \end{table}

            % \end{columns}
         

         
             \begin{naloga}
                Določite, ali so izjave pravilne.
                \begin{itemize}
                    \item Če je število deljivo s $100$, je deljivo tudi s $4$.
                    \item Če je štirikotnik pravokotnik, se diagonali razpolavljata.
                    \item Če je štirikotnik kvadrat, se diagonali sekata pod pravim kotom.
                    \item Če sta števili $2$ in $3$ lihi števili, potem je produk teh dveh števil sodo število.
                    \item Če je število $18$ deljivo z $9$, potem je deljivo s $3$.
                    \item Če je $7$ večkratnik števila $7$, potem $7$ deli število $43$.
                    \item Če je število deljivo s $4$, potem je deljivo z $2$.
                \end{itemize}
            \end{naloga}
         
         
             \subsection{Ekvivalenca}
                \textbf{Ekvivalenca} izjavi $A$ in $B$ poveže s \textbf{če in samo če} oziroma
                \textbf{natanko tedaj, ko}.
                \begin{align*} 
                    \mathbf{A\Leftrightarrow B} \quad \quad &\textmd{Izjava A velja, \textbf{če in
                    samo če} velja izjava B.} / \\
                        &\textmd{Izjava A velja \textbf{natanko tedaj, ko} velja izjava B.}
                \end{align*}
             


            % \begin{columns}
                % \column{0.62\textwidth} 
                      
                        Ekvivalenca dveh izjav je pravilna, če imata obe izjavi enako vrednost 
                        (ali sta obe pravilni ali obe nepravilni), in nepravilna, če imata izjavi
                        različno vrednost.
                     
                      
                        Ekvivalentni/enakovredni izjavi pomenita eno in isto, lahko ju nadomestimo 
                        drugo z drugo.
                     

                % \column{0.35\textwidth} 
                    \begin{table}[H]
                        \centering
                        \begin{tabular}{||c|c|c||} 
                        \hhline{|t:===:t|}
                        \rowcolor[rgb]{0.843,0.718,0.718} $A$ & $B$ & $A\Leftrightarrow B$  \\ 
                        \hhline{|:===:|}
                        $P$ & $P$ & $P$                         \\ 
                        \hline
                        $P$ & $N$ & $N$                         \\ 
                        \hline
                        $N$ & $P$ & $N$                         \\ 
                        \hline
                        $N$ & $N$ & $P$                         \\
                        \hhline{|b:===:b|}
                        \end{tabular}
                    \end{table}

            % \end{columns}
         

         
             \begin{naloga}
                Določite, ali so naslednje izjave pravilne.
                \begin{itemize}
                    \item Število je deljivo z $12$ natanko takrat, ko je deljivo s $3$ in $4$ hkrati.
                    \item Število je deljivo s $24$ natanko takrat, ko je deljivo s $4$ in $6$ hkrati.
                    \item Število je praštevilo natanko takrat, ko ima natanko dva delitelja.
                    \item Štirikotnik je kvadrat natanko tedaj, ko se diagonali sekata pod pravim kotom.
                    \item Število je sodo natanko tedaj, ko je deljivo z $2$.
                \end{itemize}
            \end{naloga}
         

         
             \subsection{Vrstni red operacij}
                Kadar so izjave povezane z več izjavnimi povezavami, pri določanju logične 
                vrednosti upoštevamo oklepaje in naslednji \textbf{vrstni red} oziroma
                \textbf{prioriteto izjavnih povezav}:
                \begin{enumerate}
                    \item negacija,
                    \item konjunkcija,
                    \item disjunkcija,
                    \item implikacija,
                    \item ekvivalenca.
                \end{enumerate}
             
              
                Če moramo zapored izvesti več enakih izjavnih povezav, velja pravilo združevanja 
                od leve proti desni.
             
         

         
             \begin{naloga}
                V sestavljeni izjavi zapišite oklepaje, ki bodo predstavljali vrstni red operacij.
                Nato tvorite pravilnostno tabelo za sestavljeno izjavo glede na različne logične 
                vrednosti elementarnih izjav.
                \begin{itemize}
                    \item $A\lor B\Leftrightarrow \lnot A\Rightarrow \lnot B$
                    \item $A\lor \lnot A\Rightarrow\lnot B\land (\lnot A\Rightarrow B) $
                    \item $A\Rightarrow B\Leftrightarrow \lnot B\Rightarrow \lnot A $
                    \item $A\land \lnot B\Leftrightarrow A \Rightarrow B$
                    \item $C\Rightarrow A\lor \lnot B\Leftrightarrow \lnot A\land C $
                    \item $\lnot A\lor \lnot B\Leftrightarrow B \land (C\Leftrightarrow \lnot A) $
                \end{itemize}
            \end{naloga}
         

         
             \subsection{Tavtologija in protislovje}

                \textbf{Tavtologija} ali \textbf{logično pravilna izjava} je sestavljena izjava, 
                ki je pri vseh naborih vrednosti elementarnih izjav, iz katerih je sestavjena, pravilna.
             
                \textbf{Protislovje} je sestavljena izjava, ki ni nikoli pravilna.
             

             \subsection{Kvantifikatorja}
                \begin{itemize}
                    \item $\forall$ (beri 'vsak') -- izjava velja za vsak element dane množice
                    \item $\exists$ (beri 'obstaja' ali 'eksistira') -- izjava je pravilna za vsaj en element dane množice
                \end{itemize}
             

         

         
             \section{Pomen izjav v matematiki}
              
                \textbf{Aksiomi} so najpreprostejše izjave, ki so očitno pravilne in zato njihove 
                pravilnosti ni treba dokazovati.
             
              
                \textbf{Izreki} ali \textbf{teoremi} so izjave, ki so pravilne, vendar pa njihova 
                pravilnost ni očitna. 
                Pravilnost izreka (teorema) moramo potrditi z dokazom, ki temelji na aksiomih in na 
                preprostejših že prej dokazanih izrekih.
             
              
                \textbf{Definicije} so izjave, s katerimi uvajamo nove pojme. Najpreprostejših pojmov 
                v matematiki ne opisujemo z definicijami (to so pojmi kot npr.: število, premica ipd.); 
                vsak nadaljnji pojem pa moramo definirati, zato da se nedvoumno ve, o čem govorimo.
             

         
        
    % \chapter{Osnove teorije množic}

         
         

\section{Naravna in cela števila}

\begin{frame}
    \sectionpage
\end{frame}

\begin{frame}
    \tableofcontents[currentsection, hideothersubsections]
\end{frame}
        
    \subsection{Naravna števila}

        \begin{frame}
            \frametitle{Naravna števila}

                \only<2->{\begin{alertblock}{Množica naravnih števil}
                    \only<3->{\textbf{Naravna števila} so števila s katerimi štejemo.}
                    \only<4->{$$\mathbf{\mathbb{N}=\{1, 2, 3, 4, \ldots\}}$$}
                \end{alertblock}}

                \only<5->{\begin{block}{}
                    Množico naravnih števil definirajo \textbf{Peanovi aksiomi}:
                    \begin{enumerate}
                        \item<6-> Vsako naravno število $n$ ima svojega \textbf{naslednika} $n+1$.
                        \item<7-> Število $1$ je naravno število, ki ni naslednik nobenega naravnega števila.
                        \item<8-> Različni naravni števili imata različna naslednika: $n+1 \neq m+1; n \neq m$.
                        \item<9-> Če neka trditev velja z vsakim naravnim številom tudi za njegovega naslednika, velja za vsa naravna števila. (\textit{aksiom/princip popolne indukcije})
                    \end{enumerate}

                \end{block}}
        \end{frame}

        \begin{frame}

            \only<2->{\begin{block}{}
                Naravna števila uredimo po velikosti in predstavimo s \textbf{točko} na \textbf{številski premici}.
                \only<3->{\begin{figure}
                    \begin{tikzpicture}
                        % \clip (0,0) rectangle (14.000000,10.000000);
                        {\footnotesize
                        
                        % Drawing segment a b
                        \draw [line width=0.016cm] (0.500000,0.500000) -- (9.000000,0.500000);%
                        
                        % Drawing arrow a b 1.00
                        \draw [line width=0.016cm] (8.702567,0.539158) -- (9.000000,0.500000);%
                        \draw [line width=0.016cm] (8.702567,0.539158) -- (8.900856,0.500000);%
                        \draw [line width=0.016cm] (8.702567,0.460842) -- (9.000000,0.500000);%
                        \draw [line width=0.016cm] (8.702567,0.460842) -- (8.900856,0.500000);%
                        
                        % Drawing segment c d
                        \draw [line width=0.016cm] (1.000000,0.350000) -- (1.000000,0.650000);%
                        
                        % Drawing segment e f
                        \draw [line width=0.016cm] (2.000000,0.350000) -- (2.000000,0.650000);%
                        
                        % Drawing segment g h
                        \draw [line width=0.016cm] (3.000000,0.350000) -- (3.000000,0.650000);%
                        
                        % Drawing segment i j
                        \draw [line width=0.016cm] (4.000000,0.350000) -- (4.000000,0.650000);%
                        
                        % Drawing segment k l
                        \draw [line width=0.016cm] (5.000000,0.350000) -- (5.000000,0.650000);%
                        
                        % Drawing segment m n
                        \draw [line width=0.016cm] (6.000000,0.350000) -- (6.000000,0.650000);%
                        
                        % Drawing segment o p
                        \draw [line width=0.016cm] (7.000000,0.350000) -- (7.000000,0.650000);%
                        
                        % Drawing segment r s
                        \draw [line width=0.016cm] (8.000000,0.350000) -- (8.000000,0.650000);%
                        
                        % Marking point O
                        \draw (1.000000,0.600000) node [anchor=south] { $O$ };%
                        
                        % Marking point E
                        \draw (2.000000,0.600000) node [anchor=south] { $E$ };%
                        
                        % Marking point 1
                        \draw (2.000000,0.400000) node [anchor=north] { $1$ };%
                        
                        % Marking point 2
                        \draw (3.000000,0.400000) node [anchor=north] { $2$ };%
                        
                        % Marking point 3
                        \draw (4.000000,0.400000) node [anchor=north] { $3$ };%
                        
                        % Marking point 4
                        \draw (5.000000,0.400000) node [anchor=north] { $4$ };%
                        
                        % Marking point 5
                        \draw (6.000000,0.400000) node [anchor=north] { $5$ };%
                        
                        % Marking point 6
                        \draw (7.000000,0.400000) node [anchor=north] { $6$ };%
                        
                        % Marking point 7
                        \draw (8.000000,0.400000) node [anchor=north] { $7$ };%
                        
                        % Marking point {enota}
                        \draw (1.500000,0.600000) node [anchor=north] { ${enota}$ };%
                        }
                    \end{tikzpicture}
                        
                \end{figure}}

            \end{block}}

            \only<4->{\begin{block}{}
                Vsako število zapišemo s \textbf{številko}. 
                Za zapis številke uporabljamo \textbf{števke}. Te so $0, 1, 2, 3, 4, 5, 6, 7, 8, 9$.
            \end{block}}

            \only<5->{\begin{block}{}
                Posamezne števke večmestnega števila od desne proti levi predstavljajo: \textbf{enice}, \textbf{desetice}, \textbf{stotice}, \textbf{tisočice}, ...
            \end{block}}

            \only<6->{\begin{block}{}
                Število, ki je zapisano s črkovnimi oznakami števk označimo s črto nad zapsiom črkovne oznake.
                \only<7->{$$ \overline{xy}=10x+y \quad \quad \quad \overline{xyz}=100x+10y+z$$}
            \end{block}}

            
        \end{frame}

        \begin{frame}
            \frametitle{Operacije v množici $\mathbb{N}$}

            \only<2->{\begin{alertblock}{Seštevanje}
                \only<3->{Poljubnima naravnima številoma $x$ in $y$ priredimo \textbf{vsoto} $\mathbf{x+y}$.}
            \end{alertblock}}

            \only<4->{\begin{block}{}
                Število $x$ oziroma $y$ imenujemo \textbf{seštevanec} ali \textbf{sumand} ali \textbf{člen}. 

                \only<5->{Število $x+y$ pa imenujemo \textbf{vsota} ali \textbf{summa}. }

            \only<6->{\begin{figure}
                \begin{tikzpicture}
                    % \clip (0,0) rectangle (14.000000,10.000000);
                    {\footnotesize
                    
                    % Drawing segment a b
                    \draw [line width=0.016cm] (5.000000,0.500000) -- (5.000000,2.500000);%
                    
                    % Drawing segment b d
                    \draw [line width=0.016cm] (5.000000,2.500000) -- (7.000000,2.500000);%
                    
                    % Drawing segment c d
                    \draw [line width=0.016cm] (7.000000,0.500000) -- (7.000000,2.500000);%
                    
                    % Drawing segment c a
                    \draw [line width=0.016cm] (7.000000,0.500000) -- (5.000000,0.500000);%
                    
                    % Drawing segment e f
                    \draw [line width=0.016cm] (7.000000,1.500000) -- (9.000000,1.500000);%
                    
                    % Drawing arrow e f 1.00
                    \draw [line width=0.016cm] (8.702567,1.539158) -- (9.000000,1.500000);%
                    \draw [line width=0.016cm] (8.702567,1.539158) -- (8.900856,1.500000);%
                    \draw [line width=0.016cm] (8.702567,1.460842) -- (9.000000,1.500000);%
                    \draw [line width=0.016cm] (8.702567,1.460842) -- (8.900856,1.500000);%
                    
                    % Drawing segment g h
                    \draw [line width=0.016cm] (2.500000,1.000000) -- (5.000000,1.000000);%
                    
                    % Drawing segment i j
                    \draw [line width=0.016cm] (2.500000,2.000000) -- (5.000000,2.000000);%
                    
                    % Drawing arrow g h 1.00
                    \draw [line width=0.016cm] (4.702567,1.039158) -- (5.000000,1.000000);%
                    \draw [line width=0.016cm] (4.702567,1.039158) -- (4.900856,1.000000);%
                    \draw [line width=0.016cm] (4.702567,0.960842) -- (5.000000,1.000000);%
                    \draw [line width=0.016cm] (4.702567,0.960842) -- (4.900856,1.000000);%
                    
                    % Drawing arrow i j 1.00
                    \draw [line width=0.016cm] (4.702567,2.039158) -- (5.000000,2.000000);%
                    \draw [line width=0.016cm] (4.702567,2.039158) -- (4.900856,2.000000);%
                    \draw [line width=0.016cm] (4.702567,1.960842) -- (5.000000,2.000000);%
                    \draw [line width=0.016cm] (4.702567,1.960842) -- (4.900856,2.000000);%
                    
                    % Marking point {vsota}
                    \draw (8.000000,1.500000) node [anchor=south] { ${vsota}$ };%
                    
                    % Marking point {summa}
                    \draw (8.000000,1.500000) node [anchor=north] { ${summa}$ };%
                    
                    % Marking point {se�tevanec}
                    \draw (3.750000,1.000000) node [anchor=south] { ${seštevanec}$ };%
                    
                    % Marking point {sumand}
                    \draw (3.750000,1.000000) node [anchor=north] { ${sumand}$ };%
                    
                    % Marking point {se�tevanec}
                    \draw (3.750000,2.000000) node [anchor=south] { ${seštevanec}$ };%
                    
                    % Marking point {sumand}
                    \draw (3.750000,2.000000) node [anchor=north] { ${sumand}$ };%
                    
                    % Drawing segment x y
                    \draw [line width=0.032cm] (6.000000,1.000000) -- (6.000000,2.000000);%
                    
                    % Drawing segment z w
                    \draw [line width=0.032cm] (5.500000,1.500000) -- (6.500000,1.500000);%
                    }
                    \end{tikzpicture}
                    
            \end{figure}}


            \end{block}}


            \only<7->{\begin{block}{}
                Vsota naravnih števil je naravno število: $x, y \in \mathbb{N} \Rightarrow x+y \in \mathbb{N}$.

            \end{block}}

        \end{frame}

        \begin{frame}
            \only<2->{\begin{alertblock}{Množenje}
                \only<3->{Poljubnima naravnima številoma $x$ in $y$ priredimo \textbf{produkt} $\mathbf{x\cdot y}$.}
            \end{alertblock}}

            \only<4->{\begin{block}{}
                Število $x$ oziroma $y$ imenujemo \textbf{množenec} ali \textbf{faktor}. 

                \only<5->{Število $x\cdot y$ pa imenujemo \textbf{zmnožek} ali \textbf{produkt}. }

                \only<6->{\begin{figure}
                \begin{tikzpicture}
                    % \clip (0,0) rectangle (14.000000,10.000000);
                    {\footnotesize
                    
                    % Drawing segment a b
                    \draw [line width=0.016cm] (5.000000,0.500000) -- (5.000000,2.500000);%
                    
                    % Drawing segment b d
                    \draw [line width=0.016cm] (5.000000,2.500000) -- (7.000000,2.500000);%
                    
                    % Drawing segment c d
                    \draw [line width=0.016cm] (7.000000,0.500000) -- (7.000000,2.500000);%
                    
                    % Drawing segment c a
                    \draw [line width=0.016cm] (7.000000,0.500000) -- (5.000000,0.500000);%
                    
                    % Drawing segment e f
                    \draw [line width=0.016cm] (7.000000,1.500000) -- (9.000000,1.500000);%
                    
                    % Drawing arrow e f 1.00
                    \draw [line width=0.016cm] (8.702567,1.539158) -- (9.000000,1.500000);%
                    \draw [line width=0.016cm] (8.702567,1.539158) -- (8.900856,1.500000);%
                    \draw [line width=0.016cm] (8.702567,1.460842) -- (9.000000,1.500000);%
                    \draw [line width=0.016cm] (8.702567,1.460842) -- (8.900856,1.500000);%
                    
                    % Drawing segment g h
                    \draw [line width=0.016cm] (2.500000,1.000000) -- (5.000000,1.000000);%
                    
                    % Drawing segment i j
                    \draw [line width=0.016cm] (2.500000,2.000000) -- (5.000000,2.000000);%
                    
                    % Drawing arrow g h 1.00
                    \draw [line width=0.016cm] (4.702567,1.039158) -- (5.000000,1.000000);%
                    \draw [line width=0.016cm] (4.702567,1.039158) -- (4.900856,1.000000);%
                    \draw [line width=0.016cm] (4.702567,0.960842) -- (5.000000,1.000000);%
                    \draw [line width=0.016cm] (4.702567,0.960842) -- (4.900856,1.000000);%
                    
                    % Drawing arrow i j 1.00
                    \draw [line width=0.016cm] (4.702567,2.039158) -- (5.000000,2.000000);%
                    \draw [line width=0.016cm] (4.702567,2.039158) -- (4.900856,2.000000);%
                    \draw [line width=0.016cm] (4.702567,1.960842) -- (5.000000,2.000000);%
                    \draw [line width=0.016cm] (4.702567,1.960842) -- (4.900856,2.000000);%
                    
                    % Marking point {zmno�ek}
                    \draw (8.000000,1.500000) node [anchor=south] { ${zmnožek}$ };%
                    
                    % Marking point {produkt}
                    \draw (8.000000,1.500000) node [anchor=north] { ${produkt}$ };%
                    
                    % Marking point {mno�enec}
                    \draw (3.750000,1.000000) node [anchor=south] { ${množenec}$ };%
                    
                    % Marking point {faktor}
                    \draw (3.750000,1.000000) node [anchor=north] { ${faktor}$ };%
                    
                    % Marking point {mno�enec}
                    \draw (3.750000,2.000000) node [anchor=south] { ${množenec}$ };%
                    
                    % Marking point {faktor}
                    \draw (3.750000,2.000000) node [anchor=north] { ${faktor}$ };%
                    
                    % Drawing circle k
                    \draw [line width=0.016cm] (6.000000,1.500000) circle (0.100000);%
                    
                    % Filling circle k
                    \fill (6.000000,1.500000) circle (0.100000);%
                    }
                    \end{tikzpicture}
                    
            \end{figure}}

            \end{block}}


            \only<7->{\begin{block}{}
                Produkt naravnih števil je naravno število: $x, y \in \mathbb{N} \Rightarrow x\cdot y \in \mathbb{N}$.
            \end{block}}

            \only<8->{\begin{block}{}
                Število $\mathbf{1}$ je \textbf{nevtralni element} za mmnoženje: $1\cdot x = x$.
            \end{block}}

        \end{frame}

        \begin{frame}
            

            \only<2->{\begin{alertblock}{Odštevanje}
                \only<3->{Številoma $x$ in $y$, pri čemer je $x$ večje od $y$ ($x>y$), priredimo \textbf{razliko} $\mathbf{x-y}$.}
            \end{alertblock}}

            \only<4->{\begin{block}{}
                Število $x$ imenujemo \textbf{zmanjševanec} ali \textbf{minuend}, število $y$  pa imenujemo \textbf{odštevanec} ali \textbf{subtrahend}. 

                \only<5->{Številu $x-y$ rečemo \textbf{razlika} ali \textbf{diferenca}. }

                \only<6->{\begin{figure}
                    \begin{tikzpicture}
                        % \clip (0,0) rectangle (14.000000,10.000000);
                        {\footnotesize
                        
                        % Drawing segment a b
                        \draw [line width=0.016cm] (5.000000,0.500000) -- (5.000000,2.500000);%
                        
                        % Drawing segment b d
                        \draw [line width=0.016cm] (5.000000,2.500000) -- (7.000000,2.500000);%
                        
                        % Drawing segment c d
                        \draw [line width=0.016cm] (7.000000,0.500000) -- (7.000000,2.500000);%
                        
                        % Drawing segment c a
                        \draw [line width=0.016cm] (7.000000,0.500000) -- (5.000000,0.500000);%
                        
                        % Drawing segment e f
                        \draw [line width=0.016cm] (7.000000,1.500000) -- (9.000000,1.500000);%
                        
                        % Drawing arrow e f 1.00
                        \draw [line width=0.016cm] (8.702567,1.539158) -- (9.000000,1.500000);%
                        \draw [line width=0.016cm] (8.702567,1.539158) -- (8.900856,1.500000);%
                        \draw [line width=0.016cm] (8.702567,1.460842) -- (9.000000,1.500000);%
                        \draw [line width=0.016cm] (8.702567,1.460842) -- (8.900856,1.500000);%
                        
                        % Drawing segment g h
                        \draw [line width=0.016cm] (2.500000,1.000000) -- (5.000000,1.000000);%
                        
                        % Drawing segment i j
                        \draw [line width=0.016cm] (2.500000,2.000000) -- (5.000000,2.000000);%
                        
                        % Drawing arrow g h 1.00
                        \draw [line width=0.016cm] (4.702567,1.039158) -- (5.000000,1.000000);%
                        \draw [line width=0.016cm] (4.702567,1.039158) -- (4.900856,1.000000);%
                        \draw [line width=0.016cm] (4.702567,0.960842) -- (5.000000,1.000000);%
                        \draw [line width=0.016cm] (4.702567,0.960842) -- (4.900856,1.000000);%
                        
                        % Drawing arrow i j 1.00
                        \draw [line width=0.016cm] (4.702567,2.039158) -- (5.000000,2.000000);%
                        \draw [line width=0.016cm] (4.702567,2.039158) -- (4.900856,2.000000);%
                        \draw [line width=0.016cm] (4.702567,1.960842) -- (5.000000,2.000000);%
                        \draw [line width=0.016cm] (4.702567,1.960842) -- (4.900856,2.000000);%
                        
                        % Marking point {razlika}
                        \draw (8.000000,1.500000) node [anchor=south] { ${razlika}$ };%
                        
                        % Marking point {diferenca}
                        \draw (8.000000,1.500000) node [anchor=north] { ${diferenca}$ };%
                        
                        % Marking point {od�tevanec}
                        \draw (3.750000,1.000000) node [anchor=south] { ${odštevanec}$ };%
                        
                        % Marking point {subtrahend}
                        \draw (3.750000,1.000000) node [anchor=north] { ${subtrahend}$ };%
                        
                        % Marking point {zmanj�evanec}
                        \draw (3.750000,2.000000) node [anchor=south] { ${zmanjševanec}$ };%
                        
                        % Marking point {minuend}
                        \draw (3.750000,2.000000) node [anchor=north] { ${minuend}$ };%
                        
                        % Drawing segment z w
                        \draw [line width=0.032cm] (5.500000,1.500000) -- (6.500000,1.500000);%
                        }
                        \end{tikzpicture}
                        
            \end{figure}}
        \end{block}}

            \only<7->{\begin{block}{}
                Razlika je število, ki ga moramo prišteti številu $y$, da dobimo število $x$.
                \only<8->{$$ (x-y)+y=x $$}
            \end{block}}

        \end{frame}

        \begin{frame}
            \only<2->{\begin{block}{}
                Seštevanje in množenje sta \textit{dvočleni notranji operaciji} v množici naravnih števil $\mathbb{N}$.

                \only<3->{Odštevanje pa ni notranja operacija v množici naravnih števil $\mathbb{N}$.}
            \end{block}}

            \only<4->{\begin{block}{Vrstni red operacij}
                \only<5->{Prednost pri računanju imajo \textbf{oklepaji} (najprej najbolj notranji),} 
                \only<6->{nato sledi \textbf{množenje},}
                \only<7->{na koncu pa imamo še \textbf{seštevanje} in \textbf{odštevanje}.}
            \end{block}}

            \only<8->{\begin{block}{}
                Kadar v izrazu nastopajo enakovredne računske operacije, računamo od leve proti desni.
            \end{block}}

            \only<9->{\begin{block}{}
                Pri množenju količin, ki so označene s črkovnimi oznakami, piko, ki označuje operacijo množenja ponavadi opustimo.
                \only<10->{$$ x\cdot y = xy$$}
            \end{block}}


        \end{frame}

        \begin{frame}
            \frametitle{Osnovni računski zakoni v $\mathbb{N}$}

            \only<2->{\begin{block}{Komutativnost seštevanja -- zakon o zamenjavi členov}
                \only<3->{$$ \mathbf{x+y=y+x}$$}
                \only<4->{Vsota ni odvisna od vrstnega reda seštevanja.}
            \end{block}}

            \only<5->{\begin{block}{Asociativnost seštevanja -- zakon o poljubnem združevanju členov}
                \only<6->{$$ \mathbf{(x+y)+z=x+(y+z)}$$}
                \only<7->{Vsota več kot dveh sumandov ni odvisna od združevanja po dveh sumandov.}
            \end{block}}

        \end{frame}

        \begin{frame}


            \only<2->{\begin{block}{Komutativnost množenja -- zakon o zamenjavi faktorjev}
                \only<3->{$$ \mathbf{x\cdot y=y\cdot x}$$}
                \only<4->{Produkt ni odvisen od vrstnega reda faktorjev.}
            \end{block}}

            \only<5->{\begin{block}{Asociativnost množenja -- zakon o poljubnem združevanju faktorjev}
                \only<6->{$$ \mathbf{(x\cdot y)\cdot z=x\cdot (y\cdot z)}$$}
                \only<7->{Produkt več kot dveh sumandov ni odvisen od združevanja faktorjev.}
            \end{block}}

            \only<8->{\begin{block}{Distributivnost -- zakon o razčlenjevanju}
                \only<9->{$$ \mathbf{x\cdot z+y\cdot z = (x+y)\cdot z} $$}
                \only<10->{Če to beremo iz desne proti levi, rečemu tudi \textit{pravilo izpostavljanja skupnega faktorja}.}
            \end{block}}

        \end{frame}

        \begin{frame}
            \only<2->{\begin{exampleblock}{Naloga}
                Izračunajte.
                \only<3->{\begin{itemize}
                    \item $(1+2\cdot 7)+3\cdot(2\cdot 2+7)$ \\ ~
                    \item $3\cdot(2+3\cdot 5)\cdot(2+1)$ \\ ~
                    \item $7+(2+6\cdot 3)+(8+4\cdot 5)$ \\ ~
                    \item $11\cdot 4+(12-6)\cdot 5$ \\ ~
                    \item $8+2\cdot(3+7)-15$ \\ ~
                    \item $37-5\cdot(10-3)$ \\ ~
                \end{itemize}}
            \end{exampleblock}}
        \end{frame}

        \begin{frame}
            \only<2->{\begin{exampleblock}{Naloga}
                Hitro izračunajte.
                \only<3->{\begin{itemize}
                    \item $45+37+15$ 
                    \item $108+46-28$
                    \item $5\cdot 13\cdot 8$
                    \item $4\cdot 7\cdot 25$
                    \item $(7+3)\cdot 2\cdot 5$
                    \item $15\cdot(4+6)\cdot 2$
                    \item $3\cdot 5+7\cdot 5$
                    \item $8\cdot 12+6\cdot 8$
                \end{itemize}}
            \end{exampleblock}}
        \end{frame}

        \begin{frame}
            \only<2->{\begin{exampleblock}{Naloga}
                Zapišite račun glede na besedilo in izračunajte.
                \only<3->{\begin{itemize}
                    \item Produktu števil $12$ in $27$ odštejte razliko števil $19$ in $11$. \\ ~
                    \item Vsoti produkta $4$ in $12$ ter produkta $5$ in $16$ odštejte $8$. \\ ~
                    \item Vsoto števil $42$ in $23$ pomnožite z razliko števil $58$ in $29$. \\ ~
                    \item Produkt števil $14$ in $17$ pomnožite z vsoto števil $5$ in $16$. \\ ~ \\ ~
                \end{itemize}}
            \end{exampleblock}}
        \end{frame}

        \begin{frame}
            \only<2->{\begin{exampleblock}{Naloga}
                Rešite besedilno nalogo.
                \only<3->{\begin{itemize}
                    \item V trgovini kupimo tri litre mleka in štiri čokoladne pudinge v prahu. Če stane liter mleka $95$ centov,
                        čokoladni puding v prahu pa $24$ centov, koliko moramo plačati? \\ ~ \\ ~ \\ ~ \\ ~
                    \item Manca bo kuhala rižoto za štiri otroke in šest odraslih. Za otroško porcijo rižote zadošča $45~g$ riža,
                        za odraslo pa $75~g$. Koliko riža mora dati kuhati za rižoto? \\ ~ \\ ~ \\ ~ \\ ~
                \end{itemize}}
            \end{exampleblock}}
        \end{frame}

\subsection{Cela števila}
        \begin{frame}
            \frametitle{Cela števila}

                \only<2->{\begin{alertblock}{Množica celih števil}
                    \only<3->{$$\mathbf{\mathbb{Z} = \{\ldots, -2, -1, 0, 1, 2, 3, \ldots\}}$$}
                \end{alertblock}}

                \only<4->{\begin{block}{}
                    Množica celih števil $\mathbb{Z}$ je definirana kot unija treh množic:
                        \begin{itemize}
                            \item<5-> množica \textbf{pozitivnih celih števil} ($\mathbb{Z}^+$) -- naravna števila $\mathbb{N}$;
                            \item<6-> \textbf{število 0};
                            \item<7-> množica \textbf{negativnih celih števil} ($\mathbb{Z}^-$) -- nasprotna števila vseh naravnih števil.
                        \end{itemize}
                      \only<8->{$$\mathbb{Z} = \mathbb{Z}^- \cup \{0\} \cup \mathbb{Z}^+$$}

                \end{block}}

                \only<9->{\begin{block}{}
                    \textbf{Nasprotna vrednost} števila $n$ je število $-n$.
                \end{block}}
        \end{frame}

        \begin{frame}
            \frametitle{Operacije v množici $\mathbb{Z}$}

            \only<2->{\textbf{\large{Seštevanje}}}

            \only<3->{\begin{block}{}
                $$\mathbf{x+0=x}; ~\forall x\in\mathbb{Z}$$
                \only<4->{Število $0$ je \textbf{nevtralni element} pri seštevanju.}
            \end{block}}

            \only<5->{\begin{block}{}
                $$\mathbf{x+(-x)=0}; ~\forall x\in\mathbb{Z} $$
                \only<6->{Vsota celega števila in njemu nasprotnega števila je enaka $0$.}
            \end{block}}

            \only<7->{\begin{block}{}
                $$\mathbf{-(-x)=x}; ~\forall x\in\mathbb{Z}$$
                \only<8->{Nasprotna vrednost nasprotne vrednosti je enaka prvotni vrednosti.}
            \end{block}}
        \end{frame}


        \begin{frame}

            \only<2->{\begin{block}{}
                Vsota dveh pozitivnih števil je pozitivno število, vsota dveh negativnih števil pa je negativno število.
            \end{block}}

            \only<3->{\begin{block}{}
                $$\mathbf{-x+(-y)=-(x+y)}$$
                \only<4->{Vsota nasprotnih vrednosti je enaka nasprotni vrednosti vsote.}
            \end{block}}

            \only<5->{\begin{block}{}
                Naj bosta $x$ in $y$ naravni števili. Vsota pozitivnega števila $x$ in negativnega števila $-y$ je:
                \begin{itemize}
                    \item<6-> pozitivno število, če je $x>y$ in
                    \item<7-> negativno število, če je $x<y$.
                \end{itemize}
            \end{block}}
        \end{frame}


        \begin{frame}
            \only<2->{\textbf{\large{Odštevanje}}}

            \only<3->{\begin{block}{}
                Razlika $x-y$ dveh pozitivnih števil $x$ in $y$ je:
                \begin{itemize}
                    \item<4-> pozitivno število, če je $x>y$ in 
                    \item<5-> negativno število, če je $x<y$.
                \end{itemize}
            \end{block}}

            \only<6->{\begin{block}{}
                Razlika dveh negativnih števil $(-x)-(-y)$ je:
                \begin{itemize}
                    \item<7-> pozitvno število, če je $x<y$ in 
                    \item<8-> negativno število, če je $x>y$.
                \end{itemize}
            \end{block}}

            \only<9->{\begin{block}{}
                Razlika pozitivnega števila $x$ in negativnega števila $-y$ je pozitvno število.
            \end{block}}


            \only<10->{\begin{alertblock}{}
                \textit{Odštevanje v množici $\mathbb{Z}$ je prištevanje nasprotne vrednosti.}
                \only<11->{$$\mathbf{x-y=x+(-y)} $$}
            \end{alertblock}}
        \end{frame}

        \begin{frame}
            \only<2->{\textbf{\large{Množenje}}}

            \only<3->{\begin{block}{}
                $$\mathbf{1\cdot x=x}; ~\forall x\in\mathbb{Z}$$
                \only<4->{Število $1$ je \textbf{nevtralni element} za množenje.}
            \end{block}}

            \only<5->{\begin{block}{}
                $$\mathbf{(-1)\cdot x=-x}; ~\forall x\in\mathbb{Z}$$
                \only<6->{Pri množenju celega števila $x$ z $-1$ dobimo nasprotno število $-x$.}
            \end{block}}

            \only<7->{\begin{block}{}
                $$\mathbf{0\cdot x=0}; ~\forall x\in\mathbb{Z}$$
                \only<8->{Rezultat množenja števila s številom $0$ je enak $0$.}
            \end{block}}

        \end{frame}

        \begin{frame}


            \only<2->{\begin{block}{}
                $$\mathbf{(-x)(-y)=xy}$$
                \only<3->{Produkt sodo mnogo negativnih števil je pozitivno število.}
            \end{block}}

            \only<4->{\begin{block}{}
                $$\mathbf{-x\cdot y=-(xy)}$$
                \only<5->{$$\mathbf{x(-y)=-(xy)}$$}
                \only<6->{Produkt pozitivnega in negativnega števila je negativno število.}
            \end{block}}

            \only<7->{\begin{block}{}
                Produkt liho mnogo negativnih faktorjev je negativno število.
            \end{block}}

            \only<8->{\begin{block}{}
                Seštevanje, odštevanje in množenje so v množici $\mathbb{Z}$ dvočlene notranje operacije.
            \end{block}}
        \end{frame}


        \begin{frame}
            \frametitle{Osnovni računski zakoni v $\mathbb{Z}$}

            \begin{columns}[T]
                \column{0.48\textwidth}
                \only<2->{\begin{block}{Komutativnost seštevanja}
                    \only<3->{$$ \mathbf{x+ y=y+ x}$$}
                \end{block}}
    
                \only<4->{\begin{block}{Asociativnost seštevanja}
                    \only<5->{$$ \mathbf{(x+ y)+ z=x+ (y+ z)}$$}
                \end{block}}

                \column{0.48\textwidth}

                \only<6->{\begin{block}{Komutativnost množenja}
                    \only<7->{$$ \mathbf{x\cdot y=y\cdot x}$$}
                \end{block}}
    
                \only<8->{\begin{block}{Asociativnost množenja}
                    \only<9->{$$ \mathbf{(x\cdot y)\cdot z=x\cdot (y\cdot z)}$$}
                \end{block}}
            \end{columns}

    
                \only<10->{\begin{block}{Distributivnost seštevanja in množenja ter odštevanja in množenja}
                    \only<11->{$$ \mathbf{x\cdot z+y\cdot z = (x+y)\cdot z} $$}
                    \only<12->{$$ \mathbf{x\cdot z-y\cdot z = (x-y)\cdot z} $$}

                \end{block}}
    
        \end{frame}


        \begin{frame}
            \only<2->{\begin{exampleblock}{Naloga}
                Izračunajte.
                \only<3->{\begin{itemize}
                    \item $17-13-2+10$ \\ ~
                    \item $50+11-32-14$ \\ ~
                    \item $3+((5+2(7-9))\cdot 2-1)$ \\ ~
                    \item $(2-5(6-10))\cdot(5-2(7-5))$ \\ ~
                    \item $9(11-3)+7(10-15)$ \\ ~
                    \item $8+9(11-18)-2\cdot 5$ \\ ~
                \end{itemize}}
            \end{exampleblock}}
        \end{frame}

        \begin{frame}
            \only<2->{\begin{exampleblock}{Naloga}
                Spretno izračunajte.
                \only<3->{\begin{itemize}
                    \item $7\cdot 8-12\cdot 8$ \\ ~
                    \item $5\cdot 18+9\cdot 5-5\cdot 2$ \\ ~
                    \item $8\cdot(4-9)\cdot 2$ \\ ~
                    \item $5\cdot 3\cdot (12-8)$ \\ ~
                    \item $(15-6)(12-3\cdot 4)$ \\ ~
                \end{itemize}}
            \end{exampleblock}}
        \end{frame}

        \begin{frame}
            \only<2->{\begin{exampleblock}{Naloga}
                Rešite besedilne naloge.
                \only<3->{\begin{itemize}
                    \item V hotelu imajo na voljo osemnajst enoposteljnih, štiriintrideset dvoposteljnih in petindevetdeset triposteljnih sob.
                        Koliko ljudi lahko še prespi v hotelu, če je v njem že sto triinštirideset gostov? \\ ~ \\ ~ \\ ~
                    \item Pohod na bližnji hrib traja tri ure. Koliko minut moramo še hoditi, če smo na poti že $145$ minut? \\ ~ \\ ~ \\ ~ \\
                \end{itemize}}
            \end{exampleblock}}
        \end{frame}

        \begin{frame}

            \only<2->{\begin{exampleblock}{Naloga}
                \only\begin{itemize}
                    \item S Ptuja in iz Postojne (razdalja med njima je približno $190~km$) sočasno odpeljeta dva motorista drug proti drugemu.
                        En vozi povprečno $40~km/h$, drugi pa $5~km/h$ manj. Kolikšna bo razdalja med njima po dveh urah vožnje? \\ ~ \\ ~ \\ ~
                \end{itemize}
            \end{exampleblock}}

            \only<3->{\begin{exampleblock}{Naloga}
                Zapišite enačbe in jih poenostavite.
                \only<4->{\begin{itemize}
                    \item Razlika petkratnika $a$ in $b$ je enaka trikratniku vsote štirikratnika $a$ in petkratnika $b$. \\ ~ \\
                    \item Vsota $x$ in dvakratnika $y$ je enaka razliki petkratnika $x$ in dvanajstkratnika $y$. \\ ~ \\ ~
                \end{itemize}}
            \end{exampleblock}}
        \end{frame}






\subsection{Urejenost naravnih in celih števil}

        \begin{frame}
            \frametitle{Urejenost naravnih in celih števil}

            \only<2->{\begin{alertblock}{}
                Številska množica je \textbf{urejena}, kadar lahko po velikosti primerjamo njena poljubna elementa.
            \end{alertblock}}

            \only<3->{\begin{block}{}
                Pri urejanju števil uporabljamo naslednje znake:
                \only<4->{\begin{table}
                    \centering
                    \addtolength{\tabcolsep}{6pt}
                    \renewcommand{\arraystretch}{1.4}                
                    \begin{tabular}{||c|c||} 
                        \hhline{|t:==:t|}
                                $\mathbf{<}$ & manjše / manj  \\ 
                        \hline
                                $\mathbf{>}$ & večje / več   \\ 
                        \hline
                                $\mathbf{\leq}$ & manjše ali enako / največ   \\ 
                        \hline
                                $\mathbf{\geq}$ & večje ali enako / vsaj, najmanj \\  
                        \hline
                                $\mathbf{=}$ & enako \\
                        \hhline{|b:==:b|}
                    \end{tabular}
                \end{table}}
            \end{block}}
        \end{frame}

        \begin{frame}
            \only<2->{\begin{block}{}
                Za poljubni števili $x,y\in\mathbb{Z}$ velja natanko ena izmed naslednjih možnosti: $x>y$, $x<y$ ali $x=y$.
            \end{block}}

            \only<3->{\begin{block}{}
                $$\mathbf{x>y \Leftrightarrow x-y>0}$$
                \only<4->{Slika števila $x$ leži na številski premici desno od slike števila $y$.}
            \end{block}}

            \only<5->{\begin{block}{}
                $$\mathbf{x<y \Leftrightarrow x-y<0}$$
                \only<6->{Slika števila $x$ leži na številski premici levo od slike števila $y$.}
            \end{block}}

            \only<7->{\begin{block}{}
                $$\mathbf{x=y \Leftrightarrow x-y=0}$$
                \only<8->{Slika števila $x$ sovpada s sliko števila $y$.}
            \end{block}}

        \end{frame}

        \begin{frame}
            \only<2->{\begin{block}{Pozitivna števila}
                \only<3->{V množici $\mathbb{Z}$ so pozitivna tista števila, ki so večja od števila $0$ 
                in njihove slike ležijo desno od izhodišča.}
            \end{block}}
            
            \only<4->{\begin{block}{Negativna števila}
                \only<5->{V množici $\mathbb{Z}$ so negativna tista števila, ki so manjša od števila $0$ 
                in njihove slike ležijo levo od izhodišča.}
            \end{block}}

            \only<6->{\begin{block}{}
                Vsako pozitivno celo število (vsako naravno število) je večje od katerega koli negativnega celega števila.
            \end{block}}

            \only<7->{\begin{block}{}
                Velja pa tudi:
                $$x\leq y \Leftrightarrow x-y\leq 0 $$
                \only<8->{$$x\geq y \Leftrightarrow x-y\geq 0 $$}
            \end{block}}
        \end{frame}


        \begin{frame}
            \only<2->{\begin{alertblock}{}
                Z relacijo \textit{biti manjši ali enak} je množica $\mathbb{Z}$ \textbf{linearno urejena}, 
                to pomeni, da veljajo:
            \end{alertblock}}
            
            \only<3->{\begin{block}{Refleksivnost}
                \only<4->{$$\forall x\in\mathbb{Z}: x\leq x$$}
            \end{block}}

            \only<5->{\begin{block}{Antisimetričnost}
                \only<6->{$$\forall x,y\in\mathbb{Z}: x\leq y \land y\leq x \Rightarrow x=y$$}
            \end{block}}

            \only<7->{\begin{block}{Tranzitivnost}
                \only<8->{$$\forall x,y,z\in\mathbb{Z}: x\leq y \land y\leq z \Rightarrow x\leq z$$}
            \end{block}}

            \only<9->{\begin{block}{Stroga sovisnost}
                \only<10->{$$\forall x,y\in\mathbb{Z}: x\leq y \lor y\leq x$$}
            \end{block}}

        \end{frame}

        \begin{frame}
            \only<2->{\begin{block}{Monotonost vsote}
                \only<3->{$$x<y \Rightarrow x+z<y+z \quad \quad x\leq y \Rightarrow x+z\leq y+z$$}
                \only<4->{Če na obeh straneh neenakosti prištejemo isto število, se neenakost ohrani.}
            \end{block}}

            \only<5->{\begin{block}{}
                $$x<y \land z>0 \Rightarrow x\cdot z<y\cdot z \quad \quad x\leq y \land z>0 \Rightarrow x\cdot z\leq y\cdot z$$
                \only<6->{Pri množenju neenakosti z negativnim številom se znak neenakosti ohrani.}
            \end{block}}

            \only<7->{\begin{block}{}
                $$x<y \land z<0 \Rightarrow x\cdot z>y\cdot z \quad \quad x\leq y \land z<0 \Rightarrow x\cdot z\geq y\cdot z$$
                \only<8->{Pri množenju neenakosti z negativnim številom se znak neenakosti obrne.}
            \end{block}}

            \only<9->{\begin{block}{}
                Obravnavane lastnosti veljajo tudi za relaciji $\geq$ in $>$.
            \end{block}}

        \end{frame}

        \begin{frame}

                \only<2->{\begin{exampleblock}{Naloga}
                    Uredite števila $3, -2, 5, -1, 0, -7, 6, -6$ po velikosti in jih predstavite na številski premici.
                \end{exampleblock}}

                \only<3->{\begin{exampleblock}{Naloga}
                    Uredite števila $104, -27, 35, -107, 36, -26, 25, -28, 81$ po velikosti.
                \end{exampleblock}}

                \only<4->{\begin{exampleblock}{Naloga}
                    Gladina Mrtvega morja leži v depresiji na $-423~m$ nadmorske višine, njegova največja globina pa je $378~m$.
                    Kolikšna je najmanjša nadmorska višina dna Mrtvega morja?
                \end{exampleblock}}

                \only<5->{\begin{exampleblock}{Naloga}
                    Za katera cela števila $x$ ima izraz $3x-5(x+2)$ večjo ali enako vrednost od izraza $4-(12+x)$?
                \end{exampleblock}}
    
        \end{frame}




%--------------------------------------------------------------------


\end{document}