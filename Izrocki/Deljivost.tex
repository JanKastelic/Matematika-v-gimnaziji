\chapter{Deljivost}


    \section{Relacija deljivosti}

        
            
                Naravno število $n$ je \textbf{delitelj} naravnega števila $n$ (\textbf{deljenec}), če obstaja naravno število $k$ (\textbf{kvocient}), da velja: $$\mathbf{n=k\cdot m}.$$
            
                ~\newline
                Naravno število $m$ deli naravno število $n$, ko je število $n$ večkratnik števila $m$. $$m\mid n \Leftrightarrow n=k\cdot m;\quad m,n,k\in\mathbb{N}$$
            
                ~\newline
                Število $m$ je delitelj samega sebe in vseh svojih večkratnikov.
            
                $1$ je delitelj vsakega naravnega števila.
                ~\newline

                Če $d$ deli naravni števili $m$ in $n$, $n>m$, potem $d$ deli tudi vsoto in razliko števil $m$ in $n$.
            
                ~\newline
                Pri deljenju poljubnega naravnega števila $n$ z naravnim številom $m$ imamo dve možnosti: $n$ je deljivo z $m$ ali $n$ ni deljivo z $m$.
        
                ~\newline
                Relacija deljivosti je:
                \begin{enumerate}
                    \item \textbf{refleksivna}: $$a\mid a;$$
                    \item \textbf{antisimetrična}: $$a\mid b \wedge b\mid a \Rightarrow a=b;$$
                    \item \textbf{tranzitivna}:  $$a\mid b \wedge b\mid c \Rightarrow a\mid c.$$
                \end{enumerate}
            
                Relacija s temi lastnostmi je relacija \textbf{delne urejenosti}, zato relacija deljivosti delno ureja množico $\mathbb{N}$.
            
        
                ~\newline~
        
            \begin{naloga}
                Zapišite vse delitelje števil.
                \begin{itemize}
                    \item $6$ 
                    \item $16$ 
                    \item $37$ 
                    \item $48$ 
                    \item $120$ 
                \end{itemize}
            \end{naloga}        

        
            \begin{naloga}
                Pokažite, da trditev velja.
                \begin{itemize}
                    \item Izraz $x-3$ deli izraz $x^2-2x-3$. 
                    \item Izraz $x+2$ deli izraz $x^3+x^2-4x-4$. 
                    \item Izraz $x-2$ deli izraz $x^3-8$. 
                \end{itemize}
            \end{naloga}        

        
            \begin{naloga}
                Pokažite, da trditev velja.
                \begin{itemize}
                    \item $19\mid \left(3^{21}-3^{20}+3^{18}\right)$ 
                    \item $7\mid \left(3\cdot 4^{11}+4^{12}+7\cdot 4^{10}\right)$ 
                    \item $14\mid \left(5\cdot 3^6+2\cdot 3^8-3\cdot 3^7\right)$ 
                    \item $25\mid \left(7\cdot 2^{23}-3\cdot 2^{24}+3\cdot 2^{25}-2^{22}\right)$ 
                    \item $11\mid \left(2\cdot 10^6+3\cdot 10^7+10^8\right)$ 
                    \item $35\mid \left(6^{32}-36^{15}\right)$ 
                \end{itemize}
            \end{naloga}        

        
            \begin{naloga}
                Pokažite, da trditev velja.
                \begin{itemize}
                    \item $3\mid \left(2^{2n+1}-5\cdot 2^{2n}+9\cdot 2^{2n-1}\right)$ 
                    \item $29\mid \left(5^{n+3}-2\cdot 5^{n+1}+7\cdot 5^{n+2}\right)$ 
                    \item $10\mid \left(3\cdot 7^{4n-1}-4\cdot 7^{4n-2}+7^{4n+1}\right)$ 
                    \item $10\mid \left(9^{3n-1}+9\cdot 9^{3n+1}+9^{3n}-9^{3n+2}\right)$ 
                    \item $5\mid \left(7\cdot 2^{4n-2}+3\cdot 4^{2n}-16^n\right)$ 
                \end{itemize}
            \end{naloga}        


        
            \begin{naloga}
                Pokažite, da je za poljubno naravno število $u$ vrednost izraza $$(u+7)(7-u)-3(3-u)(u+5)$$ večkratnik števila $4$.
            \end{naloga}        
