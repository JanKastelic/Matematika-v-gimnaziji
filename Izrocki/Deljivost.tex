\chapter{Deljivost}


    \section{Relacija deljivosti}
            
        Naravno število $n$ je \textbf{delitelj} naravnega števila $n$ (\textbf{deljenec}), če obstaja naravno število $k$ (\textbf{kvocient}), da velja: $$\mathbf{n=k\cdot m}.$$
    
        ~\newline
        Naravno število $m$ deli naravno število $n$, ko je število $n$ večkratnik števila $m$. $$m\mid n \Leftrightarrow n=k\cdot m;\quad m,n,k\in\mathbb{N}$$
    
        ~\newline
        Število $m$ je delitelj samega sebe in vseh svojih večkratnikov.
    
        $1$ je delitelj vsakega naravnega števila.
        ~\newline

        Če $d$ deli naravni števili $m$ in $n$, $n>m$, potem $d$ deli tudi vsoto in razliko števil $m$ in $n$.
    
        ~\newline
        Pri deljenju poljubnega naravnega števila $n$ z naravnim številom $m$ imamo dve možnosti: $n$ je deljivo z $m$ ali $n$ ni deljivo z $m$.

        ~\newline
        Relacija deljivosti je:
        \begin{enumerate}
            \item \textbf{refleksivna}: $$a\mid a;$$
            \item \textbf{antisimetrična}: $$a\mid b \wedge b\mid a \Rightarrow a=b;$$
            \item \textbf{tranzitivna}:  $$a\mid b \wedge b\mid c \Rightarrow a\mid c.$$
        \end{enumerate}
    
        Relacija s temi lastnostmi je relacija \textbf{delne urejenosti}, zato relacija deljivosti delno ureja množico $\mathbb{N}$.
    
        ~\newline~
    
        \begin{naloga}
            Zapišite vse delitelje števil.
            \begin{itemize}
                \item $6$ 
                \item $16$ 
                \item $37$ 
                \item $48$ 
                \item $120$ 
            \end{itemize}
        \end{naloga}        

    
        \begin{naloga}
            Pokažite, da trditev velja.
            \begin{itemize}
                \item Izraz $x-3$ deli izraz $x^2-2x-3$. 
                \item Izraz $x+2$ deli izraz $x^3+x^2-4x-4$. 
                \item Izraz $x-2$ deli izraz $x^3-8$. 
            \end{itemize}
        \end{naloga}        

    
        \begin{naloga}
            Pokažite, da trditev velja.
            \begin{itemize}
                \item $19\mid \left(3^{21}-3^{20}+3^{18}\right)$ 
                \item $7\mid \left(3\cdot 4^{11}+4^{12}+7\cdot 4^{10}\right)$ 
                \item $14\mid \left(5\cdot 3^6+2\cdot 3^8-3\cdot 3^7\right)$ 
                \item $25\mid \left(7\cdot 2^{23}-3\cdot 2^{24}+3\cdot 2^{25}-2^{22}\right)$ 
                \item $11\mid \left(2\cdot 10^6+3\cdot 10^7+10^8\right)$ 
                \item $35\mid \left(6^{32}-36^{15}\right)$ 
            \end{itemize}
        \end{naloga}        

    
        \begin{naloga}
            Pokažite, da trditev velja.
            \begin{itemize}
                \item $3\mid \left(2^{2n+1}-5\cdot 2^{2n}+9\cdot 2^{2n-1}\right)$ 
                \item $29\mid \left(5^{n+3}-2\cdot 5^{n+1}+7\cdot 5^{n+2}\right)$ 
                \item $10\mid \left(3\cdot 7^{4n-1}-4\cdot 7^{4n-2}+7^{4n+1}\right)$ 
                \item $10\mid \left(9^{3n-1}+9\cdot 9^{3n+1}+9^{3n}-9^{3n+2}\right)$ 
                \item $5\mid \left(7\cdot 2^{4n-2}+3\cdot 4^{2n}-16^n\right)$ 
            \end{itemize}
        \end{naloga}        


    
        \begin{naloga}
            Pokažite, da je za poljubno naravno število $u$ vrednost izraza $$(u+7)(7-u)-3(3-u)(u+5)$$ večkratnik števila $4$.
        \end{naloga}        

\newpage
    \section{Kriteriji deljivost}
    
        \subsection*{Deljivost z $2$}
            Število je deljivo z $2$ natanko takrat, ko so enice števila deljive z $2$.

        \subsection*{Deljivost s $3$}
            Število je deljivo s $3$ natanko takrat, ko je vsota števk števila deljiva s $3$.

        \subsection*{Deljivost s $4$ oziroma $25$}
            Število je deljivo s $4$ oziroma $25$ natanko takrat, ko je dvomestni konec števila deljivo s $4$ oziroma~$25$.

        \subsection*{Deljivost s $5$}
            Število je deljivo s $5$ natanko takrat, ko so enice števila enake $0$ ali $5$.
    
        \subsection*{Deljivost s $6$}
            Število je deljivo s $6$ natanko takrat, ko je deljivo z $2$ in s $3$ hkrati.

        \subsection*{Deljivost z $8$ oziroma s $125$}
            Število je deljivo z $8$ oziroma s $125$ natanko takrat, ko je trimestni konec števila deljivo z~$8$ oziroma s $125$.

        \subsection*{Deljivost z $9$}
            Število je deljivo z $9$ natanko takrat, ko je vsota števk števila deljiva z $9$.

        \subsection*{Deljivost z $10$ oziroma $10^n$}
            Število je deljivo z $10$ natanko takrat, ko so enice števila enake $0$.
            \\Število je deljivo z $10^n$ natanko takrat, ko ima število na zadnjih $n$ mestih števko $0$.
    
        \subsection*{Deljivost z $11$}
            Število je deljivo z $11$ natanko takrat, ko je alternirajoča vsota števk tega števila deljiva z $11$.

        \subsection*{Deljivost s $7$}
            Algoritem za preverjanje deljivosti s $7$:
            \begin{enumerate}
                \item vzamemo enice danega števila in jih pomnožimo s $5$,
                \item prvotnemu številu brez enic prištejemo dobljeni produkt,
                \item vzamemo enice dobljene vsote in jih pomnožimo s $5$,
                \item produkt prištejemo prej novo dobljenemu številu ...     
            \end{enumerate}
            Postopek ponavljamo, dokler ne dobimo dvomestnega števila -- 
            če je to deljivo s $7$, je prvotno število deljivo s $7$. 
            
    

    
        \begin{naloga}
            S katerimi od števil $2$, $3$, $4$, $5$, $6$, $7$, $8$, $9$, $10$, $11$ so deljiva naslednja števila?
            \begin{itemize}
                \item $84742$ 
                \item $393948$ 
                \item $12390$ 
                \item $19401$ 
            \end{itemize}
        \end{naloga}
    

    
        \begin{naloga}
            Določite vse možnosti za števko $a$, da je število $\overline{65833a}$:
            \begin{itemize}
                \item deljivo s $3$, 
                \item deljivo s $4$, 
                \item deljivo s $5$, 
                \item deljivo s $6$. 
            \end{itemize}
        \end{naloga}
    

    
        \begin{naloga}
            Določite vse možnosti za števko $b$, da je število $\overline{65b90b}$:
            \begin{itemize}
                \item deljivo z $2$, 
                \item deljivo s $3$, 
                \item deljivo s $6$, 
                \item deljivo z $9$, 
                \item deljivo z $10$. 
            \end{itemize}
        \end{naloga}
    

    
        \begin{naloga}
            Določite vse možnosti za števki $c$ in $d$, da je število $\overline{115c1d}$ deljivo s $6$.
            
        \end{naloga}

        \begin{naloga}
            Določite vse možnosti za števki $e$ in $f$, da je število $\overline{115e1f}$ deljivo z $8$.
            
        \end{naloga}

    


    
        \begin{naloga}
            Pokažite, da za vsako naravno število $n$ $12$ deli $n^4-n^2$.
            
        \end{naloga}

        \begin{naloga}
            Preverite, ali je število $8641 969$ deljivo s $7$.
            
        \end{naloga}
        
            
    