\chapter{Osnove logike}

      
             \section{Izjave}

                \textbf{Matematična izjava} je vsaka smiselna poved, za katero 
                lahko določimo resničnost oziroma pravilnost.

                 
                Matematična izjava lahko zavzame dve logični vrednosti:
                \begin{itemize}
                    \item izjava je \textbf{resnična}/\textbf{pravilna}, 
                        oznaka $\mathbf{R}$/$\mathbf{P}$/$\mathbf{1}$/$\mathbf{\top}$;
                    \item izjava je \textbf{neresnična}/\textbf{nepravilna}, 
                        oznaka $\mathbf{N}$/$\mathbf{0}$/$\mathbf{\bot }$.
                \end{itemize}                

                 
                Izjave označujemo z velikimi tiskanimi črkami ($A$, $B$, $C$ ...).
             
         

         
             \begin{naloga}
                Ali so naslednje povedi izjave?
                \begin{itemize}   
                    \item Danes sije sonce.
                    \item Koliko je ura?
                    \item Piramida je geometrijski lik.
                    \item Daj mi jabolko.
                    \item Število $12$ deli število $3$.
                    \item Število $3$ deli število $10$.
                    \item Ali si pisal matematični test odlično?
                    \item Matematični test si pisal odlično.
                    \item Ali je $10~dl$ isto kot $1~l$?
                    \item Število $41$ je praštevilo.
                \end{itemize}
            \end{naloga}
             
         

         
             \begin{naloga}
                Spodnjim izjavam določite logične vrednosti.
                \begin{itemize}   
                    \item $A$: Najvišja gora v Evropi je Mont Blanc.
                    \item $B$: Število je deljivo s $4$ natanko takrat, ko je vsota števk deljiva s $4$.
                    \item $C$: Ostanek pri deljenju s $4$ je lahko $1$, $2$ ali $3$.
                    \item $D$: Mesec februar ima 28 dni.
                    \item $E$: Vsa praštevila so liha števila.
                    \item $F$: Število $1$ je naravno število.
                    \item $G$: Praštevil je neskončno mnogo.
                \end{itemize}
            \end{naloga}
            
         

         
              \subsection{Enostavne in sestavjene izjave}
                
                Izjave delimo med:
                \begin{itemize}
                    \item \textbf{elementarne}/\textbf{enostavne izjave} -- ne moremo 
                        jih razstaviti na bolj enostavne;
                    \item \textbf{sestavljene izjave} -- sestavljene iz elementarnih izjav, 
                        ki jih med seboj povezujejo \textbf{logične operacije} (imenovane 
                        tudi izjavne povezave oziroma~ logična vezja).
                \end{itemize}
             

               
                Vrednost sestavljene izjave izračunamo glede na vrednosti elementarnih 
                izjav in izjavnih povezav med njimi.
             
               
                Pravilnost sestavljenih izjav nazorno prikazujejo 
                \textbf{resničnostne}/\textbf{pravilnostne tabele}.
             

         

         
             \section{Logične operacije}

              \subsection{Negacija}
                \textbf{Negacija} izjave $A$ je izjava, ki \textbf{trdi nasprotno} 
                kot izjava $A$.
                Oznaka: $\mathbf{\lnot A}$.
                $$ \mathbf{\lnot A} \quad \quad \textmd{\textbf{Ni res}, da velja izjava A.}$$
             

            % \begin{columns}
            %     \column{0.65\textwidth} 
                      
                        Če je izjava $A$ pravilna, je $\lnot A$ nepravilna in obratno: 
                        če je $\lnot A$ pravilna, je $A$ nepravilna.
                     
                      
                        Negacija negacije izjave je potrditev izjave. \quad $\lnot(\lnot A)=A$
                     

                % \column{0.3\textwidth} 
                \begin{table}[H]
                    \centering
                    \begin{tabular}{||c|c||} 
                    \hhline{|t:==:t|}
                    \rowcolor[rgb]{0.843,0.718,0.718} $A$ & $\lnot A$  \\ 
                    \hhline{|:==:|}
                    $P$                                   & $N$                       \\ 
                    \hline
                    $N$                                   & $P$                       \\
                    \hhline{|b:==:b|}
                    \end{tabular}                    
                \end{table}                

            % \end{columns}
         

         
             \begin{naloga}
                Izjavam določite logično vrednost, potem jih zanikajte in določite logično vrednost negacij.
                \begin{itemize}
                    \item $A$: $5 \cdot 8 = 30$
                    \item $B$: Število $3$ je praštevilo.
                    \item $C$: Največje dvomestno število je $99$.
                    \item $D$: Število $62$ je večratnik števila $4$.
                    \item $E$: Praštevil je neskončno mnogo.
                    \item $F$: $7 \leq 5$
                    \item $G$: Naša pisava je cirilica.
                \end{itemize}
            \end{naloga}
         

         
             \subsection{Konjunkcija}
                \textbf{Konjunkcija} izjav $A$ in $B$ nastane tako, da povežemo izjavi $A$ in $B$ 
                z \textbf{in hkrati}.
                $$ \mathbf{A\land B} \quad \quad \textmd{Velja izjava A \textbf{in (hkrati)} izjava B.}$$
             
            % \begin{columns}
                % \column{0.6\textwidth} 
                    %   
                        Če sta izjavi $A$ in $B$ pravilni, je pravilna tudi njuna konjunkcija, 
                        če je pa ena od izjav nepravilna, je nepravilna tudi njuna konjunkcija.
                     

                % \column{0.35\textwidth} 
                    \begin{table}[H]
                        \centering
                        \begin{tabular}{||c|c|c||} 
                        \hhline{|t:===:t|}
                        \rowcolor[rgb]{0.843,0.718,0.718} $A$ & $B$ & $A\land B$  \\ 
                        \hhline{|:===:|}
                        $P$ & $P$ & $P$                         \\ 
                        \hline
                        $P$ & $N$ & $N$                         \\ 
                        \hline
                        $N$ & $P$ & $N$                         \\ 
                        \hline
                        $N$ & $N$ & $N$                         \\
                        \hhline{|b:===:b|}
                        \end{tabular}
                    \end{table}

            % \end{columns}


         

         
             \begin{naloga}
                Določite logično vrednost konjunkcijam.
                \begin{itemize}
                    \item Število $28$ je večratnik števila $3$ in večkratnik števila $8$.
                    \item Število $7$ je praštevilo in je deljivo s številom $1$.
                    \item Vsakemu celemu številu lahko pripišemo nasprotno število in obratno celo število.
                    \item Ostanki pri deljenju števila s $3$ so lahko $0$, $1$ ali $2$, 
                        pri deljenju s $5$ pa $0$, $1$, $2$, $3$ ali $4$.
                    \item Število je deljivo s $3$, če je vsota števk deljiva s $3$, in je 
                        deljivo z $9$, če je vsota števk deljiva z $9$.
                \end{itemize}
            \end{naloga}
         

         
             \subsection{Disjunkcija}
                \textbf{Disjunkcija} izjav $A$ in $B$ nastane s povezavo \textbf{ali}.
                $$ \mathbf{A\lor B} \quad \quad \textmd{Velja izjava A \textbf{ali} izjava B 
                (lahko tudi obe hkrati).}$$
             
            % \begin{columns}
                % \column{0.62\textwidth} 
                      
                        Disjunkcija je nepravilna, če sta nepravilni obe izjavi, ki jo sestavljata,
                        v preostalih treh primerih je pravilna.
                     

                % \column{0.35\textwidth} 
                    \begin{table}[H]
                        \centering
                        \begin{tabular}{||c|c|c||} 
                        \hhline{|t:===:t|}
                        \rowcolor[rgb]{0.843,0.718,0.718} $A$ & $B$ & $A\lor B$  \\ 
                        \hhline{|:===:|}
                        $P$ & $P$ & $P$                         \\ 
                        \hline
                        $P$ & $N$ & $P$                         \\ 
                        \hline
                        $N$ & $P$ & $P$                         \\ 
                        \hline
                        $N$ & $N$ & $N$                         \\
                        \hhline{|b:===:b|}
                        \end{tabular}
                    \end{table}

            % \end{columns}


         

         
             \begin{naloga}
                Določite logično vrednost disjunkcijam.
                \begin{itemize}
                    \item Število $24$ je večratnik števila $3$ ali $8$.
                    \item Število $35$ ni večratnik števila $7$ ali $6$.
                    \item Število $5$ deli število $16$ ali $18$.
                    \item Ploščina kvadrata s stranico $a$ je $a^2$ ali obseg kvadrata je $4a$.
                    \item Ni res, da je vsota notranjih kotov trikotnika $160^\circ$, ali ni res, 
                        da Pitagorov izrek velja v poljubnem trikotniku.
                \end{itemize}
            \end{naloga}
         


         
             \subsection{Komutativnost konjunkcije in disjunkcije}
                $$ A\land B = B\land A $$
                $$ A\lor B = B\lor A$$
                   
            
             \subsection{Asociativnost konjunkcije in disjunkcije}
                $$(A\land B)\land C = A\land(B\land C) $$
                $$ (A\lor B)\lor C = A\lor (B\lor C) $$
             

             \subsection{Distributivnost zakona za konjunkcijo in disjunkcijo}
                $$(A\lor B)\land C = (A\land C)\lor(B\land C) $$
                $$ (A\land B)\lor C = (A\lor C)\land(B\lor C) $$
             

             \subsection{De Morganova zakona}
                \begin{itemize}
                    \item negacija konjunkcije je disjunkcija negacij: 
                        $\lnot(A\land B)=\lnot A\lor\lnot B$
                    \item negacija disjunkcije je konjunkcija negacij: 
                        $\lnot(A\lor B)=\lnot A\land\lnot B$
                \end{itemize}                
             
        
         

         
             \begin{naloga}
                Katere od spodnjih izjav so pravilne in katere nepravilne?
                \begin{itemize}
                    \item $(3\cdot 4 = 12)\land(12:4=3)$
                    \item $(a^3\cdot a^5=a^{15})\lor (a^3\cdot a^5=a^8)$
                    \item $(3|30)\land(3|26)$
                    \item $(3|30)\lor(3|26)$
                    \item $(2^3=9)\lor(3^2=9)$
                    \item $((-2)^2=4)\land\lnot(-2^2=4)$
                \end{itemize}
            \end{naloga}
         

         
             \subsection{Implikacija}
                \textbf{Implikacija} izjav $A$ in $B$ je sestavljena izjava, ki jo lahko beremo
                na različne načine.
                $$ \mathbf{A\Rightarrow B} \quad \quad \textmd{\textbf{Če} velja izjava A, 
                \textbf{potem} velja izjava B. / \textbf{Iz} A \textbf{sledi} B.}$$
                Izjava $A$ je \textbf{pogoj} ali \textbf{privzetek}, izjava $B$ pa 
                \textbf{(logična) posledica} izjave $A$.
             
            % \begin{columns}
                % \column{0.62\textwidth} 
                      
                        Implikacija je nepravilna, ko je izjava $A$ pravilna, izjava $B$ pa 
                        nepravilna, v preostalih treh primerih je pravilna.
                     

                % \column{0.35\textwidth} 
                    \begin{table}[H]
                        \centering
                        \begin{tabular}{||c|c|c||} 
                        \hhline{|t:===:t|}
                        \rowcolor[rgb]{0.843,0.718,0.718} $A$ & $B$ & $A\Rightarrow B$  \\ 
                        \hhline{|:===:|}
                        $P$ & $P$ & $P$                         \\ 
                        \hline
                        $P$ & $N$ & $N$                         \\ 
                        \hline
                        $N$ & $P$ & $P$                         \\ 
                        \hline
                        $N$ & $N$ & $P$                         \\
                        \hhline{|b:===:b|}
                        \end{tabular}
                    \end{table}

            % \end{columns}
         

         
             \begin{naloga}
                Določite, ali so izjave pravilne.
                \begin{itemize}
                    \item Če je število deljivo s $100$, je deljivo tudi s $4$.
                    \item Če je štirikotnik pravokotnik, se diagonali razpolavljata.
                    \item Če je štirikotnik kvadrat, se diagonali sekata pod pravim kotom.
                    \item Če sta števili $2$ in $3$ lihi števili, potem je produk teh dveh števil sodo število.
                    \item Če je število $18$ deljivo z $9$, potem je deljivo s $3$.
                    \item Če je $7$ večkratnik števila $7$, potem $7$ deli število $43$.
                    \item Če je število deljivo s $4$, potem je deljivo z $2$.
                \end{itemize}
            \end{naloga}
         
         
             \subsection{Ekvivalenca}
                \textbf{Ekvivalenca} izjavi $A$ in $B$ poveže s \textbf{če in samo če} oziroma
                \textbf{natanko tedaj, ko}.
                \begin{align*} 
                    \mathbf{A\Leftrightarrow B} \quad \quad &\textmd{Izjava A velja, \textbf{če in
                    samo če} velja izjava B.} / \\
                        &\textmd{Izjava A velja \textbf{natanko tedaj, ko} velja izjava B.}
                \end{align*}
             


            % \begin{columns}
                % \column{0.62\textwidth} 
                      
                        Ekvivalenca dveh izjav je pravilna, če imata obe izjavi enako vrednost 
                        (ali sta obe pravilni ali obe nepravilni), in nepravilna, če imata izjavi
                        različno vrednost.
                     
                      
                        Ekvivalentni/enakovredni izjavi pomenita eno in isto, lahko ju nadomestimo 
                        drugo z drugo.
                     

                % \column{0.35\textwidth} 
                    \begin{table}[H]
                        \centering
                        \begin{tabular}{||c|c|c||} 
                        \hhline{|t:===:t|}
                        \rowcolor[rgb]{0.843,0.718,0.718} $A$ & $B$ & $A\Leftrightarrow B$  \\ 
                        \hhline{|:===:|}
                        $P$ & $P$ & $P$                         \\ 
                        \hline
                        $P$ & $N$ & $N$                         \\ 
                        \hline
                        $N$ & $P$ & $N$                         \\ 
                        \hline
                        $N$ & $N$ & $P$                         \\
                        \hhline{|b:===:b|}
                        \end{tabular}
                    \end{table}

            % \end{columns}
         

         
             \begin{naloga}
                Določite, ali so naslednje izjave pravilne.
                \begin{itemize}
                    \item Število je deljivo z $12$ natanko takrat, ko je deljivo s $3$ in $4$ hkrati.
                    \item Število je deljivo s $24$ natanko takrat, ko je deljivo s $4$ in $6$ hkrati.
                    \item Število je praštevilo natanko takrat, ko ima natanko dva delitelja.
                    \item Štirikotnik je kvadrat natanko tedaj, ko se diagonali sekata pod pravim kotom.
                    \item Število je sodo natanko tedaj, ko je deljivo z $2$.
                \end{itemize}
            \end{naloga}
         

         
             \subsection{Vrstni red operacij}
                Kadar so izjave povezane z več izjavnimi povezavami, pri določanju logične 
                vrednosti upoštevamo oklepaje in naslednji \textbf{vrstni red} oziroma
                \textbf{prioriteto izjavnih povezav}:
                \begin{enumerate}
                    \item negacija,
                    \item konjunkcija,
                    \item disjunkcija,
                    \item implikacija,
                    \item ekvivalenca.
                \end{enumerate}
             
              
                Če moramo zapored izvesti več enakih izjavnih povezav, velja pravilo združevanja 
                od leve proti desni.
             
         

         
             \begin{naloga}
                V sestavljeni izjavi zapišite oklepaje, ki bodo predstavljali vrstni red operacij.
                Nato tvorite pravilnostno tabelo za sestavljeno izjavo glede na različne logične 
                vrednosti elementarnih izjav.
                \begin{itemize}
                    \item $A\lor B\Leftrightarrow \lnot A\Rightarrow \lnot B$
                    \item $A\lor \lnot A\Rightarrow\lnot B\land (\lnot A\Rightarrow B) $
                    \item $A\Rightarrow B\Leftrightarrow \lnot B\Rightarrow \lnot A $
                    \item $A\land \lnot B\Leftrightarrow A \Rightarrow B$
                    \item $C\Rightarrow A\lor \lnot B\Leftrightarrow \lnot A\land C $
                    \item $\lnot A\lor \lnot B\Leftrightarrow B \land (C\Leftrightarrow \lnot A) $
                \end{itemize}
            \end{naloga}
         

         
             \subsection{Tavtologija in protislovje}

                \textbf{Tavtologija} ali \textbf{logično pravilna izjava} je sestavljena izjava, 
                ki je pri vseh naborih vrednosti elementarnih izjav, iz katerih je sestavjena, pravilna.
             
                \textbf{Protislovje} je sestavljena izjava, ki ni nikoli pravilna.
             

             \subsection{Kvantifikatorja}
                \begin{itemize}
                    \item $\forall$ (beri 'vsak') -- izjava velja za vsak element dane množice
                    \item $\exists$ (beri 'obstaja' ali 'eksistira') -- izjava je pravilna za vsaj en element dane množice
                \end{itemize}
             

         

         
             \section{Pomen izjav v matematiki}
              
                \textbf{Aksiomi} so najpreprostejše izjave, ki so očitno pravilne in zato njihove 
                pravilnosti ni treba dokazovati.
             
              
                \textbf{Izreki} ali \textbf{teoremi} so izjave, ki so pravilne, vendar pa njihova 
                pravilnost ni očitna. 
                Pravilnost izreka (teorema) moramo potrditi z dokazom, ki temelji na aksiomih in na 
                preprostejših že prej dokazanih izrekih.
             
              
                \textbf{Definicije} so izjave, s katerimi uvajamo nove pojme. Najpreprostejših pojmov 
                v matematiki ne opisujemo z definicijami (to so pojmi kot npr.: število, premica ipd.); 
                vsak nadaljnji pojem pa moramo definirati, zato da se nedvoumno ve, o čem govorimo.
             

         
        
    % \chapter{Osnove teorije množic}

         
         