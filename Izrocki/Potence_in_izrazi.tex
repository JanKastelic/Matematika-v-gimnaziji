\chapter{Potence in izrazi}
\section{Potence z naravnim eksponentom}


            Potenca $\mathbf{x^n}$ z \textbf{osnovo}/\textbf{bazo} $x$ in \textbf{eksponentom}/\textbf{stopnjo} $n \in \mathbb{N}$, je produkt $n$ faktorjev enakih $x$.

            $$ \mathbf{x^n=\underbrace{x\cdot x\cdot \ldots \cdot x}_\text{n faktorjev}}  $$
        
            % \begin{figure}[H]
            %     \centering
                
            %     \begin{tikzpicture}[scale=0.75]

            %         \draw[thick] (0,0)--(2,0); 
            %         \fill[red!40] (3,0) rectangle (5,2);
            %         \draw[black,thick] (3,0) rectangle (5,2);
            %         \node at (1,0.15) {$x$};


            %         \fill[red!40] (6,0) rectangle (8,2);
            %         \fill[red!40] (7,2.5)--(6,2)--(8,2)--(9,2.5)--cycle;
            %         \fill[red!40] (8,0)--(9,0.5)--(9,2.5)--(8,2)--cycle;
            %         \node at (4,1) {$x^2$};


            %         \draw[black,thick] (6,0) rectangle (8,2);
            %         \draw[black,thick] (7,2.5)--(6,2)--(8,2)--(9,2.5)--cycle;
            %         \draw[black,thick] (8,0)--(9,0.5)--(9,2.5)--(8,2)--cycle;
            %         \node at (7,1) {$x^3$};
                
            %     \end{tikzpicture}
            % \end{figure}
    

    
    
        \section{Pravila za računanje s potencami}

                    Dve potenci z isto osnovo zmnožimo tako, da osnovo ohranimo, eksponenta pa seštejemo.

            $$ x^n \cdot x^m=\underbrace{(x\cdot x\cdot\ldots\cdot x)}_\text{n faktorjev}\cdot\underbrace{(x\cdot x\cdot\ldots\cdot x)}_\text{m faktorjev}=x^{n+m}$$
            
            Potenco potenciramo tako, da osnovo ohranimo, ekponenta pa zmnožimo.

            $$ (x^n)^m=\underbrace{\underbrace{(x\cdot x\cdot\ldots\cdot x)}_\text{n faktorjev}\cdot\underbrace{(x\cdot x\cdot\ldots\cdot x)}_\text{n faktorjev}\cdot\ldots\cdot\underbrace{(x\cdot x\cdot\ldots\cdot x)}_\text{n faktorjev}}_\text{m faktorjev}=x^{n\cdot m}$$
            
    

    
            Produkt dveh ali več števil potenciramo tako, da potenciramo posamezne faktorje in jih potem zmnožimo.

            $$ (xy)^n =\underbrace{(xy\cdot xy\cdot\ldots\cdot xy)}_\text{n faktorjev}=\underbrace{(x\cdot x\cdot\ldots\cdot x)}_\text{n faktorjev}\cdot\underbrace{(y\cdot y\cdot\ldots\cdot y)}_\text{n faktorjev}=x^n y^n$$
            

            Za naravne eksponente velja še:
            $$(-x)^{2n}=x^{2n}$$
            $$(-x)^{2n+1}=-x^{2n+1}$$

            $$(-1)^n=\begin{cases}
                1; &n=2k \\
                -1; &n=2k-1
            \end{cases}; k\in\mathbb{N}$$
\newline ~\newline
            
            \begin{naloga}
                    Števila $-3^2$, $(-4)^2$, $-2^4$, $(-1)^{2024}$, $(-2)^3$ in $(-3)^2$ uredite po velikosti od najmanjšega do največjega. 
            \end{naloga}

                \begin{naloga}
                    Poiščite podatke in jih zapišite na dva načina: s potenco in številom brez potence.
                    \begin{itemize}
                        \item Razdalja med Zemljo in Soncem
                        \item Zemljina masa
                        \item Masa Sonca
                        \item Število zvezd v naši Galaksiji
                    \end{itemize}
                \end{naloga}
    
                \begin{naloga}
                    Izračunajte.
                    \begin{itemize}
                        \item $(-3)^2+2^4$ 
                        \item $(5-3)^3+(-3)^2$ 
                        \item $(2^2+1)^2+(-3)^3+(-2)^4$ 
                        \item $(-1)^{2024}+((-2)^5+5^2-(7-3^2)^3)^2$ 
                        \item $-1^{2n-1}+(-1)^{2n-1}$ 
                    \end{itemize}
                \end{naloga}

                \begin{naloga}
                        Poenostavite izraz.
                    \begin{itemize}
                        \item $2^7\cdot 2^3$ 
                        \item $a^3\cdot a^{12}\cdot a^5$ 
                        \item $(2z)^3$ 
                        \item $(m^2\cdot m^4)^3$ 
                        \item $a^3+2a^3-6a^3$ 
                        \item $x^2\cdot x^4+(-2x^3)^2-2(-x)^6$ 
                    \end{itemize}
                \end{naloga}
