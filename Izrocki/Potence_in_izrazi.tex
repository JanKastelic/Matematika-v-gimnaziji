\chapter{Potence in izrazi}
\section{Potence z naravnim eksponentom}


            Potenca $\mathbf{x^n}$ z \textbf{osnovo}/\textbf{bazo} $x$ in \textbf{eksponentom}/\textbf{stopnjo} $n \in \mathbb{N}$, je produkt $n$ faktorjev enakih $x$.

            $$ \mathbf{x^n=\underbrace{x\cdot x\cdot \ldots \cdot x}_\text{n faktorjev}}  $$
        
            % \begin{figure}[H]
            %     \centering
                
            %     \begin{tikzpicture}[scale=0.75]

            %         \draw[thick] (0,0)--(2,0); 
            %         \fill[red!40] (3,0) rectangle (5,2);
            %         \draw[black,thick] (3,0) rectangle (5,2);
            %         \node at (1,0.15) {$x$};


            %         \fill[red!40] (6,0) rectangle (8,2);
            %         \fill[red!40] (7,2.5)--(6,2)--(8,2)--(9,2.5)--cycle;
            %         \fill[red!40] (8,0)--(9,0.5)--(9,2.5)--(8,2)--cycle;
            %         \node at (4,1) {$x^2$};


            %         \draw[black,thick] (6,0) rectangle (8,2);
            %         \draw[black,thick] (7,2.5)--(6,2)--(8,2)--(9,2.5)--cycle;
            %         \draw[black,thick] (8,0)--(9,0.5)--(9,2.5)--(8,2)--cycle;
            %         \node at (7,1) {$x^3$};
                
            %     \end{tikzpicture}
            % \end{figure}
    

    
    
        \section{Pravila za računanje s potencami}

                    Dve potenci z isto osnovo zmnožimo tako, da osnovo ohranimo, eksponenta pa seštejemo.

            $$ x^n \cdot x^m=\underbrace{(x\cdot x\cdot\ldots\cdot x)}_\text{n faktorjev}\cdot\underbrace{(x\cdot x\cdot\ldots\cdot x)}_\text{m faktorjev}=x^{n+m}$$
            
            Potenco potenciramo tako, da osnovo ohranimo, ekponenta pa zmnožimo.

            $$ (x^n)^m=\underbrace{\underbrace{(x\cdot x\cdot\ldots\cdot x)}_\text{n faktorjev}\cdot\underbrace{(x\cdot x\cdot\ldots\cdot x)}_\text{n faktorjev}\cdot\ldots\cdot\underbrace{(x\cdot x\cdot\ldots\cdot x)}_\text{n faktorjev}}_\text{m faktorjev}=x^{n\cdot m}$$
            
    

    
            Produkt dveh ali več števil potenciramo tako, da potenciramo posamezne faktorje in jih potem zmnožimo.

            $$ (xy)^n =\underbrace{(xy\cdot xy\cdot\ldots\cdot xy)}_\text{n faktorjev}=\underbrace{(x\cdot x\cdot\ldots\cdot x)}_\text{n faktorjev}\cdot\underbrace{(y\cdot y\cdot\ldots\cdot y)}_\text{n faktorjev}=x^n y^n$$
            

            Za naravne eksponente velja še:
            $$(-x)^{2n}=x^{2n}$$
            $$(-x)^{2n+1}=-x^{2n+1}$$

            $$(-1)^n=\begin{cases}
                1; &n=2k \\
                -1; &n=2k-1
            \end{cases}; k\in\mathbb{N}$$
\newline ~\newline
            
            \begin{naloga}
                    Števila $-3^2$, $(-4)^2$, $-2^4$, $(-1)^{2024}$, $(-2)^3$ in $(-3)^2$ uredite po velikosti od najmanjšega do največjega. 
            \end{naloga}

                \begin{naloga}
                    Poiščite podatke in jih zapišite na dva načina: s potenco in številom brez potence.
                    \begin{itemize}
                        \item Razdalja med Zemljo in Soncem
                        \item Zemljina masa
                        \item Masa Sonca
                        \item Število zvezd v naši Galaksiji
                    \end{itemize}
                \end{naloga}
    
                \begin{naloga}
                    Izračunajte.
                    \begin{itemize}
                        \item $(-3)^2+2^4$ 
                        \item $(5-3)^3+(-3)^2$ 
                        \item $(2^2+1)^2+(-3)^3+(-2)^4$ 
                        \item $(-1)^{2024}+((-2)^5+5^2-(7-3^2)^3)^2$ 
                        \item $-1^{2n-1}+(-1)^{2n-1}$ 
                    \end{itemize}
                \end{naloga}

                \begin{naloga}
                        Poenostavite izraz.
                    \begin{itemize}
                        \item $2^7\cdot 2^3$ 
                        \item $a^3\cdot a^{12}\cdot a^5$ 
                        \item $(2z)^3$ 
                        \item $(m^2\cdot m^4)^3$ 
                        \item $a^3+2a^3-6a^3$ 
                        \item $x^2\cdot x^4+(-2x^3)^2-2(-x)^6$ 
                    \end{itemize}
                \end{naloga}

                
        \begin{naloga}
            Izračunajte, rezultat zapišite s potenco.
            \begin{itemize}
                \item $2\cdot 10^3\cdot 3\cdot 10^2\cdot 5\cdot 10^6$ 
                \item $(10^3)^2\cdot5\cdot 10^4\cdot 2\cdot 10^3$ 
                \item $(-2)^3\cdot 2^7$ 
                \item $-2^3\cdot (-2)^4\cdot 2^3$ 
                \item $2^3\cdot(-3)^2\cdot 6^4\cdot 3$ 
                \item $(-3)^3\cdot(-7)^2\cdot 21^7\cdot 7$ 
            \end{itemize}
        \end{naloga}

        \begin{naloga}
            Poenostavite.
            \begin{itemize}
                \item $2^3\cdot 3^4\cdot(2^4\cdot 3^2)^5$ 
                \item $(5^2\cdot 7)^3\cdot 5^2\cdot 7^3$ 
                \item $(-2^3\cdot 3^5)^4\cdot 2^6\cdot 3^5$ 
                \item $(-4)^2\cdot(-7)^{13}\cdot (-28)^5\cdot (-7^2)^3$ 
                \item $-6^2\cdot(-3)^2\cdot 8^5\cdot (-3^2)^3$ 
            \end{itemize}
        \end{naloga}

        \begin{naloga}
            Poenostavite.
            \begin{itemize}
                \item $a^3\cdot b^2\cdot a^7\cdot b^3\cdot b^5$ 
                \item $4x^4\cdot(2x^3)^2$ 
                \item $(k^3\cdot 2h^5)^2$ 
                \item $(x^2y^4)^2\cdot (x^3y)^3$ 
                \item $(a^2b^5)^3(ab^3)^2$ 
                \item $x^2y^3(x^3y^6)^2$ 
            \end{itemize}
        \end{naloga}

    
        \begin{naloga}
            Poenostavite.
            \begin{itemize}
                \item $2^3\cdot x^2\cdot 3^2\cdot(-x)^6$ 
                \item $(-a^3b)^4(-a^2b^5a^3)^3$ 
                \item $(2s^2\cdot(-s^2)^5)^5$ 
                \item $(-2(z^4)^2(-2z)^3z^5)^3$ 
                \item $(-3ab^2)^3(-a^4b^2(a^3)^5)^2(ab^3)^2$ 
                \item $(xy^2z)^3(x^3(-y^2)^5(-z))^3(x^2y^3(-z^2)^3)$ 
            \end{itemize}
        \end{naloga}

        \begin{naloga}
            Odpravite oklepaje in poenostavite, če je mogoče.
            \begin{itemize}
                \item $a^n\cdot a^{n+2}\cdot(-a)^3$ \\ ~
                \item $(-x^n)^4\cdot x^2$ 
                \item $a^n\cdot(a^2-a^3+2)$ 
                \item $(x^2+3x^n-5)\cdot x^{n+1}$ 
            \end{itemize}
        \end{naloga}


        \begin{naloga}
            Poenostavite.
            \begin{itemize}
                \item $(2s(g^2)^2)^2-3(s^4g)g^7$ 
                \item $(-4x^2xy^3)^2+(xy)^5(-2^3xy)$
                \item $a^2(a^3-b^2)-a^5+(-a)^2b^2$ 
                \item $(p^2(q^3)^2)^2-2p^4q^{12}+7(-p^3p)(q^4)^3-(-2)^3(pq^3)^4$ 
            \end{itemize}
        \end{naloga}

        \begin{naloga}
            Poenostavite.
            \begin{itemize}
                \item $5a^{n+1}+4a^{n+1}-6a^{n+1}$ 
                \item $3x^{n+2}+5x^n\cdot x^2+2x\cdot x^{n+1}$
                \item $3^{5x}\cdot 9^x-3^{7x}+27^x\cdot 9^{2x}$ 
                \item $4^{2y}+3\cdot(2^y)^4-5\cdot 8^y\cdot 2^y$ 
                \item $5^p\cdot 125^p\cdot 25^p+2(5^p)^6-4\cdot 25^{3p}$ 
            \end{itemize}
        \end{naloga}




\newpage
        \section{Večkratniki}

        
            
            
                \textbf{Večkratnik} ali tudi \textbf{$k$-kratnik} števila $x$ je vsota $k$ enakih sumandov $x$:
                    $${k\cdot x=\underbrace{x+x+\ldots+x}_\text{$k$ sumandov}}.$$
            
    
            
                Vse večkratnike števila $x$ dobimo tako, da število $x$ zapored pomnožimo z vsemi celimi števili:
                $$\left\{\ldots,-5x, -4x, -3x, -2x, -x, 0, x, 2x, 3x, 4x, 5x, \ldots\right\}=\left\{kx;\ k,x\in\mathbb{Z}\right\}. $$
            
    
            
                Število $\mathbf{k}$ je \textbf{koeficient} števila oziroma spremenljivke $x$.
            
        
    
        \section{Algebrski izrazi}
    
        
    
            
                \textbf{Algebrski izraz} ali \textbf{izraz} je smiseln zapis sestavljen iz:
                \begin{itemize}
                    \item števil,
                    \item spremenljivk/parametrov, ki predstavljajo števila in jih označujemo s črkami,
                    \item oznak računskih operacij in funkcij, ki jih povezujejo,
                    \item oklepajev, ki določajo vrstni red računanja. 
                \end{itemize}
            ~
    
            
                Če v izraz namesto spremenljivk vstavimo konkretna števila in izračunamo rezultat, dobimo \textbf{vrednost izraza} (pri dani izbiri spremenljivk).
    
            ~
    
            
                Dva matematična izraza sta \textbf{enakovredna}, če imata pri katerikoli izbiri spremenljivk vedno enako vrednost.
    
            
        
    
        
        \section{Računanje z algebrskimi izrazi}
    
    
    
        
            
                Pri poenostavljanju izrazov veljajo vsi računski zakoni, ki veljajo za računanje s števili.
            
    
                \subsubsection*{Komutativnost seštevanja}
                    $$ \mathbf{x+ y=y+ x}$$
                
    
                    \subsubsection*{Asociativnost seštevanja}
                    $$ \mathbf{(x+ y)+ z=x+ (y+ z)}$$
                
    
    
                    \subsubsection*{Komutativnost množenja}
                   $$ \mathbf{x\cdot y=y\cdot x}$$
                
    
                   \subsubsection*{Asociativnost množenja}
                    $$ \mathbf{(x\cdot y)\cdot z=x\cdot (y\cdot z)}$$
                
    
    
                \subsubsection*{Distributivnost seštevanja in množenja}
                    $$ (x+y)\cdot z=\mathbf{x\cdot z+y\cdot z} $$
                
        
    
        
            
                Če v distributivnostnem zakonu zamenjamo levo in desno stran, dobimo pravilo o \textbf{izpostavljanju skupnega faktorja}: $xz+yz=(x+y)z$.
            
    
            \subsection{Seštevanje in izpostavljanje izrazov}
                Med seboj lahko seštevamo samo člene, ki se razlikujejo kvečjemu v koeficientu. To naredimo tako, da seštejemo koeficienta.
                $$mx^2+ny+kx^2+ly=mx^2+kx^2+ny+ly=(m+k)x^2+(n+l)y $$
            
    
            \subsection{Množenje izrazov}
                Dva izraza zmnožimo tako, da vsak člen prvega izraza zmnožimo z vsakim členom drugega izraza. Potem pa seštejemo podobne člene.
                $$(x+y)(z+w)=xz+xw+yz+yw $$
            
        
    
        
            \begin{naloga}
                Poenostavite.
                \begin{itemize}
                    \item $3a+2b-a+7b$ 
                    \item $2a^2b-ab^2+3a^2b$ 
                    \item $5a^4-(2a)^4+(-3a^2)^2-3(a^2)^2$ 
                    \item $3(a-2(a+b))-2(b-a(-2)^2)$ 
                \end{itemize}
            \end{naloga}
        
    
        
            \begin{naloga}
                Zapišite izraz.
                \begin{itemize}
                    \item Kvadrat razlike števil $x$ in $y$. 
                    \item Razlika kvadratov števil $x$ in $y$. 
                    \item Razlika petkratnika $m$ in kvadrata števila $3$. 
                    \item Kub razlike sedemkratnika števila $x$ in trikratnika števila $y$. 
                \end{itemize}
            \end{naloga}
        
    
        
            \begin{naloga}
                Izpostavite skupni faktor.
                \begin{itemize}
                    \item $3x+12y^2$ 
                    \item $m^3+8mp$ 
                    \item $22a^3-33ab$ 
                    \item $kr^2-rk^2$ 
                    \item $4u^2v^3-6uv^2$ 
                    \item $12a^2b-8(ab)^2-(2ab)^4$ 
                \end{itemize}
            \end{naloga}
        
    
        
            \begin{naloga}
                Izpostavite skupni faktor.
                \begin{itemize}
                    \item $3x(x+1)+5(x+1)$ 
                    \item $(a-1)(a+1)+(a-1)$ 
                    \item $4(m-1)-(1-m)(a+b)$ 
                    \item $3(c-2)+5c(2-x)$ 
                    \item $(-y+x)3a-(y-x)b$ 
                \end{itemize}
            \end{naloga}
        
    
        
            \begin{naloga}
                Izpostavite skupni faktor.
                \begin{itemize}
                    \item $5^{11}-5^{10}+5^9$ 
                    \item $2\cdot 3^8+5\cdot 3^6$ 
                    \item $4\cdot 5^{10}-10\cdot 5^8-8\cdot 5^9$ 
                    \item $7^5-7^6+7\cdot 7^4$ 
                \end{itemize}
            \end{naloga}
        
    
        
            \begin{naloga}
                Izpostavite skupni faktor.
                \begin{itemize}
                    \item $3^n-2\cdot 3^{n+1}+3^{n+2}$ 
                    \item $2^{k+2}-2^k$ 
                    \item $5\cdot 3^m+2\cdot 3^{m+1}$ 
                    \item $2^{n-3}+3\cdot 2^{n-2}-2^{n-1}$ 
                    \item $3\cdot 5^{n+1}-5^{n+2}+4\cdot 5^{n+3}$ 
                    \item $7^n+2\cdot 7^{n-1}-3\cdot 7^{n+1}$ 
                \end{itemize}
            \end{naloga}
        
    
        
            \begin{naloga}
                Izpostavite skupni faktor in izračunajte.
                \begin{itemize}
                    \item $2^{2n}+4^n+(2^n)^2$ 
                    \item $5^{2n+1}-25^n+3\cdot 5^{2n-1}$ 
                    \item $5\cdot 2^{3n}-3\cdot 8^{n-1}$ 
                    \item $49^n-2\cdot 7^{2n-1}$ 
                \end{itemize}
            \end{naloga}
        
    
        
            \begin{naloga}
                Izpostavite skupni faktor.
                \begin{itemize}
                    \item $4a^n+6a^{n+1}$ 
                    \item $b^n+b^{n+1}-2b^{n-1}$ 
                    \item $a^{n-3}+5a^n$ 
                    \item $3x^{n+1}-15x^n+18x^{n-1}$ 
                \end{itemize}
            \end{naloga}
        
    
        
            \begin{naloga}
                Zmnožite.
                \begin{itemize}
                    \item $(x-3)(x+2)$ 
                    \item $(2m+3)(5m-1)$ 
                    \item $(1-a)(1+a)$ 
                    \item $(x-3y)(2x+y)$ 
                    \item $(m-2k)(3m-k)$ 
                \end{itemize}
            \end{naloga}
        
    
        
            \begin{naloga}
                Zmnožite.
                \begin{itemize}
                    \item $(a+b-1)(a-b)$ 
                    \item $(2x+y)(3x-4y+5)$ 
                    \item $(m+2n-k)(m+2n+k)$ 
                \end{itemize}
            \end{naloga}
        
    
        
            \begin{naloga}
                Zmnožite.
                \begin{itemize}
                    \item $(x^2-3)(x^3+2)$ 
                    \item $(3x^2-y)(5y^4-7x^3)$ 
                    \item $(u^3-1)(u^3+1)$ 
                    \item $(a^5b^2-4b)(3a^7+2a^2b)$ 
                    \item $(a-b)(a^2+ab+b^2)$ 
                    \item $(z+w)(z^2-zw+w^2)$ 
                \end{itemize}
            \end{naloga}
        
    
        
            \begin{naloga}
                Poenostavite.
                \begin{itemize}
                    \item $(2x-y)(3+y)+(y-4)(y+4)-2xy+3(y-2x+5)$ 
                    \item $(x-y)(x+y)-(x^2+xy+y^2)(x-y)-(1-x)x^2+(-y)y^2$ 
                    \item $2ab+(a-3b^2)(a+3b^2)+2^3(-b^2)^2-(a-b)(b-a)-2a^3$  
                \end{itemize}
            \end{naloga}
        
    