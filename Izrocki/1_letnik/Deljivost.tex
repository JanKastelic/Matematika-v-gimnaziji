\chapter{Deljivost}


    \section{Relacija deljivosti}
            
        Naravno število $m$ je \textbf{delitelj} naravnega števila $n$ (\textbf{deljenec}), če obstaja naravno število $k$ (\textbf{kvocient}), da velja: $$\mathbf{n=k\cdot m}.$$
    
        ~\newline
        Naravno število $m$ deli naravno število $n$, ko je število $n$ večkratnik števila $m$. $$m\mid n \Leftrightarrow n=k\cdot m;\quad m,n,k\in\mathbb{N}$$
    
        ~\newline
        Število $m$ je delitelj samega sebe in vseh svojih večkratnikov.
    
        $1$ je delitelj vsakega naravnega števila.
        ~\newline

        Če $d$ deli naravni števili $m$ in $n$, $n>m$, potem $d$ deli tudi vsoto in razliko števil $m$ in $n$.
    
        ~\newline
        Pri deljenju poljubnega naravnega števila $n$ z naravnim številom $m$ imamo dve možnosti: $n$ je deljivo z $m$ ali $n$ ni deljivo z $m$.

        ~\newline
        Relacija deljivosti je:
        \begin{enumerate}
            \item \textbf{refleksivna}: $$a\mid a;$$
            \item \textbf{antisimetrična}: $$a\mid b \wedge b\mid a \Rightarrow a=b;$$
            \item \textbf{tranzitivna}:  $$a\mid b \wedge b\mid c \Rightarrow a\mid c.$$
        \end{enumerate}
    
        Relacija s temi lastnostmi je relacija \textbf{delne urejenosti}, zato relacija deljivosti delno ureja množico $\mathbb{N}$.
    
        ~\newline~
    
        \begin{naloga}
            Zapišite vse delitelje števil.
            \begin{itemize}
                \item $6$ 
                \item $16$ 
                \item $37$ 
                \item $48$ 
                \item $120$ 
            \end{itemize}
        \end{naloga}        

    
        \begin{naloga}
            Pokažite, da trditev velja.
            \begin{itemize}
                \item Izraz $x-3$ deli izraz $x^2-2x-3$. 
                \item Izraz $x+2$ deli izraz $x^3+x^2-4x-4$. 
                \item Izraz $x-2$ deli izraz $x^3-8$. 
            \end{itemize}
        \end{naloga}        

    
        \begin{naloga}
            Pokažite, da trditev velja.
            \begin{itemize}
                \item $19\mid \left(3^{21}-3^{20}+3^{18}\right)$ 
                \item $7\mid \left(3\cdot 4^{11}+4^{12}+7\cdot 4^{10}\right)$ 
                \item $14\mid \left(5\cdot 3^6+2\cdot 3^8-3\cdot 3^7\right)$ 
                \item $25\mid \left(7\cdot 2^{23}-3\cdot 2^{24}+3\cdot 2^{25}-2^{22}\right)$ 
                \item $11\mid \left(2\cdot 10^6+3\cdot 10^7+10^8\right)$ 
                \item $35\mid \left(6^{32}-36^{15}\right)$ 
            \end{itemize}
        \end{naloga}        

    
        \begin{naloga}
            Pokažite, da trditev velja.
            \begin{itemize}
                \item $3\mid \left(2^{2n+1}-5\cdot 2^{2n}+9\cdot 2^{2n-1}\right)$ 
                \item $29\mid \left(5^{n+3}-2\cdot 5^{n+1}+7\cdot 5^{n+2}\right)$ 
                \item $10\mid \left(3\cdot 7^{4n-1}-4\cdot 7^{4n-2}+7^{4n+1}\right)$ 
                \item $10\mid \left(9^{3n-1}+9\cdot 9^{3n+1}+9^{3n}-9^{3n+2}\right)$ 
                \item $5\mid \left(7\cdot 2^{4n-2}+3\cdot 4^{2n}-16^n\right)$ 
            \end{itemize}
        \end{naloga}        


    
        \begin{naloga}
            Pokažite, da je za poljubno naravno število $u$ vrednost izraza $$(u+7)(7-u)-3(3-u)(u+5)$$ večkratnik števila $4$.
        \end{naloga}        

\newpage
    \section{Kriteriji deljivost}
    
        \subsection*{Deljivost z $2$}
            Število je deljivo z $2$ natanko takrat, ko so enice števila deljive z $2$.

        \subsection*{Deljivost s $3$}
            Število je deljivo s $3$ natanko takrat, ko je vsota števk števila deljiva s $3$.

        \subsection*{Deljivost s $4$ oziroma $25$}
            Število je deljivo s $4$ oziroma $25$ natanko takrat, ko je dvomestni konec števila deljiv s $4$ oziroma~$25$.

        \subsection*{Deljivost s $5$}
            Število je deljivo s $5$ natanko takrat, ko so enice števila enake $0$ ali $5$.
    
        \subsection*{Deljivost s $6$}
            Število je deljivo s $6$ natanko takrat, ko je deljivo z $2$ in s $3$ hkrati.

        \subsection*{Deljivost z $8$ oziroma s $125$}
            Število je deljivo z $8$ oziroma s $125$ natanko takrat, ko je trimestni konec števila deljiv z~$8$ oziroma s $125$.

        \subsection*{Deljivost z $9$}
            Število je deljivo z $9$ natanko takrat, ko je vsota števk števila deljiva z $9$.

        \subsection*{Deljivost z $10$ oziroma $10^n$}
            Število je deljivo z $10$ natanko takrat, ko so enice števila enake $0$.
            \\Število je deljivo z $10^n$ natanko takrat, ko ima število na zadnjih $n$ mestih števko $0$.
    
        \subsection*{Deljivost z $11$}
            Število je deljivo z $11$ natanko takrat, ko je alternirajoča vsota števk tega števila deljiva z $11$.

        \subsection*{Deljivost s $7$}
            Algoritem za preverjanje deljivosti s $7$:
            \begin{enumerate}
                \item vzamemo enice danega števila in jih pomnožimo s $5$,
                \item prvotnemu številu brez enic prištejemo dobljeni produkt,
                \item vzamemo enice dobljene vsote in jih pomnožimo s $5$,
                \item produkt prištejemo prej novo dobljenemu številu ...     
            \end{enumerate}
            Postopek ponavljamo, dokler ne dobimo dvomestnega števila -- 
            če je to deljivo s $7$, je prvotno število deljivo s $7$. 
            
    

    
        \begin{naloga}
            S katerimi od števil $2$, $3$, $4$, $5$, $6$, $7$, $8$, $9$, $10$, $11$ so deljiva naslednja števila?
            \begin{itemize}
                \item $84742$ 
                \item $393948$ 
                \item $12390$ 
                \item $19401$ 
            \end{itemize}
        \end{naloga}
    

    
        \begin{naloga}
            Določite vse možnosti za števko $a$, da je število $\overline{65833a}$:
            \begin{itemize}
                \item deljivo s $3$, 
                \item deljivo s $4$, 
                \item deljivo s $5$, 
                \item deljivo s $6$. 
            \end{itemize}
        \end{naloga}
    

    
        \begin{naloga}
            Določite vse možnosti za števko $b$, da je število $\overline{65b90b}$:
            \begin{itemize}
                \item deljivo z $2$, 
                \item deljivo s $3$, 
                \item deljivo s $6$, 
                \item deljivo z $9$, 
                \item deljivo z $10$. 
            \end{itemize}
        \end{naloga}
    

    
        \begin{naloga}
            Določite vse možnosti za števki $c$ in $d$, da je število $\overline{115c1d}$ deljivo s $6$.
            
        \end{naloga}

        \begin{naloga}
            Določite vse možnosti za števki $e$ in $f$, da je število $\overline{115e1f}$ deljivo z $8$.
            
        \end{naloga}

    


    
        \begin{naloga}
            Pokažite, da za vsako naravno število $n$ $12$ deli $n^4-n^2$.
            
        \end{naloga}

        \begin{naloga}
            Preverite, ali je število $8641 969$ deljivo s $7$.
            
        \end{naloga}
        
            
    

\newpage
\section{Osnovni izrek o deljenju}

        

            \subsection*{Osnovni izrek o deljenju}
                Za poljubni naravni števili $\mathbf{m}$ (\textbf{deljenec}) in $\mathbf{n}$ (\textbf{delitelj}), $m\geq n$, 
                obstajata natanko določeni nenegativni števili $\mathbf{k}$ (\textbf{količnik}/\textbf{kvocient}) in $\mathbf{r}$ (\textbf{ostanek}), 
                da velja:
                $$m=k\cdot n+r; \quad  0\leq r<n; \quad m,n\in\mathbb{N}; k,r\in\mathbb{N}_0.$$
            

            
                Če je ostanek pri deljenju enak $0$, je število $m$ \textbf{večkratnik} števila $n$. 
                Tedaj je število $m$ deljivo s številom $n$. Pravimo, da $n$ deli število $m$: $n\mid m$.
        

        
            \begin{naloga}
                Določite, katera števila so lahko ostanki pri deljenju naravnega števila $n$ s:
                \begin{itemize}
                    \item številom $3$; 
                    \item številom $7$; 
                    \item številom $365$. 
                \end{itemize}
            \end{naloga}

            \begin{naloga}
                Zapišite prvih nekaj naravnih števil, ki dajo:
                \begin{itemize}
                    \item pri deljenju s $4$ ostanek $3$; 
                    \item pri deljenju s $7$ ostanek $4$; 
                    \item pri deljenju z $9$ ostanek $4$. 
                \end{itemize}
            \end{naloga}
        
            \begin{naloga}
                Zapišite naravno število, ki da:
                \begin{itemize}
                    \item pri deljenju s $7$ količnik $5$ in ostanek $3$; 
                    \item pri deljenju z $10$ količnik $9$ in ostanek $1$; 
                    \item pri deljenju s $23$ količnik $2$ in ostanek $22$. 
                \end{itemize}
            \end{naloga}

            \begin{naloga}
                Zapišite množico vseh naravnih števil $n$, ki dajo:
                \begin{itemize}
                    \item pri deljenju z $2$ ostanek $1$; 
                    \item pri deljenju z $2$ ostanek $0$; 
                    \item pri deljenju s $5$ ostanek $2$. 
                \end{itemize}
            \end{naloga}
        

        
            \begin{naloga}
                Katero število smo delili s $7$, če smo dobili kvocient $3$ in ostanek $5$? 
            \end{naloga}

            \begin{naloga}
                S katerim številom smo delili število $73$, če smo dobili kvocient $12$ in ostanek $1$? 
            \end{naloga}

            \begin{naloga}
                Marjeta ima čebulice tulipana, ki jih želi posaditi v več vrst. 
                V vsaki od $3$ vrst je izkopala po $8$ jamic, potem pa ugotovila, da ji bosta $2$ čebulici ostali.
                Koliko čebulic ima Marjeta?  
            \end{naloga}
        

        
            \begin{naloga}
                Če neko število delimo z $8$, dobimo ostanek $7$. Kolikšen je ostanek, če to isto število delimo s $4$? 
            \end{naloga}

            \begin{naloga}
                Če neko število delimo s $24$ dobimo ostanek $21$. Kolikšen je ostanek, če to isto število delimo s $3$? 
            \end{naloga}



            \newpage
\section{Praštevila in sestavljena števila}
                
            Glede na število deliteljev, lahko naravna števila razdelimo na tri skupine:
            \begin{itemize}
                \item \textbf{število $1$} -- število, ki ima samo enega delitelja (samega sebe);
                \item \textbf{praštevila} -- števila, ki imajo natanko dva delitelja ($1$ in samega sebe);
                \item \textbf{sestavljena števila} -- števila, ki imajo več kot dva delitelja.
            \end{itemize}
            
            $$ \mathbb{N}=\{1\}\cup \mathbb{P}\cup \{sestavljena~števila\} $$
        

            Praštevil je neskončno mnogo.
            \\

            Število $n$ je praštevilo, če ni deljivo z nobenim praštevilom, manjšim ali enakim $\sqrt{n}$.
            \\

    
        \textbf{Eratostenovo sito:}
            \begin{longtable}{|c|c|c|c|c|c|c|c|c|c|}
                \hline
                1 & 2 & 3 & 4 & 5 & 6 & 7 & 8 & 9 & 10 \\
                \hline
                11 & 12 & 13 & 14 & 15 & 16 & 17 & 18 & 19 & 20 \\
                \hline
                21 & 22 & 23 & 24 & 25 & 26 & 27 & 28 & 29 & 30 \\
                \hline
                31 & 32 & 33 & 34 & 35 & 36 & 37 & 38 & 39 & 40 \\
                \hline
                41 & 42 & 43 & 44 & 45 & 46 & 47 & 48 & 49 & 50 \\
                \hline
                51 & 52 & 53 & 54 & 55 & 56 & 57 & 58 & 59 & 60 \\
                \hline
                61 & 62 & 63 & 64 & 65 & 66 & 67 & 68 & 69 & 70 \\
                \hline
                71 & 72 & 73 & 74 & 75 & 76 & 77 & 78 & 79 & 80 \\
                \hline
                81 & 82 & 83 & 84 & 85 & 86 & 87 & 88 & 89 & 90 \\
                \hline
                91 & 92 & 93 & 94 & 95 & 96 & 97 & 98 & 99 & 100 \\
                \hline
                \end{longtable}
            
    
            

        \begin{naloga}
            Preverite, ali so števila $103, 163, 137, 197, 147, 559$ praštevila.
        \end{naloga}
            
    
    
    
\section{Osnovni izrek aritmetike}
    
    
        Vsako naravno število lahko enolično/na en sam način (do vrstnega reda faktorjev natančno) zapišemo kot produkt potenc s praštevilskimi osnovami:
        $$ n=p_1^{k_1}\cdot p_2^{k_2}\cdot\ldots\cdot p_l^{k_l};  p_i\in\mathbb{P}\land n, k_i\in\mathbb{N}.$$

    

        Zapis naravnega števila kot produkt potenc s praštevilskimi osnovami imenujemo tudi \textbf{praštevilski razcep}.
                

        ~\\
        \begin{naloga}
            Zapišite število $8755$ kot produkt samih praštevil in njihovih potenc. 
        \end{naloga}

        \begin{naloga}
            Razcepite število $3520$ na prafaktorje. 
        \end{naloga}

        \begin{naloga}
            Zapišite praštevilski razcep števila $38250$. 
        \end{naloga}

        \begin{naloga}
            Zapišite praštevilski razcep števila $3150$. 
        \end{naloga}
    
        \begin{naloga}
            Razcepite število $66$ na prafaktorje in zapišite vse njegove delitelje. 
        \end{naloga}

        \begin{naloga}
            Razcepite število $204$ na prafaktorje in zapišite vse njegove delitelje. 
        \end{naloga}
    
        \begin{naloga}
            Zapišite vse izraze, ki delijo dani izraz.
            \begin{itemize}
                \item $x^2+x-1$ 
                \item $x^3-x^2-4x+4$ 
                \item $x^3-27$ 
            \end{itemize}
        \end{naloga}

    

\newpage

\section{Največji skupni delitelj in najmanjši skupni večkratnik}

        

                \textbf{Največji skupni delitelj} števil $m$ in $n$ je največje število od tistih, ki delijo števili $m$ in $n$. 
                Oznaka: $D(m,n)$.           
                \\

                \textbf{Najmanjši skupni večkratnik} števil $m$ in $n$ je najmanjše število od tistih, ki so deljiva s številoma $m$ in $n$. 
                Oznaka: $v(m,n)$.
                \\

            
                Števili $m$ in $n$, katerih največji skupni delitelj je $1$, sta \textbf{tuji števili}.
                \\
        

        
            \textbf{Računanje $D$ in $v$ s prafaktorizacijo števil}
                \begin{itemize}
                    \item Števili $m$ in $n$ prafaktoriziramo.
                    \item Za $D(m,n)$ vzamemo potence, ki so skupne obema številom v prafaktorizaciji.
                    \item Za $v(m,n)$ vzamemo vse potence, ki se pojavijo v prafaktorizaciji števil, z največjim eksponentom.
                \end{itemize}                
            ~

            
                Za poljubni naravni števili $m$ in $n$ velja zveza $\mathbf{D(m,n)\cdot v(m,n)=m\cdot n}$.
            \\

            \textbf{Evklidov algoritem}

                V tem algoritmu zapored uporabljamo osnovni izrek o deljenju. 
                \\ Najprej ga uporabimo na danih dveh številih.
                \\ V naslednjem koraku deljenec postane prejšnji delitelj, delitelj pa prejšnji ostanek. 
                \\ V vsakem koraku imamo manjša števila, zato se algoritem konča v končno mnogo korakih.
                \\ ~\\ Največji skupni delitelj danih števil $m$ in $n$ je zadnji od $0$ različen ostanek pri deljenju v Evklidovem algoritmu.
            

        

        ~\\
            \begin{naloga}
                Izračunajte največji skupni delitelj in najmanjši skupni večkratnik danih parov števil.
                \begin{itemize}
                    \item $6$ in $8$ 
                    \item $36$ in $48$ 
                    \item $550$ in $286$ 
                    \item $6120$ in $4158$ 
                \end{itemize}
            \end{naloga}
        
            \begin{naloga}
                Preverite, ali sta števili $522$ in $4025$ tuji števili. 
            \end{naloga}

            \begin{naloga}
                Izračunajte največji skupni delitelj in najmanjši skupni večkratnik treh števil.
                \begin{itemize}
                    \item $1320$, $6732$ in $297$ 
                    \item $372$, $190$ in $11264$ 
                \end{itemize}
            \end{naloga}
        
            \begin{naloga}
                Z Evklidovim algoritmom izračunajte največji skupni delitelj parov števil.
                \begin{itemize}
                    \item $754$ in $3146$ 
                    \item $4446$ in $6325$ 
                \end{itemize}
            \end{naloga}

            \begin{naloga}
                Izračuanjte število $b$, če velja: $D(78 166, b)=418$ in $v(78 166, b)=1 485 154$. 
            \end{naloga}

            \begin{naloga}
                Določite največji skupni delitelj izrazov.
                \begin{itemize}
                    \item $x^3-5x^2-24x$ in $x^2-64$ 
                    \item $x^2+3x+10$, $x^3-4x$ in $x^3-8$ 
                    \item $x^2-25$ in $x^3-27$ 
                \end{itemize}
            \end{naloga}
~
            \begin{naloga}
                Določite najmanjši skupni večkratnik izrazov.
                \begin{itemize}
                    \item $x^2-64$ in $x+8$ 
                    \item $x$, $8-x$ in $x^2-64$ 
                    \item $x^2+3x-10$, $2x$ in $x^2+5x$ 
                \end{itemize}
            \end{naloga}

            \begin{naloga}
                Velika Janezova terasa je dolga $1035~cm$ in široka $330~cm$. Janez bi jo rad sam tlakoval s kvadratnimi vinilnimi ploščami.
                Ker ni najbolj vešč tega dela, bo kupil tako velike plošče, da mu jih ne bo treba rezati.
                Koliko so največ lahko velik kvadratne plošče? Koliko plošč bo potreboval za tlakovanje? 
            \end{naloga}

            \begin{naloga}
                Neca gre v knjižnico vsake $14$ dni, Nace pa vsakih $10$ dni. V knjižnici se srečata v ponedeljek 1. marca.
                Čez koliko dni se bosta naslednjič srečala? Na kateri dan in datum?                     
            \end{naloga}

        