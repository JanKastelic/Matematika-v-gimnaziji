\chapter{Racionalna števila}




%%% Ulomki in racionalna števila

    \section{Ulomki in racionalna števila}

        
            
                \textbf{Ulomek} $\dfrac{x}{y}$ je zapis, ki predstavlja zapis deljenja 
                $$x:y=\dfrac{x}{y};\quad y\neq 0\land x,y\in\mathbb{Z}.$$
                Število/izraz $x$ imenujemo \textbf{števec}, $y$ pa \textbf{imenovalec}, med njima je \textbf{ulomkova črta}.
            
                ~
            
                Ulomek $\dfrac{x}{0}$ ni definiran (nima pomena), saj z $0$ ne moremo deliti.
            
                ~
            
                \textbf{Algebrski ulomek} je ulomek, v katerem v števcu in/ali imenovalcu nastopajo algebrski izrazi.
            
                ~
            
                Vsako celo število $x\in\mathbb{Z}$ lahko zapišemo z ulomkom: $x=\dfrac{x}{1}$.
            
                ~
            
                \textbf{Ničelni ulomek} je ulomek oblike $\dfrac{0}{y}=0; y\neq 0$.
            
                ~
            
                V ulomku, kjer v števcu ali imenovalcu nastopa negativno število, upoštevamo enakost 
                $$-\dfrac{x}{y}=\dfrac{-x}{y}=\dfrac{x}{-y}.$$
            
                ~
            
                Vsakemu neničelnemu ulomku $\dfrac{x}{y}$ lahko priredimo njegovo \textbf{obratno vrednost}:
                $$\left(\dfrac{x}{y}\right)^{-1}=\dfrac{y}{x}; \quad x,y\in\mathbb{Z}\setminus\{0\}.$$
            

        


        
            \subsection*{Racionalna števila}

            
                Množica racionalnih števil $\mathbb{Q}$ je sestavljena iz vseh ulomkov (kar pomeni, da vsebuje tudi vsa naravna in cela števila).
            
                \begin{figure}[H]
                \centering
                \begin{tikzpicture}
                    % \clip (0,0) rectangle (14.000000,10.000000);
                    {\footnotesize
                    
                    % Drawing segment A B
                    \draw [line width=0.016cm] (1.000000,1.500000) -- (4.460000,1.500000);%
                    \draw [line width=0.016cm] (4.540000,1.500000) -- (8.000000,1.500000);%
                    
                    % Marking point 0 by circle
                    \draw [line width=0.016cm] (4.500000,1.500000) circle (0.040000);%
                    \draw (4.500000,1.500000) node [anchor=south] { $0$ };%
                    
                    
                    % Changing color 255 0 0
                    \definecolor{r255g0b0}{rgb}{1.000000,0.000000,0.000000}%
                    \color{r255g0b0}% 
                    
                    % Marking point \mathbb{Q}^+
                    \draw (6.250000,1.500000) node [anchor=south] { $\mathbb{Q}^+$ };%
                    
                    % Drawing segment B 0
                    \draw [line width=0.016cm] (8.000000,1.500000) -- (4.540000,1.500000);%
                    }

                    
                    % Changing color 0 255 0
                    \definecolor{r0g255b0}{rgb}{0.000000,1.000000,0.000000}%
                    \color{r0g255b0}% 
                    
                    % Marking point \mathbb{Q}^-
                    \draw (2.750000,1.500000) node [anchor=south] { $\mathbb{Q}^-$ };%
                    
                    % Drawing segment A 0
                    \draw [line width=0.016cm] (1.000000,1.500000) -- (4.460000,1.500000);%
                    

                    % Changing color 0 0 0
                    \definecolor{r0g0b0}{rgb}{0.000000,0.000000,0.000000}%
                    \color{r0g0b0}% 
                    
                    % Marking point \mathbb{Q}
                    \draw (1.500000,2.000000) node  { $\mathbb{Q}$ };%
                    \color{black}
                    
                    \end{tikzpicture}
                \end{figure}
                    
            

            
                Glede na predznak razdelimo racionalna števila v tri množice:
                \begin{itemize}
                    \item \textcolor{green}{množico negativnih racionalnih števil $\mathbf{\mathbb{Q}^-}$},
                    \item množico z elementom nič: $\mathbf{\{0\}}$ in
                    \item \textcolor{red}{množico pozitivnih racionalnih števil: $\mathbf{\mathbb{Q}^+}$}.
                \end{itemize}
                $$ \mathbb{Q}=\textcolor{green}{\mathbb{Q}^-}\cup\{0\}\cup\textcolor{red}{\mathbb{Q}^+} $$
            
            

            % 
            %     Množica racionalnih števil je povsod gosta, saj lahko med poljubnima racionalnima številoma vedno najdemo racionalno število (posledično je med dvema racionalnima številoma neskončno mnogo racionalnih števil).
            % 

        

        
            
                Ulomka $\dfrac{x}{y}$ in $\dfrac{z}{w}$ sta enaka/enakovredna natanko takrat, ko je $xz=wy$; $y,z\neq 0$.
                $$\dfrac{x}{y}=\dfrac{w}{z}\Leftrightarrow xz=wy; \quad y,z\neq 0$$
            

            
                Enaka/enakovredna ulomka sta različna zapisa za isto racionalno število.
            
        ~\\



%%% naloge

        
            \begin{naloga}
                Za katere vrednosti $x$ ulomek ni definiran?
                \begin{itemize}
                    \item $\frac{x-2}{x+1}$ 
                    \item $\frac{2}{x-5}$ 
                    \item $\frac{x+2}{3}$ 
                    \item $\frac{13}{2x-5}$ 
                \end{itemize}
            \end{naloga}
        

        
            \begin{naloga}
                Za katere vrednosti $x$ ima ulomek vrednost enako $0$?
                \begin{itemize}
                    \item $\frac{x-2}{x+1}$ 
                    \item $\frac{2}{x-5}$ 
                    \item $\frac{x+2}{3}$ 
                    \item $\frac{13}{2x-5}$ 
                \end{itemize}
            \end{naloga}
        

        
            \begin{naloga}
                Ali imata ulomka isto vrednost?
                \begin{itemize}
                    \item $\frac{2}{3}$ in $\frac{10}{15}$ 
                    \item $\frac{-1}{2}$ in $\frac{1}{-2}$ 
                    \item $\frac{4}{5}$ in $\frac{-8}{-10}$ 
                    \item $\frac{5}{8}$ in $\frac{8}{5}$ 
                \end{itemize}
            \end{naloga}
        

        
            \begin{naloga}
                Za kateri $x$ imata ulomka isto vrednost?
                \begin{itemize}
                    \item $\frac{x+1}{2}$ in $\frac{3}{4}$ 
                    \item $\frac{4}{2x-1}$ in $\frac{1}{3}$ 
                    \item $\frac{x+1}{2}$ in $\frac{x-1}{-3}$ 
                    \item $\frac{x+1}{x-2}$ in $\frac{2}{5}$ 
                \end{itemize}
            \end{naloga}
        

        
            \begin{naloga}
                Ali ulomka predstavljata isto vrednost?
                \begin{itemize}
                    \item $\left(\frac{1}{2}\right)^{-1}$ in $-\frac{1}{2}$ 
                    \item $\left(\frac{2}{3}\right)^{-1}$ in $\frac{3}{2}$ 
                    \item $ 1\frac{3}{7}$ in $\left(\frac{7}{10}\right)^{-1}$ 
               \end{itemize}
            \end{naloga}
        

        
            \begin{naloga}
                Ali ulomka predstavljata isto vrednost?
                \begin{itemize}
                    \item $ 2\cdot\frac{3}{4}$ in $\frac{3}{2}$ 
                    \item $ 2\frac{3}{4}$ in $\frac{3}{2}$ 
                    \item $\left(1\frac{2}{5}\right)^{-1}$ in $ 1\frac{5}{2}$ 
                    \item $\left(1\frac{2}{5}\right)^{-1}$ in $\frac{5}{7}$ 
               \end{itemize}
            \end{naloga}
        

        
            \begin{naloga}
                Zapišite s celim delom oziroma z ulomkom.
                \begin{itemize}
                            \item $\frac{14}{5}$ 
                            \item $-\frac{5}{2}$ 
                            \item $\frac{4}{3}$ 
                            \item $\frac{110}{17}$ 
                            \item $ 3\frac{5}{8}$ 
                            \item $ 2\frac{9}{2}$                  
               \end{itemize}
            \end{naloga}
        

        % 
        %     \begin{naloga}
        %         Poenostavite.
        %         \begin{itemize}
        %             \item a 
        %         \end{itemize}
        %     \end{naloga}
        % 

        % 
        %     \begin{naloga}
        %         Poenostavite.
        %         \begin{itemize}
        %             \item a 
        %         \end{itemize}
        %     \end{naloga}
        % 

        %%% Razširjanje in krajšanje ulomkov

    \section{Razširjanje in krajšanje ulomkov}

        

            \subsection*{Razširjanje ulomka}
                Ulomek ohrani svojo vrednost, če števec in imenovalec pomnožimo z istim neničelnim številom oziroma izrazom.
                Temu postopku pravimo \textbf{razširjanje ulomka}.

                $$\dfrac{x}{y}=\dfrac{x\cdot z}{y\cdot z}; \quad x\in\mathbb{Z} \land y,z\in\mathbb{Z}\setminus\{0\}$$
            

            
                Ko ulomke seštevamo ali odštevamo, jih razširimo na \textbf{najmanjši skupni imenovalec}, 
                ki je najmanjši skupni večkratnik vseh imenovalcev.
            

        

        
            \subsection*{Krajšanje ulomka}
                Vrednost ulomka se ne spremeni, če števec in imenovalec delimo z istim neničelnim številom oziroma izrazom.
                Temu postopku rečemo \textbf{krajšanje ulomka}.

                $$\dfrac{x\cdot z}{y\cdot z}=\dfrac{x}{y}; \quad x\in\mathbb{Z}\land y,z\in\mathbb{Z}\setminus\{0\} $$
            
                ~

                Ulomek $\dfrac{x}{y}$ je \textbf{okrajšan}, če je $(x,y)=1$, torej če sta števec in imenovalec tuji števili.
            
        
            ~\\

        %%% naloge

        
            \begin{naloga}
                Razširite ulomke na najmanjši skupni imenovalec.
                \begin{itemize}
                            \item $\frac{1}{3}$, $\frac{3}{5}$ in $\frac{5}{6}$ 
                            \item $\frac{2}{7}$, $1$ in $\dfrac{1}{2}$ 
                            \item $\frac{5}{6}$, $\frac{1}{2}$ in $-\frac{2}{3}$ 
                            \item $\frac{1}{5}$, $-\frac{1}{2}$ in $\frac{-1}{3}$ 
                            \item $\frac{2}{-1}$, $\frac{3}{2}$ in $\frac{1}{-3}$ 
                            \item $\frac{3}{-4}$, $\frac{-1}{2}$ in $-\frac{2}{5}$ 
                \end{itemize}
            \end{naloga}
        

        
            \begin{naloga}
                Razširite ulomke na najmanjši skupni imenovalec.
                \begin{itemize}
                            \item $\frac{1}{x-1}$, $\frac{1}{x+1}$ in $1$ 
                            \item $\frac{2}{x}$, $\frac{1}{x-3}$ in $\frac{1}{(x-3)^2}$ 
                            \item $\frac{3}{x^2-4x}$, $\frac{1}{x}$ in $\frac{2}{x-4}$ 
                            \item $\frac{4}{x-4}$, $\frac{2}{x-2}$ in $\frac{1}{x^2-6x+8}$ 
                            \item $\frac{2}{x-1}$ in $\frac{3}{1-x}$ 
                            \item $\frac{1}{2-x}$, $\frac{2}{x+2}$ in $\frac{3}{x^2-4}$ 
                \end{itemize}
            \end{naloga}
        

        
            \begin{naloga}
                Okrajšajte ulomek.
                \begin{itemize}
                    \item $\frac{100}{225}$ 
                    \item $\frac{34}{51}$ 
                    \item $\frac{121}{3}$ 
                    \item $\frac{45}{75}$ 
                \end{itemize}
            \end{naloga}
        

        
            \begin{naloga}
                Okrajšajte ulomek.
                \begin{itemize}
                            \item $\frac{x^2-4}{x^2+2x}$ 
                            \item $\frac{x^3+8}{2x+4}$ 
                            \item $\frac{x^3-1}{x^2-4x+3}$ 
                            \item $\frac{x^3-2x^2-x+2}{x^2-3x+2}$ 
                            \item $\frac{x^2-9}{3-x}$ 
                            \item $\frac{x-4}{16-x^2}$ 
                \end{itemize}
            \end{naloga}
        

        % 
        %     \begin{naloga}
        %         Poenostavite.
        %         \begin{itemize}
        %             \item a 
        %         \end{itemize}
        %     \end{naloga}
        % 

        % 
        %     \begin{naloga}
        %         Poenostavite.
        %         \begin{itemize}
        %             \item a 
        %         \end{itemize}
        %     \end{naloga}
        % 


        \newpage
%%% Seštevanje in odštevanje ulomkov

        \section{Seštevanje in odštevanje ulomkov}

        
            \subsection*{Seštevanje ulomkov}
                Ulomke \textbf{seštevamo} tako, da jih razširimo na skupni imenovalec, nato seštejemo števce, imenovalce pa prepišemo.
                $$\dfrac{x}{y}+\dfrac{z}{w}=\dfrac{xw}{yw}+\dfrac{yz}{yw}=\dfrac{xw+yz}{yw}; \quad x,z\in\mathbb{Z}\land y,w\in\mathbb{Z}\setminus\{0\} $$
            

            \subsection*{Odštevanje ulomkov}
                Ulomke \textbf{odštevamo} tako, da prištejemo nasprotni ulomek.
                $$\dfrac{x}{y}-\dfrac{z}{w}=\dfrac{x}{y}+\left(-\dfrac{z}{w}\right)=\dfrac{xw}{yw}+\dfrac{-yz}{yw}=\dfrac{xw-yz}{yw}; \quad x,z\in\mathbb{Z}\land y,w\in\mathbb{Z}\setminus\{0\} $$
            

    
        
            ~\\

%%% naloge
        
            \begin{naloga}
                Izračunajte.
                \begin{itemize}
                    \item $\frac{5}{7}+\frac{1}{14}$ 
                    \item $\frac{2}{9}-\frac{1}{3}$ 
                    \item $\frac{3}{8}+1\frac{1}{2}$ 
                    \item $1-\frac{5}{6}$ 
                \end{itemize}
            \end{naloga}
        

        
            \begin{naloga}
                Izračunajte.
                \begin{itemize}
                    \item $\left(\frac{2}{3}-2\frac{1}{4}\right)+\frac{1}{12}$ 
                    \item $\frac{2}{7}-\frac{3}{4}+\left(\frac{1}{2}-2\right)$ 
                    \item $\left(\frac{2}{3}-\left(\frac{1}{3}-3\right)+\frac{1}{4}\right)-\frac{1}{2}$ 
                    \item $1-\left(2-\left(3-4-\left(5-\frac{1}{2}\right)\right)+\frac{1}{3}\right)$ 
                \end{itemize}
            \end{naloga}
        


        
            \begin{naloga}
                Poenostavite.
                \begin{itemize}
                    \item $\frac{x}{x-1}-\frac{x}{x+1}$ 
                    \item $\frac{3}{x^2}+\frac{4}{x^3}-\frac{1}{x}$ 
                    \item $\frac{3}{x^2-4x}-\left(\frac{1}{x-4}+\frac{2}{x^2-5x+4}\right)$ 
                    \item $\frac{2}{xy}+\frac{3}{x}-\frac{2}{y}$ 
                \end{itemize}
            \end{naloga}
        


        
            \begin{naloga}
                Poenostavite.
                \begin{itemize}
                    \item $\frac{(x-3)^2+(x+3)^2}{x^2-9}-\frac{3x^2}{2x^2-x^2}$ 
                    \item $\frac{(a-3)^3-(a-1)^3+26}{6a}+\left(-\frac{1}{2}\right)^{-1}$ 
                    \item $\frac{x^3-2x^2-x+2}{-x(1-x)-2}-\left(\frac{x-1}{x}-1\right)^{-1}$ 
                    \item $\left(\frac{x}{2}-\left(\frac{x}{3}-\left(\frac{x}{4}-\frac{x}{5}\right)\right)\right)-\left(\frac{60}{x}\right)^{-1}$ 
                \end{itemize}
            \end{naloga}
        

            ~
        %%% Množenje ulomkov
        
        \section{Množenje ulomkov}


                Ulomka \textbf{množimo} tako, da števce množimo s števci, imenovalce pa množimo z imenovalci.
                $$\dfrac{x}{y}\cdot \dfrac{z}{w}=\dfrac{xz}{yw}; \quad x,z\in\mathbb{Z}\land y,w\in\mathbb{Z}\setminus\{0\}$$
                
            
                Produkt danega in njemu obratnega ulomka je enak $1$.
                $$\dfrac{x}{y}\cdot\left(\dfrac{x}{y}\right)^{-1}=\dfrac{x}{y}\cdot\dfrac{y}{x}=1$$
            
        
            ~
        %%% naloge

        
            \begin{naloga}
                Izračunajte.
                \begin{itemize}
                            \item $\frac{1}{3}\cdot \frac{3}{7}$ 
                            \item $\frac{-2}{13}\cdot \left(-\frac{39}{4}\right)$ 
                            \item $\frac{2}{5}\cdot \frac{4}{9}$ 
                            \item $2\frac{1}{3}\cdot 3\frac{3}{4}$ 
                            \item $\frac{-2}{5}\cdot 4\frac{2}{7}$ 
                            \item $3\cdot\frac{2}{3}$ 
                \end{itemize}
            \end{naloga}
        


        
            \begin{naloga}
                Poenostavite.
                \begin{itemize}
                    \item $\frac{x^2-9}{x^2+3x+9}\cdot\frac{x^3-27}{x^2-6k+9}$ 
                    \item $\frac{x^2+5x}{-x+2}\cdot\frac{2x^2-8}{x^2+7x+10}$ 
                    \item $\frac{x^3-4x^2-4x+16}{2x+4}\cdot\frac{6x}{3x-6}$ 
                    \item $2\cdot\frac{x}{x-1}\cdot\frac{x^2-1}{x^2+x}$ 
                \end{itemize}
            \end{naloga}
        


        
            \begin{naloga}
                Poenostavite.
                \begin{itemize}
                    \item $\frac{x^2-4}{x^2-1}\cdot\frac{x^3-1}{x^3+x^2+x}\cdot\frac{x^2+x}{2-x}$ 
                    \item $\left(\frac{6-x}{x^2+6x}-\frac{x}{36-x^2}\right)\cdot\left(\frac{2x-6}{x^2+6x}\right)^{-1}+\frac{x}{6-x}$ 
                    \item $\left(\left(x-y+\left(\frac{x+y}{2xy}\right)^{-1}\right)\cdot\left(\frac{1}{x+y}\right)^{-1}-2xy\right)\cdot(x-y)^{-1}$ 
                    \item $\left(xy+y^2-\frac{xy+y^2}{3xy-3x^2}\right)\cdot\left(\frac{x+y}{3x}\right)^{-1}-\left(-\frac{y-x}{y}\right)^{-1}$ 
                \end{itemize}
            \end{naloga}
        


            ~
        %%% Deljenje ulomkov

        \section{Deljenje ulomkov}

        
                Ulomek \textbf{delimo} z neničelnim ulomkom tako, da prvi ulomek množimo z obratno vrednostjo drugega ulomka.
                $$\dfrac{x}{y}:\dfrac{z}{w}=\dfrac{x}{y}\cdot\left(\dfrac{z}{w}\right)^{-1}=\dfrac{x}{y}\cdot\dfrac{w}{z}=\dfrac{xw}{yz}; \quad x\in\mathbb{Z}\land y,z,w\in\mathbb{Z}\setminus\{0\} $$
            

            
                Deljenju ulomkov lahko zapišemo kot \textbf{dvojni ulomek}.
                $$\dfrac{x}{y}:\dfrac{z}{w}=\dfrac{\frac{x}{y}}{\frac{z}{w}}; \quad x\in\mathbb{Z}\land y,z,w\in\mathbb{Z}\setminus\{0\} $$
            

                ~
        


        %%% naloge

        
            \begin{naloga}
                Izračunajte.
                \begin{itemize}
                    \item $2:\frac{4}{5}$ 
                    \item $1\frac{2}{3}:2\frac{5}{6}$ 
                    \item $\frac{7}{12}:14$ 
                    \item $\frac{3}{8}:\frac{9}{32}$ 
                \end{itemize}
            \end{naloga}
        


        
            \begin{naloga}
                Izračunajte.
                \begin{itemize}
                            \item $\dfrac{\frac{3}{4}}{\frac{6}{8}}$ 
                            \item $\dfrac{\frac{1}{2}}{2}$ 
                            \item $\dfrac{3}{\frac{5}{6}}$ 
                            \item $\dfrac{\frac{2}{-5}}{\frac{-1}{5}}$ 
                            \item $\dfrac{\frac{3}{5}}{-2}$ 
                            \item $\dfrac{-\frac{1}{2}}{2^{-1}}$ 

                \end{itemize}
            \end{naloga}
        


        
            \begin{naloga}
                Poenostavite.
                \begin{itemize}
                    \item $\frac{x^2+x-6}{x+2}:(x-2)$ 
                    \item $\frac{x-1}{2x^2-4x}:\frac{x^2}{x-2}$ 
                    \item $x:\frac{x^2+x}{x^3+1}$ 
                \end{itemize}
            \end{naloga}
        

        
            \begin{naloga}
                Poenostavite.
                \begin{itemize}
                    \item $\frac{x-1}{x^2+4}:\frac{1-x^2}{x-2}$ 
                    \item $\frac{x-2}{(x+2)^{-1}}:\left(\frac{1}{x^2-1}\right)^{-1}$ 
                    \item $\frac{3-x}{2-x}:\frac{x-3}{x-2}$ 
                \end{itemize}
            \end{naloga}
        

