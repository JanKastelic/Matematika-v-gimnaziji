\chapter{Potence in koreni}




    \section{Koreni poljubnih stopenj}

        
            % \subsection{Kvadratni koren}

            %     \textbf{Kvadratni koren} $\sqrt{a}$ realnega števila $a\geq 0$ je tisto nenegativno realno število $x$,
            %     katerega kvadrat je enak $a$.
            %     $$\sqrt{a}=x \Leftrightarrow a=x^2; \quad a,x\in\mathbb{R}^+ $$

            %     Število $a$ imenujemo \textbf{korenjenec}, simbol $\sqrt{~}$ pa \textbf{korenski znak}.
            

            % \subsubsection*{Pravila za računanje s kvadratnimi koreni}
            %         \begin{multicols}{2}
            %             \begin{itemize}
            %                 \item $\left(\sqrt{a}\right)^2=a; ~a\geq 0$
            %                 \item $\sqrt{a^2}=\begin{cases}
            %                     a, & a\geq 0 \\
            %                     -a, & a<0
            %                 \end{cases}=\lvert a\rvert$
            %                 \item $\sqrt{a\cdot b}=\sqrt{a}\cdot\sqrt{b}; ~a,b\geq 0$
            %                 \item $\sqrt{\dfrac{a}{b}}=\dfrac{\sqrt{a}}{\sqrt{b}}; ~a\geq 0, b>0$
            %             \end{itemize}
            %         \end{multicols}
                    
            

        

        
            % \subsection{Kubični koren}

            %     \textbf{Kubični koren} $\sqrt[3]{a}$ realnega števila $a$ je tisto realno število $x$,
            %     katerega kub je enak $a$.
            %     $$\sqrt[3]{a}=x \Leftrightarrow a=x^3; \quad a,x\in\mathbb{R}$$

            %     Število $a$ imenujemo \textbf{korenjenec}, simbol $\sqrt{~}$ \textbf{korenski znak}, število $3$ pa \textbf{korenski eksponent}.
            

            % \subsubsection*{Pravila za računanje s kubičnimi koreni}
            %     \begin{multicols}{2}
            %         \begin{itemize}
            %             \item $\left(\sqrt[3]{a}\right)^3=a$
            %             \item $\sqrt[3]{a^3}=a$
            %             \item $\sqrt[3]{a\cdot b}=\sqrt[3]{a}\cdot\sqrt[3]{b}$
            %             \item $\sqrt[3]{\dfrac{a}{b}}=\dfrac{\sqrt[3]{a}}{\sqrt[3]{b}}; ~b\neq 0$
            %         \end{itemize}
            %     \end{multicols}
                
            

        


        
            % \subsection{Koreni poljubnih stopenj}

                Za sodo naravno število $n$ je \textbf{$n$-ti koren} $\sqrt[n]{a}$ realnega števila $a\geq 0$ tisto nenegativno realno število $x$,
                za katerega velja $a=x^n$.
                $$\displaystyle \sqrt[n]{a}=x \Leftrightarrow a=x^n; \quad a,x\in\mathbb{R}^+ $$
                ~
                
                Za liho naravno število $n$ je \textbf{$n$-ti koren} $\sqrt[n]{a}$ realnega števila $a$ tisto realno število $x$,
                za katerega velja $a=x^n$.
                $$\displaystyle \sqrt[n]{a}=x \Leftrightarrow a=x^n; \quad a,x\in\mathbb{R} $$
                ~
                
                Število $a$ imenujemo \textbf{korenjenec}, simbol $\sqrt{~}$ \textbf{korenski znak}, število $n$ pa \textbf{korenski eksponent}.
            

        

         
                        
            \subsection*{Pravila za računanje s koreni poljubnih stopenj}
                \begin{multicols}{2}
                    \begin{itemize}
                        \item $\displaystyle \left(\sqrt[n]{a}\right)^n=a$ 
                        \item $\displaystyle \sqrt[n]{a^n}=\begin{cases}
                                \lvert a\rvert, & n=2k, k\in\mathbb{N} \\
                                a, & n=2k-1, k\in\mathbb{N}
                            \end{cases}$ ~
                        \item $\displaystyle \sqrt[n]{a^w}=\left(\sqrt[n]{a}\right)^w$ 
                        \item $\displaystyle \sqrt[n]{a^w}=\sqrt[nz]{a^{wz}}$ 
                        \item $\displaystyle \sqrt[n]{\sqrt[m]{a}}=\sqrt[nm]{a}$ 
                        \item $\displaystyle \sqrt[n]{a\cdot b}=\sqrt[n]{a}\cdot\sqrt[n]{b}$ 
                        \item $\displaystyle \sqrt[n]{\dfrac{a}{b}}=\dfrac{\sqrt[n]{a}}{\sqrt[n]{b}}; ~b\neq 0$ 
                        \item $\displaystyle \sqrt[n]{a^w}\cdot\sqrt[n]{a^z}=\sqrt[n]{a^{w+z}}$ 
                        \item $\displaystyle \dfrac{\sqrt[n]{a^w}}{\sqrt[n]{a^z}}=\sqrt[n]{a^{w-z}}; ~a\neq 0$
                    \end{itemize}

                \end{multicols}
                
                

                Pri tem za sode korenske stopnje $n$ privzamemo $a,b\in[0,\infty)$; za lihe stopnje $n$ pa $a,b\in\mathbb{R}$.
            

        
                ~\\~\\~

    %%% naloge

        
            \begin{naloga}
                Poenostavite izraz in ga delno korenite.
                \begin{multicols}{2}
                    \begin{itemize}
                        \item $\displaystyle \sqrt[3]{xy^2\sqrt{x^5y}}$ 
                        \item $\displaystyle \sqrt{a\sqrt{a^2\sqrt{a^3}}}$ 
                        \item $\displaystyle \sqrt[4]{a^3b^2\sqrt{ab^5}}$ 
                        \item $\displaystyle \sqrt[4]{ab^2\sqrt[3]{ab}}$ 
                        \item $\displaystyle \sqrt[3]{a\sqrt[4]{a\sqrt[5]{a}}}$ 
                        \item $\displaystyle \sqrt[5]{x^4y\sqrt[4]{x^5y^3}}$ 
                        \item $\displaystyle \sqrt[6]{a^2b^3\sqrt{a^8\sqrt[3]{b}}}$    
                        \item $\displaystyle \sqrt[3]{x\sqrt{y^3\sqrt[4]{x^3\sqrt[5]{y^6y^{-1}}}}}$                  
                    \end{itemize}
                \end{multicols}
            \end{naloga}
        


        
            \begin{naloga}
                Izračunajte.
                \begin{multicols}{2}
                    \begin{itemize}
                        \item $\displaystyle \sqrt[5]{\dfrac{1}{32}}$ 
                        \item $\displaystyle \sqrt[4]{\dfrac{16}{81}}$ 
                        \item $\displaystyle \sqrt[3]{-8}$ 
                        \item $\displaystyle \sqrt[4]{-625}$ 
                        \item $\displaystyle \sqrt[3]{0.125}$ 
                        \item $\displaystyle \sqrt[4]{0.0016}$ 
                    \end{itemize}
                \end{multicols}
            \end{naloga}
        
        
            \begin{naloga}
                Poenostavite.
                \begin{multicols}{2}
                    \begin{itemize}
                        \item $\displaystyle \sqrt[18]{x^{15}}$ 
                        \item $\displaystyle \sqrt[9]{a^6}$ 
                        \item $\displaystyle \sqrt[30]{y^{18}}$ 
                        \item $\displaystyle \sqrt[20]{b^{30}}$ 
                    \end{itemize}
                \end{multicols}
            \end{naloga}
        
        
            \begin{naloga}
                Racionalizirajte ulomke.
                \begin{multicols}{2}
                    \begin{itemize}
                        \item $\displaystyle \dfrac{1}{3-\sqrt{x}}$ 
                        \item $\displaystyle \dfrac{1}{2-4\sqrt[3]{a}}$ 
                        \item $\displaystyle \dfrac{2}{a-\sqrt[3]{b}}$ 
                        \item $\displaystyle \dfrac{x-1}{\sqrt[3]{x}-1}$ 
                        \item $\displaystyle \dfrac{8x}{2\sqrt[3]{x}+1}$ 
                        \item $\displaystyle \dfrac{1}{2-\sqrt[4]{3}}$ 
                        \item $\displaystyle \dfrac{1}{\sqrt[4]{2}-1}$ 
                        \item $\displaystyle \dfrac{\sqrt[4]{y}}{2-\sqrt[4]{y}}$ 
                        \item $\displaystyle \dfrac{3}{1+\sqrt[5]{2}}$ 
                    \end{itemize}
                \end{multicols}
            \end{naloga}
        
        
            \begin{naloga}
                Poenostavite in delno korenite izraz.
                \begin{multicols}{2}
                    \begin{itemize}
                        \item $\displaystyle \dfrac{\sqrt[4]{2}}{\sqrt{2\sqrt{8}}}$ 
                        \item $\displaystyle \dfrac{\sqrt[3]{9}}{\sqrt[5]{3}\sqrt{27}}$ 
                        \item $\displaystyle \dfrac{\sqrt{\sqrt{\sqrt{1}}}}{\sqrt[17]{1}}$ 
                        \item $\displaystyle \dfrac{\sqrt{\sqrt{a}}}{\sqrt[3]{a^2}}$ 
                        \item $\displaystyle \dfrac{\sqrt{a\sqrt[3]{a^{-1}}\cdot\sqrt[3]{a^2\sqrt[5]{a}}}}{\sqrt[5]{a\sqrt{a^{-5}}}}$ 
                        \item $\displaystyle \dfrac{\sqrt{x^3\sqrt[4]{x^3\sqrt{x}}}}{\sqrt[4]{x^{-3}\sqrt[4]{x}}}$ 
                        \item $\displaystyle \dfrac{\sqrt[7]{b^{13}\sqrt{b^{-2}}}}{\sqrt{\sqrt{b^{-1}}}}$ 
                        \item $\displaystyle \dfrac{\sqrt[3]{x^2\sqrt[4]{x^{-1}}}\cdot\sqrt[4]{x^3\sqrt{x}}}{\sqrt[4]{x\sqrt{x\sqrt[3]{x^{-1}}}}}$ 
                        \item $\displaystyle \dfrac{\sqrt{8ab^{-1}}}{\sqrt{0.5}\sqrt[3]{8ab^2}}$ 
                    \end{itemize}
                \end{multicols}
            \end{naloga}
        
        
            \begin{naloga}
                Izračunajte natančno vrednost korena.
                \begin{multicols}{2}
                    \begin{itemize}
                        \item $\displaystyle \sqrt{31-12\sqrt{3}}$ 
                        \item $\displaystyle \sqrt{18+8\sqrt{2}}$ 
                        \item $\displaystyle \sqrt{9-4\sqrt{5}}$ 
                        \item $\displaystyle \sqrt{17+2\sqrt{2}}$ 
                    \end{itemize}
                \end{multicols}
            \end{naloga}
        
        
            \begin{naloga}
                Poenostavite izraz in ga delno korenite.
                \begin{multicols}{2}
                    \begin{itemize}
                        \item $\displaystyle \dfrac{\sqrt[5]{xy^3\sqrt[4]{x^2y^3}}}{\sqrt[10]{\sqrt{x}}}$ 
                        \item $\displaystyle \dfrac{\sqrt[4]{ab^3\sqrt[3]{a^2b^3}}}{\sqrt{\sqrt[6]{a}}}$ 
                        \item $\displaystyle \left(\dfrac{1-z}{1-\sqrt[3]{z}}-\sqrt[3]{z}\right)\left(1-\sqrt[6]{z^4}\right)$ 
                        \item $\displaystyle \sqrt[3]{\sqrt{\sqrt{4096}}}+\sqrt{\sqrt{\sqrt{16}}}-\sqrt[5]{32}$ 
                        \item $\displaystyle \dfrac{\sqrt[6]{ab^3\sqrt{a^3b}}}{\sqrt[4]{b^{-3}\sqrt[3]{a}}}$ 
                    \end{itemize}
                \end{multicols}
            \end{naloga}
        



            \newpage
    \section{Potence z racionalnimi eksponenti}

        

            Potenca z racionalnim eksponentom je definirana kot: 
                $$\displaystyle x^\frac{m}{n}=\sqrt[n]{x^m},$$
                kjer je $m\in\mathbb{Z}$, $n\in\mathbb{N}$ in $a\in[0,\infty)$.
            

            \subsection*{Pravila za računanje s potencami s celimi eksponenti}
                \begin{multicols}{2}
                    \begin{itemize}
                        \item $\displaystyle x^p\cdot x^q=x^{p+q}$
                        \item $\displaystyle x^p\cdot y^p=(xy)^p$
                        \item $\displaystyle \left(x^p\right)^q=x^{pq}$
                        \item $\displaystyle x^p:x^q=\dfrac{x^p}{x^q}=x^{p-q}; \quad x\neq 0$
                        \item $\displaystyle x^p:y^p=\dfrac{x^p}{y^p}=\left(\dfrac{x}{y}\right)^p; \quad y\neq 0$
                    \end{itemize}
                \end{multicols}
                
                V pravilih upoštevamo primerni realni osnovi $x,y\in\mathbb{R}$ in racionalne eksponente $p,q\in\mathbb{Q}$.

                        ~\\~\\

    %%% naloge

            \begin{naloga}
                Izračunajte.
                \begin{itemize}
                    \item $\displaystyle 8^\frac{1}{3}-16^\frac{2}{4}$ 
                    \item $\displaystyle 27^\frac{2}{3}-125^\frac{1}{3}$ 
                    \item $\displaystyle \left(-8\right)^{-\frac{1}{3}}$ 
                    \item $\displaystyle 1000^\frac{2}{3}-343^\frac{2}{3}$ 
                \end{itemize}
            \end{naloga}

            
            \begin{naloga}
                Izračunajte.
                \begin{multicols}{2}
                    \begin{itemize}
                        \item $\displaystyle \sqrt{625^\frac{3}{4}-\left(\dfrac{1}{2}\right)^{-2}}+4^\frac{1}{3}\cdot 16^\frac{1}{3} $ 
                        \item $\displaystyle 4\cdot 0.16^{-\frac{1}{2}}-\sqrt[3]{5\cdot 8^\frac{1}{3}+2\cdot 81^\frac{3}{4}} $ 
                        \item $\displaystyle \left(2\cdot 9^\frac{3}{2}+5\cdot 16^\frac{1}{4}\right)^\frac{1}{3} $ 
                        \item $\displaystyle \left(\left(\dfrac{4}{9}\right)^{-\frac{1}{2}}\cdot 32^\frac{1}{5}+169^\frac{1}{2}\right)^\frac{1}{2} $ 
                        \item $\displaystyle 0.25^{-\frac{1}{2}}\cdot 0.001^{-\frac{1}{3}} -\sqrt[3]{10^2+0.2^{-2}} $ 
                        \item $\displaystyle \left(3\dfrac{3}{8}\right)^\frac{2}{3}\cdot\left(\dfrac{1}{4}\right)^{-\frac{1}{2}}\cdot\left(3-\sqrt{5}\right)\sqrt{7+3\sqrt{5}} $ 
                    \end{itemize}
                \end{multicols}
            \end{naloga}
        
        
            \begin{naloga}
                Izračunajte.
                \begin{multicols}{2}
                    \begin{itemize}
                        \item $\displaystyle 2.25^{-0.5}\cdot\sqrt{4^{1.5}+1} $ 
                        \item $\displaystyle 6.25^{-0.5}\cdot 2.25^{1.5}+\sqrt{16^{0.75}+1} $ 
                        \item $\displaystyle \left(3\dfrac{1}{16}\right)^{-0.5}\sqrt{0.125^{-\frac{2}{3}}+3}^4+0.002^{-\frac{2}{3}} $ 
                        \item $\displaystyle \sqrt{10}\left(5^{-0.5}-2\right)^{-1}-\sqrt{90} $ 
                   \end{itemize}
                \end{multicols}
                    \begin{itemize}
                        \item $\displaystyle \sqrt{27^\frac{2}{3}+0.25^{-2}}+\left(2-\sqrt{5}\right)\sqrt{9+4\sqrt{5}}-\dfrac{1+\sqrt{12}}{2+\sqrt{3}} $ 
                    \end{itemize}

            \end{naloga}

        
            \begin{naloga}
                Izraz zapišite s potencami in ga poenostavite.
                \begin{multicols}{2}
                    \begin{itemize}
                        \item $\displaystyle \left(\dfrac{1-z}{1-\sqrt[3]{z}}-\sqrt[3]{z}\right)\left(1-\sqrt[6]{z^4}\right)$ 
                        \item $\displaystyle \dfrac{\sqrt[6]{ab^3\sqrt{a^3b}}}{\sqrt[4]{b^{-3}\sqrt[3]{a}}}$ 
                        \item $\displaystyle \left(y^\frac{2}{3}x^{-0.25}\right)^6 :\left(\sqrt{x^{-4}y^2}\cdot\sqrt{y\sqrt[3]{xy^{-3}}}\right)^3 $ 
                        \item $\displaystyle \dfrac{\sqrt[3]{x^{-4}\sqrt{x^2y^{-3}}}}{\sqrt[4]{x^{-3}y^2}}\cdot\left(x^{0.3}y^{0.2}\right)^5 $ 
                        \item $\displaystyle \dfrac{\sqrt[5]{x^{-2}\sqrt[3]{x^{-3}y^4}}}{y^{-\frac{1}{3}}x^\frac{1}{2}}\left(\sqrt[6]{\sqrt{y^{-3}}}\right)^4 $ 
                        \item $\displaystyle \dfrac{\sqrt[4]{x^{-2}y}}{\sqrt[6]{x^3\sqrt{y^{-7}}}}\sqrt[4]{x^2y^{-5}}^2 $ 
                    \end{itemize}
                \end{multicols}
            \end{naloga}
        





            \newpage

    \section{Iracionalne enačbe}

    
                \textbf{Iracionalna enačba} je enačba, v kateri neznanka nastopa po korenom poljubne stopnje.
            

            \subsubsection*{Reševanje iracionalne enačbe}
                Iracionalno enačbo rešujemo tako, da jo s pomočjo potenciranja prevedemo v enačbo, ki nima neznanke pod korenom.

                Tako dobimo enačbo, ki ni nujno ekvivalentna prvotni enačba, saj lahko s potenciranjem pridobimo kakšno rešitev, ki ne ustreza prvotni enačbo.

                Na koncu reševanja moramo vedno narediti \textbf{preizkus}, s katerim izločimo morebitne neustrezne rešitve.
            
        
~\\~\\

    %%% naloge

        
            \begin{naloga}
                Rešite enačbo.
                \begin{itemize}
                    \item $\displaystyle \sqrt{x-1}-5=0$ 
                    \item $\displaystyle \sqrt{x+5}=2$ 
                    \item $\displaystyle \sqrt{3-x}-5=0$ 
                    \item $\displaystyle 1+\sqrt{x-5}=0$ 
                \end{itemize}
            \end{naloga}

        
            \begin{naloga}
                Rešite enačbo.
                \begin{multicols}{2}
                    \begin{itemize}
                        \item $\displaystyle \sqrt{2x-1}+2x=x$ 
                        \item $\displaystyle 2+\sqrt[3]{x-1}=0$ 
                        \item $\displaystyle \sqrt{x^2+2}-\sqrt{3x}=0$ 
                        \item $\displaystyle x-\sqrt{5x-11}=1$ 
                        \item $\displaystyle 2x+3=\sqrt{3x^2+5x-1}$ 
                        \item $\displaystyle \sqrt{-8x-4}=-2x$ 
                        \item $\displaystyle \sqrt{x^2-1}-2=0$ 
                        \item $\displaystyle \sqrt{x+3}=-9$ 
                    \end{itemize}
                \end{multicols}
            \end{naloga}

        
            \begin{naloga}
                Rešite enačbo.
                \begin{multicols}{2}
                    \begin{itemize}
                        \item $\displaystyle \sqrt{x}+\sqrt{x+1}=3$ 
                        \item $\displaystyle \sqrt{x-2}-2=\sqrt{x+2}$ 
                        \item $\displaystyle \sqrt{x+1}=\sqrt{2}-\sqrt{x-1}$ 
                        \item $\displaystyle \sqrt{x-6}+\sqrt{x+2}=2$ 
                        \item $\displaystyle \sqrt{x+5}-3=-\sqrt{x}$ 
                        \item $\displaystyle \sqrt{3x+1}-1=\sqrt{x+4}$ 
                        \item $\displaystyle \sqrt[3]{x+2-\sqrt{10+x}}=-2$ 
                        \item $\displaystyle \sqrt{5+x}-1=\sqrt{3x+4}$ 
                    \end{itemize}
                \end{multicols}
            \end{naloga}

        
            \begin{naloga}
                Rešite enačbo.
                \begin{multicols}{2}
                    \begin{itemize}
                        \item $\displaystyle \sqrt[3]{x^3+7x^2+x+26}-3=x-1$ 
                        \item $\displaystyle \sqrt{x-2}-\sqrt{2x-3}=2$ 
                        \item $\displaystyle \sqrt{x^2+3x}+x=2$ 
                        \item $\displaystyle \sqrt{x+7-\sqrt{2x-1}}=3$ 
                        \item $\displaystyle \sqrt[3]{5-x+\sqrt{2x+14}}-2=0$ 
                        \item $\displaystyle \sqrt{x-6}-\sqrt{x+2}-2=0$ 
                        \item $\displaystyle \sqrt{x+3+\sqrt{x+2}}=\sqrt{3}$ 
                        \item $\displaystyle \sqrt[5]{x^2+3x+34}=2$ 
                    \end{itemize}
                \end{multicols}
            \end{naloga}
        
