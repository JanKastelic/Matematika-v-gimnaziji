\chapter{Kvadratna funkcija}

\section{Kvadratna funkcija v splošni in temenski obliki}


%%%%%% naloge
\newpage
    

        \begin{naloga}
            Narišite graf funkcije, zapišite koordinati temena in začetno vrednost funkcije.
            
            \begin{multicols}{2}
                
                \begin{itemize}
                    \item $ f(x)=2x^2 $ 
                    \item $ g(x)=-\frac{1}{4}x^2 $ 
                    \item $ h(x)=x^2+2 $ 
                    \item $ i(x)=x^2-1 $ 
                    \item $ j(x)=\frac{1}{2}x^2-2 $ 
                    \item $ k(x)=-2x^2+1 $
                    \item $ l(x)=\left(x-2\right)^2 $ 
                    \item $ m(x)=\left(x+1\right)^2 $ 
                    \item $ n(x)=-2\left(x+1\right)^2 $ 
                    \item $ o(x)=\frac{1}{2}\left(x+4\right)^2 $
                \end{itemize}
            \end{multicols}

        \end{naloga}
    

        
    

        \begin{naloga}
            Graf $y=x^2$ transformiramo po navodilu.
            Zapišite predpis funkcije v splošni obliki, katere graf je transformiran po navodilu.
            Določite koordinati temena. 
            Zapišite zalogo vrednosti določene funkcije.
            
                \begin{itemize}
                    \item Togi premik za vektor $\vec{v}=(-2,3)$.
                    \item Togi premik za vektor $\vec{v}=(-1,-0.5)$.
                    \item Togi premik za vektor $\vec{v}=(0,1)$ in razteg za faktor $2$ v smeri ordinatne osi.
                    \item Togi premik za vektor $\vec{v}=(1,3)$ in zrcaljenje čez abscisno os.
                    \item Togi premik za vektor $\vec{v}=(2,0)$ in zrcaljenje čez ordinatno os.
                    \item Togi premik za vektor $\vec{v}=(-1,-2)$, skrčitev za faktor $2$ v smeri ordinatne osi in zrcaljenje čez abscisno os.
                \end{itemize}

        \end{naloga}
    


        \begin{multicols}{2}
            

        \begin{naloga}
            Izračunajte teme kvadratne funkcije in njen predpis zapišite v temenski obliki.
            

                \begin{itemize}
                    \item $ f(x)=2x^2-12x+19 $ 
                    \item $ g(x)=-x^2-2x+2 $ 
                    \item $ h(x)=-3x^2-12x-13 $ 
                    \item $ i(x)=\frac{1}{2}x^2-4x+7 $ 
                    \item $ j(x)=\frac{1}{3}x^2+4x+10 $ 
                    \item $ k(x)=-\frac{1}{2}x^2+3x-4 $
                \end{itemize}

        \end{naloga}
    


    

        \begin{naloga}
            Izračunajte teme parabole $y=3x^2+6x+5$. 
            Parabolo premaknemo za vektor $\vec{v}=(3,-1)$. 
            Zapišite splošno enačbo premaknjene parabole in določite njeno teme.
            
        \end{naloga}

        ~\\~\\~\\~

        \end{multicols}


        \begin{multicols}{2}
            
        \begin{naloga}
            Za kateri vektor v smeri abscisne osi moramo premakniti dano parabolo, da bo dobljena krivulja graf sode funkcije?
            Zapišite njeno enačbo.

                \begin{itemize}
                    \item $ y=2x^2-12x+17 $ 
                    \item $ y=-x^2-6x-5 $ 
                    \item $ y=\frac{1}{2}x^2-2x+3 $ 
                    \item $ y=-\frac{3}{4}x^2-12x-13 $
                \end{itemize}

        \end{naloga}
    


    
            \begin{naloga}
                Zapišite simetrijsko os in teme parabole.
                
                    \begin{itemize}
                        \item $ y=5x^2-40x+90 $ 
                        \item $ y=-2x^2-12x+1 $ 
                        \item $ y=-\frac{5}{6}x^2-3\frac{1}{3}x $ 
                        \item $ y=\frac{2}{3}x^2-\frac{3}{4}x+\frac{1}{7} $ 
                    \end{itemize}

            \end{naloga}

            ~\\~

        \end{multicols}


        \begin{naloga}
                Zapišite splošno obliko enačbe parabole, ki:
                
                    \begin{itemize}
                        \item ima teme v točki $T(3,-2,)$ in poteka skozi točko $A(1,6)$.
                        \item ima teme v točki $T(1,5)$ in seka ordinatno os pri $4$.
                        \item ima teme v točki $T(-2,3)$ in na njej leži točka $B(-1,0)$.
                        \item ima teme v točki $T(0.5,-0.75)$ in gre skozi koordinatno izhodišče.
                    \end{itemize}

            \end{naloga}

  

            \begin{multicols}{2}
                
    

            \begin{naloga}
                Zapišite enačbo parabole, ki gre skozi točke $A$, $B$ in $C$.
                Ali točka $D$ leži na tej paraboli?
                
                    \begin{itemize}
                        \item $A(1,-3)$, $B(0,-7)$, $C(-1,-13)$ in $D(2,-1)$ 
                        \item $A(1,0)$, $B(2,3)$, $C(-1,6)$ in $D(0,1)$ 
                        \item $A(1,3)$, $B(0.5,5)$, $C(0,5)$ in $D(3,10)$

                    \end{itemize}

            \end{naloga}


            \begin{naloga}
                Dana je družina parabol $y=\left(k+1\right)x^2+2x+1$. 
                Za katero vrednost parametra $k$ bo:
                
                    \begin{itemize}
                        \item abscisa temena $x=\frac{1}{3}$?
                        \item teme ležalo na abscisni osi?
                        \item premica $x=-2$ simetrijska os parabole?
                        \item teme ležalo na premici $y=x+1$?
                        \item parabola sekala ordinatno os pri $1$?
                    \end{itemize}

            \end{naloga}

            \end{multicols}


        \begin{multicols}{2}
            

            \begin{naloga}
                Dana je družina parabol $y=mx^2-3x+(m+1)$. 
                Za katero vrednost parametra $m$ bo:
                
                    \begin{itemize}
                        \item abscisa temena $x=6$?
                        \item parabola sekala ordinatno os pri $3$?
                        \item premica $x=3$ simetrijska os parabole?
                        \item teme ležalo na premici $y=1$?

                    \end{itemize}

            \end{naloga}

            
            \begin{naloga}
                Dana je kvadratna funkcija $f(x)=\left(x-2\right)^2+1$.
                Zapišite njen predpis v splošni obliki.
                Zapišite predpis funkcije, ki jo dobimo pri: 
                
                    \begin{itemize}
                        \item zrcaljenju čez abscisno os.
                        \item zrcaljenju čez ordinatno os.
                        \item zrcaljenju čez koordinatno izhodišče.
                    \end{itemize}

            \end{naloga}

    
        \end{multicols}


        \begin{multicols}{2}
    
            \begin{naloga}
                Dana je funkcija $f(x)=2\left(x-1\right)^2-2$. Narišite grafe:
                
                    \begin{itemize}
                        \item $ y=f(x) $ 
                        \item $ y=\lvert f(x) \rvert $ 
                        \item $ y=f(|x|) $ 
                        \item $ y=-f(x) $ 
                        \item $ y=f(-x) $ 
                    \end{itemize}

            \end{naloga}

            \begin{naloga}
                Dana je funkcija $f(x)=\frac{1}{2}\left(x-3\right)^2-\frac{3}{2}$. Narišite grafe:
                
                    \begin{itemize}
                        \item $ y=f(x) $ 
                        \item $ y=\lvert f(x) \rvert $ 
                        \item $ y=f(|x|) $ 
                        \item $ y=-f(x) $ 
                        \item $ y=f(-x) $ 
                    \end{itemize}

            \end{naloga}

        \end{multicols}

        
        
    %%%%%%%%%%%%%%%%%%%%%%%%%%
    \newpage
    \section{Ničle kvadratne funkcije in rešitve kvadratne enačbe}


%%%%% naloge
\newpage

    

        \begin{naloga}
            Rešite kvadratno enačbo.
            
            \begin{multicols}{2}
                
                \begin{itemize}
                    \item $ x^2-14x+24=0 $ 
                    \item $ -x^2+10x+39=0 $ 
                    \item $ 2x^2+24x+70=0 $ 
                    \item $ \frac{1}{2}x^2+x-60=0 $ 
                    \item $ x^2-10x+25=0 $ 
                    \item $ x^2-9=0 $
   
                    \item $ 3x^2-2x=2x^2-35-14x $ 
                    \item $ x^2-10x=36-x^2+4x $ 
                    \item $ x^2-10x=5x-2x^2 $ 
                    \item $ 70+x^2-x=2x^2-2x-2 $ 
                    \item $ 2x^2-10x=5x+x^2 $ 
                    \item $ 3x^2-4x=25-4x+2x^2 $
                \end{itemize}
            \end{multicols}
        \end{naloga}
    


            \begin{multicols}{2}
            \begin{naloga}
                Rešite kvadratno enačbo.
                
                    \begin{itemize}
                        \item $ 4x^2+5x-6=0 $ 
                        \item $ 12x^2+11x+2=0 $ 
                        \item $ 3x^2+1x-8=0 $ 
                        \item $ x^2-6x+2=0 $
                    \end{itemize}

            \end{naloga}

            \begin{naloga}
                Rešite enačbo.
                
                    \begin{itemize}
                        \item $ 2x^3-5x^2-3x=0 $ 
                        \item $ \left(2x-1\right)^2-5\left(2x-1\right)+6=0 $ 
                        \item $ \dfrac{1}{x}-\dfrac{1}{x+2}=\dfrac{2}{15} $ 
                    \end{itemize}

            \end{naloga}


            \end{multicols}


    

        \begin{naloga}
                Izračunajte ničli kvadratne funkcije in jo zapišite v faktorizirani obliki.
                
                    \begin{itemize}
                        \item $ f(x)=2x^2-x-1 $ 
                        \item $ g(x)=4x^2+2x+2 $ 
                        \item $ h(x)=-3x^2-4x+4 $ 
                        \item $ i(x)=8x^2-2x+3 $ 
                    \end{itemize}

            \end{naloga}


    


    

            \begin{naloga}
                V splošni obliki zapišite predpis kvadratne funkcije, ki:
                                
                    \begin{itemize}
                        \item ima ničli $x_1=-2$ in $x_2=3$ ter začetno vrednost $f(0)=-12$.
                        \item ima ničli $x_1=1$ in $x_2=3$, največja vrednost, ki jo zavzame je $5$.
                        \item ima ničli $x_1=-7$ in $x_2=1$, $x=1$ pa preslika v $y=4$.
                        \item ima dvojno ničlo $x_{1,2}=-3$ in začetno vrednost $i(0)=3$.

                    \end{itemize}

            \end{naloga}


            \begin{naloga}
                Zapišite enačbo parabole, ki:
                
                    \begin{itemize}
                        \item seka abscisno od v $x_1=-1$ in $x_2=4$, ordinatno os pa pri $8$.
                        \item seka abscisno od v $x_1=-1$ in $x_2=5$, teme pa leži na premici $y=9$.
                        \item seka abscisno od v $x_1=4$ in $x_2=7$, gre skozi točko $A(2,20)$.
                        \item seka abscisno od v $x_1=-2$ in $x_2=-6$, zaloga vrednosti pa je $(-\infty,2]$.
                    \end{itemize}

            \end{naloga}

    


    

            \begin{naloga}
                V faktorizirani obliki zapišite kvadratno funkcijo, ki ima:
                                
                    \begin{itemize}
                        \item teme v točki $T(7,-3)$ in ničlo $x_1=6$.
                        \item teme v točki $T(1,9)$ in ničlo $x_1=-2$.
                        \item teme v točki $T(3,-4)$ in ničlo $x_1=-1$.

                    \end{itemize}

            \end{naloga}


            \begin{naloga}
                V temenski obliki zapišite kvadratno funkcijo, ki ima:
                
                    \begin{itemize}
                        \item ničli $x_1=-5$ in $x_2=3$, teme pa v točki $T(x,32)$.
                        \item ničli $x_1=-\frac{1}{2}$ in $x_2=\frac{5}{2}$, teme pa v točki $T(x,-9)$.
                        \item ničli $x_1=-4$ in $x_2=2$, teme pa v točki $T(x,18)$.
                    \end{itemize}

            \end{naloga}

    


    

            \begin{naloga}
                Dana je družina kvadratnih funkcij.
                Za katero vrednost parametra $m$ ima funkcija eno dvojno ničlo?
                Izračunajte tudi ničlo.
                                
                    \begin{itemize}
                        \item $f(x)=4x^2+(m+1)x+1$ 
                        \item $g(x)=-2x^2+mx-x-18$
                        \item $h(x)=-x^2+mx-x+m-1$

                    \end{itemize}

            \end{naloga}


            \begin{naloga}
                Dana je družina parabol.
                Za katero vrednost parametra $n$ se parabola dotika abscisne osi. 
                Izračunajte dotikališče.
                
                    \begin{itemize}
                        \item $y=2x^2+(n-3)x+2$
                        \item $y=-4x^2+nx-2x-1$
                    \end{itemize}

            \end{naloga}

    



%%%%%%%%%%%%%%%%%%%%%%
\newpage
    \section{Kvadratna neenačba}


%%%%% naloge
\newpage





%%%%%%%%%%%%%%%%%%%%%%%%%%
\newpage
    \section{Presečišče dveh krivulj}




%%%%% naloge
\newpage





%%%%%%%%%%%%%%%%%%%%%%%%%%
\newpage
    \section{Uporaba kvadratne funkcije}

%%%%% naloge
\newpage





%%%%%%%%%%%%%%%%%%%%%%%%%%%%%%%%%%%%%%%%%%%%%

\newpage
\section*{Kvadratna funkcija -- vaje}
\normalsize
\begin{naloga}
    Dana je funkcija $f(x)=2x^2-3$.
    \begin{itemize}
        \item Narišite njen graf.
        \item Če graf premaknete za $2$ v levo, dobite graf funkcije $g$. Zapišite predpis funkcije $g$ v splošni obliki.
        \item Zapišite koordinati temena funkcije $f$ in funkcije $g$.
        \item Narišite še grafe funkcij $|g(x)|$, $g(|x|)$ in $-g(x)$.
    \end{itemize}
\end{naloga}

\begin{naloga}
    Izračunajte teme parabole $y=\frac{3}{4}x^2-2x+\frac{7}{2}$.
\end{naloga}

\begin{naloga}
    Zapišite splošno enačbo parabole, ki ima teme v točki $T(4,5)$ in ordinatno os seka pri $-3$.
\end{naloga}

\begin{naloga}
    Dana je družina parabol $y=mx^2-3x+(m+1)$. Za katero vrednost parametra $m$ bo:
    \begin{itemize}
        \item abscisa temena $x=6$?
        \item parabola sekala ordinatno os pri $3$?
        \item premica $x=3$ simetrijska os parabole?
        \item teme ležalo na premici $y=1$?
    \end{itemize}
\end{naloga}

\begin{naloga}
    Rešite enačbo.
    \begin{itemize}
        \item $t^2-2t=0$
        \item $m^2-8m=16$
        \item $2x+3=2x^2$
        \item $8a^2+14a-4=0$
        \item $z^2+z=-1$
        \item $m^2+81=-18m$
    \end{itemize}
\end{naloga}

\begin{naloga}
    Izračunajte ničli funkcije in predpis funkcije zapišite v faktorizirani obliki.
    \begin{itemize}
        \item $f(x)=25x^2-10x+1$
        \item $g(x)=x^2-6x+2$
        \item $h(x)=-5x^2+3x$
    \end{itemize}
\end{naloga}

\begin{naloga}
    V splošni obliki zapišite predpis kvadratne funkcije, ki ima ničlo in teme v točki $T(4,0)$ in katere graf poteka skozi točko $(5,-3)$.
\end{naloga}

\begin{naloga}
    V splošni obliki zapišite predpis kvadratne funkcije, ki ima teme v točki $T(\frac{1}{2},-6)$ in ima ničlo $x=-3$.
\end{naloga}

\begin{naloga}
    Rešite neenačbo $-x^2+5x+13>x^2-7-x$.
\end{naloga}

\begin{naloga}
    Dana je družina kvadratnih funkcij $f(x)=mx^2+4x+m-3$. Za katero vrednost parametra $m$ ima funkcija dve različni realni ničli?
\end{naloga}

\begin{naloga}
    Dani sta funkciji $f$ in $g$. Na katerem intervalu leži graf funkcije $g$ nad grafom funkcije $f$?
    \begin{itemize}
        \item $f(x)=x^2+2x-1$ in $g(x)=-2x^2-4x-1$
        \item $f(x)=2x^2-12x+20$ in $g(x)=x^2-6x-10$
        \item $f(x)=\frac{1}{2}x^2-x+4$ in $g(x)=-x^2+6x$
        \item $f(x)=-x^2+6x-9$ in $g(x)=x^2+2$
    \end{itemize}
\end{naloga}

\begin{naloga}
    Rešite sistem neenačb.
    \begin{itemize}
        \item $(-x^2+22>9x)\land(2x^2-x+5<10x)$
        \item $(\frac{1}{2}x^2<-2x)\land(-x^2+4\geq 0)$
    \end{itemize}
\end{naloga}

\begin{naloga}
    Izračunajte koordinate skupnih točk parabol $y=-x^2+3x-4$ in $y=x^2+8x-2$, če le-te obstajajo.
\end{naloga}