\section{Funkcija}

\begin{frame}
    \sectionpage
\end{frame}

\begin{frame}
    \tableofcontents[currentsection, hideothersubsections]
\end{frame}
        
        \begin{frame}
            \frametitle{Funkcija}
        \end{frame}


        %%%% naloge

        \begin{frame}

            \only<2->{\begin{exampleblock}{Naloga}
                Funkcijo $f: A\to B$ predstavite s tabelo. Izračunajte, kam posamezna funkcija preslika $x=1$.
                \begin{itemize}
                    \item $A=\left\{-2, -1, 0, 1, 2, 3\right\}$, $B=\left\{0, 1, 2, 3, 4, 5\right\}$, $f(x)=|x|+1$ \\~
                    \item $A=\left\{1, 2, 3, 4, 5\right\}$, $B=\mathbb{N}$, $f(x)=2x+1$ \\~
                    \item $A=B=\left\{\frac{1}{3}, \frac{1}{2}, 1, 2, 3\right\}$, $f(x)=\dfrac{1}{x}$ \\~
                \end{itemize} 
            \end{exampleblock}}
           
            \only<3->{\begin{exampleblock}{Naloga}
                Tabelirajte funkcijo $g(x)=2x+|x|$ od $-3$ do $3$ s korakom $1$. \\~
            \end{exampleblock}}


        \end{frame}


        \begin{frame}
            \only<2->{\begin{exampleblock}{Naloga}
                Zapišite definicijska območja funkcij.
                \vskip-0.5em
                \begin{columns}
                    \column{0.5\textwidth}
                    \begin{itemize}
                        \item $f(x)=\dfrac{-7}{x+1}$ \\~
                        \item $g(x)=\dfrac{1}{(x+2)(x+6)}$ \\~
                        \item $h(x)=\dfrac{3x^2+1}{5}$ \\~
                        \item $i(x)=\sqrt{x-2}$ \\~
                    \end{itemize}

                    \column{0.47\textwidth}
                    \begin{itemize}
                        \item $j(x)=x^3-\frac{2}{3}$ \\~
                        \item $k(x)=\sqrt{x^2+7}$ \\~
                        \item $l(x)=\dfrac{3}{x}$ \\~
                        \item $m(x)=\dfrac{x^2+1}{x^2-5x-6}$ \\~
                    \end{itemize}
                \end{columns}

            \end{exampleblock}}
        \end{frame}

        \begin{frame}
            \only<2->{\begin{exampleblock}{Naloga}
                Izračunajte ničle funkcij.
                \vskip-0.5em
                \begin{columns}
                    \column{0.5\textwidth}
                    \begin{itemize}
                        \item $f(x)=\frac{4}{5}-6x$ \\~
                        \item $g(x)=x^2-7x+12$ \\~
                        \item $h(x)=\dfrac{3x+6}{5}$ \\~
                        \item $i(x)=x^2-9$ \\~
                    \end{itemize}

                    \column{0.47\textwidth}
                    \begin{itemize}
                        \item $j(x)=x^2+1$ \\~
                        \item $k(x)=x^2-3x^2-4x+12$ \\~
                        \item $l(x)=\sqrt{x+7}$ \\~
                        \item $m(x)=\dfrac{3}{x}$ \\~
                    \end{itemize}
                \end{columns}

            \end{exampleblock}}
        \end{frame}


        \begin{frame}
            \only<2->{\begin{exampleblock}{Naloga}
                Izračunajte začetne vrednosti funkcij.
                \vskip-0.5em
                \begin{columns}
                    \column{0.5\textwidth}
                    \begin{itemize}
                        \item $f(x)=\frac{4}{5}-6x$ \\~
                        \item $g(x)=x^2-7x+12$ \\~
                        \item $h(x)=\dfrac{3x+6}{5}$ \\~
                        \item $i(x)=x^2-9$ \\~
                    \end{itemize}

                    \column{0.47\textwidth}
                    \begin{itemize}
                        \item $j(x)=x^2-3x^2-4x+12$ \\~
                        \item $k(x)=\sqrt{x+7}$ \\~
                        \item $l(x)=\dfrac{3}{x}$ \\~
                        \item $m(x)=\dfrac{x^3-2x^2-4}{x^4+2x^3+3}$ \\~
                    \end{itemize}
                \end{columns}

            \end{exampleblock}}
        \end{frame}





    \subsection{Linerana funkcija}

        \begin{frame}
            \frametitle{Predpis linearne funkcije}
        \end{frame}


    \subsection{Predpis linearne funkcije}

        \begin{frame}
            \frametitle{Predpis linearne funkcije}
        \end{frame}


    \subsection{Graf linearne funkcije}

        \begin{frame}
            \frametitle{Graf linearne funkcije}
        \end{frame}

        