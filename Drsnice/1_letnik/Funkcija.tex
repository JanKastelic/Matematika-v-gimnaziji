\section{Funkcija}

\begin{frame}
    \sectionpage
\end{frame}

\begin{frame}
    \tableofcontents[currentsection, hideothersubsections]
\end{frame}
        
        \begin{frame}
            \frametitle{Preslikava}

            \vskip-2em
            \begin{columns}
                \column{0.73\textwidth}
                    \begin{alertblock}{Preslikava}
                        Naj bosta $\mathcal{X}$ in $\mathcal{Y}$ neprazni množici. \\
                        \textbf{Preslikava} $f$ sestoji iz:
                        \begin{itemize}
                            \item množice $\mathcal{X}$, ki ji pravimo \textbf{domena},
                            \item množice $\mathcal{Y}$, ki ji pravimo \textbf{kodomena} in 
                            \item \textbf{prirejanja}, ki vsakemu elementu $x$ domene priredi natanko en element $y$ kodomene.
                        \end{itemize}
                    \end{alertblock}

                \column{0.24\textwidth}

                    \begin{alertblock}{}
                        $$\begin{aligned}
                            f&: \mathcal{X}\to\mathcal{Y} \\ f&: x\mapsto y
                        \end{aligned}  $$
                    \end{alertblock}
            
        \end{columns}

        \vskip-0.4em
        \begin{alertblock}{}
            Elemente $x$ kodomene $\mathcal{X}$ imenujemo \textbf{originali} preslikave.
            \\ Če elementu $x$ priredimo element $y$ iz kodomene, potem imenujemo $y$ \textbf{slika} elemeta $x$.
        \end{alertblock}

        \vskip-0.4em
        \begin{block}{}
            Preslikavo lahko podamo s predpisom, puščičnim diagramom, besednim opisom ...
        \end{block}

        \end{frame}


        \begin{frame}
            \frametitle{Funkcija}

            \begin{alertblock}{Funkcija}
                Naj bosta $\mathcal{X}$ in $\mathcal{Y}$ neprazni številski množici. \\
                \textbf{Funkcija} $f$ je preslikava med številskima množicama $\mathcal{X}$ in $\mathcal{Y}$:
                \vskip-0.5em
                $$f: \mathcal{X}\to\mathcal{Y}.$$
            \end{alertblock}


            \begin{alertblock}{}
                Število $y$ je \textbf{funkcijska vrednost} števila $x$, če se število $x$ preslika v število $y$. 
                \vskip-0.5em
                $$ f(x)=y $$
            \end{alertblock}

            \begin{block}{}
                $x$ je neodvisna spremenjlivka, $f(x)$ je od $x$ odvisna spremenljivka.
            \end{block}

        \end{frame}

        
        \begin{frame}
            
            \begin{block}{}
                V nekaterih primerih za opis funkcije uporabimo poseben izraz:
                \begin{itemize}
                    \item $f:\mathcal{X}\to\mathbb{R}; \mathcal{X}\subseteq\mathbb{R}$ -- realna funkcija realne spremenljivke;
                    \item $f:\mathcal{X}\to\mathbb{R}; \mathcal{X}\subseteq\mathbb{N}$ -- realna funkcija naravne spremenljivke;
                    \item $f:\mathcal{X}\to\mathbb{N}; \mathcal{X}\subseteq\mathbb{R}$ -- naravna funkcija realne spremenljivke;
                    \item $f:\mathcal{X}\to\mathbb{N}; \mathcal{X}\subseteq\mathbb{N}$ -- naravna funkcija naravne spremenljivke.
                \end{itemize}
            \end{block}
        \end{frame}

        \begin{frame}
            \frametitle{Definicijsko območje in zaloga vrednosti}

            \begin{alertblock}{Definicijsko območje}
                \textbf{Definicijsko območje} preslikave ali funkcije $f:\mathcal{X}\to\mathcal{Y}$ je množica vseh originalov, ki jih v danem primeru opazujemo. 
                Oznaka: $D_f$.                
            \end{alertblock}

            \vskip-0.5em
            \begin{block}{}
                Za definicijsko območje navadno vzamemo največjo možno množico, za katero je predpis funkcije veljaven/definiran.
            \end{block}

            \begin{alertblock}{Zaloga vrednosti}
                \textbf{Zaloga vrednosti} preslikave ali funkcije $f:\mathcal{X}\to\mathcal{Y}$ je množica vseh slik oziroma funkcijskih vrednosti.
                Oznaka: $Z_f$.
            \end{alertblock}

            \vskip-0.5em
            \begin{block}{}
                Zaloga vrednosti $Z_f$ je podmnožica kodomene $\mathcal{Y}$: $Z_f\subseteq \mathcal{Y}$.
            \end{block}

        \end{frame}



        %%%% naloge

        \begin{frame}

            \only<2->{\begin{exampleblock}{Naloga}
                Funkcijo $f: A\to B$ predstavite s tabelo. Izračunajte, kam posamezna funkcija preslika $x=1$.
                \begin{itemize}
                    \item $A=\left\{-2, -1, 0, 1, 2, 3\right\}$, $B=\left\{0, 1, 2, 3, 4, 5\right\}$, $f(x)=|x|+1$ \\~
                    \item $A=\left\{1, 2, 3, 4, 5\right\}$, $B=\mathbb{N}$, $f(x)=2x+1$ \\~
                    \item $A=B=\left\{\frac{1}{3}, \frac{1}{2}, 1, 2, 3\right\}$, $f(x)=\dfrac{1}{x}$ \\~
                \end{itemize} 
            \end{exampleblock}}
           
            \only<3->{\begin{exampleblock}{Naloga}
                Tabelirajte funkcijo $g(x)=2x+|x|$ od $-3$ do $3$ s korakom $1$. \\~
            \end{exampleblock}}


        \end{frame}


        \begin{frame}
            \only<2->{\begin{exampleblock}{Naloga}
                Zapišite definicijska območja funkcij.
                \vskip-0.5em
                \begin{columns}
                    \column{0.5\textwidth}
                    \begin{itemize}
                        \item $f(x)=\dfrac{-7}{x+1}$ \\~
                        \item $g(x)=\dfrac{1}{(x+2)(x+6)}$ \\~
                        \item $h(x)=\dfrac{3x^2+1}{5}$ \\~
                        \item $i(x)=\sqrt{x-2}$ \\~
                    \end{itemize}

                    \column{0.47\textwidth}
                    \begin{itemize}
                        \item $j(x)=x^3-\frac{2}{3}$ \\~
                        \item $k(x)=\sqrt{x^2+7}$ \\~
                        \item $l(x)=\dfrac{3}{x}$ \\~
                        \item $m(x)=\dfrac{x^2+1}{x^2-5x-6}$ \\~
                    \end{itemize}
                \end{columns}

            \end{exampleblock}}
        \end{frame}

        %%%%%%%%%%%%%%%%%%%%%%%%%%%%%%%%%%%%%%%%%%%%%%%%


        \begin{frame}
            \frametitle{Ničla in začetna vrednost funkcije}

            \begin{alertblock}{Ničla funkcije}
                \textbf{Ničla} funkcije $f:\mathcal{X}\to\mathcal{Y}$ je tista vrednost $x_0\in\mathcal{X}$ neodvisne spremenljivke, 
                pri kateri je vrednost funkcije $f$ enaka $0$: $f(x_0)=0$.
            \end{alertblock}

            \vskip-0.5em
            \begin{block}{}
                Ničle funkcije $f$ poiščemo tako, da rešimo enačbo $f(x)=0$. \\
                Ničle so le tiste izmed vrednosti, ki ležijo v definicijskem območju $D_f$ funkcije $f$.
            \end{block}

            \begin{alertblock}{Začetna vrednost}
                \textbf{Začetna vrednost} funkcije $f:\mathcal{X}\to\mathcal{Y}$ je funkcijska vrednost pri $x=0$, to je $f(0)$.
            \end{alertblock}

            \vskip-0.5em
            \begin{block}{}
                Začetna vrednost obstaja le, če je $0$ v definicijskem območju funkcije $f$: $0\in D_f$.
            \end{block}
        \end{frame}



        %%%% naloge

        \begin{frame}
            \only<2->{\begin{exampleblock}{Naloga}
                Izračunajte ničle funkcij.
                \vskip-0.5em
                \begin{columns}
                    \column{0.5\textwidth}
                    \begin{itemize}
                        \item $f(x)=\frac{4}{5}-6x$ \\~
                        \item $g(x)=x^2-7x+12$ \\~
                        \item $h(x)=\dfrac{3x+6}{5}$ \\~
                        \item $i(x)=x^2-9$ \\~
                    \end{itemize}

                    \column{0.47\textwidth}
                    \begin{itemize}
                        \item $j(x)=x^2+1$ \\~
                        \item $k(x)=x^2-3x^2-4x+12$ \\~
                        \item $l(x)=\sqrt{x+7}$ \\~
                        \item $m(x)=\dfrac{3}{x}$ \\~
                    \end{itemize}
                \end{columns}

            \end{exampleblock}}
        \end{frame}


        \begin{frame}
            \only<2->{\begin{exampleblock}{Naloga}
                Izračunajte začetne vrednosti funkcij.
                \vskip-0.5em
                \begin{columns}
                    \column{0.5\textwidth}
                    \begin{itemize}
                        \item $f(x)=\frac{4}{5}-6x$ \\~
                        \item $g(x)=x^2-7x+12$ \\~
                        \item $h(x)=\dfrac{3x+6}{5}$ \\~
                        \item $i(x)=x^2-9$ \\~
                    \end{itemize}

                    \column{0.47\textwidth}
                    \begin{itemize}
                        \item $j(x)=x^2-3x^2-4x+12$ \\~
                        \item $k(x)=\sqrt{x+7}$ \\~
                        \item $l(x)=\dfrac{3}{x}$ \\~
                        \item $m(x)=\dfrac{x^3-2x^2-4}{x^4+2x^3+3}$ \\~
                    \end{itemize}
                \end{columns}

            \end{exampleblock}}
        \end{frame}


        %%%%%%%%%%%%%%%%%%%%%%%%%%%%%%%


        \begin{frame}
            \frametitle{Graf funkcije}


        \end{frame}

        %%%% naloge

        \begin{frame}
            \only<2->{\begin{exampleblock}{Naloga}
                Narišite grafe funkcij in zapišite začetne vrednosti in ničle, če jih funkcija ima.
                    \begin{itemize}
                        \item $f(x)=x \quad \quad D_f=\mathbb{R}$ \\~
                        \item $g(x)=-2x+1 \quad \quad D_g=\mathbb{R}$ \\~
                        \item $h(x)=x^2-1 \quad \quad D_h=\mathbb{R}$ \\~
                        \item $i(x)=\dfrac{1}{x^2} \quad \quad D_i=\left\{-2, -1, -\frac{1}{2}, \frac{1}{2}, 1, 2\right\}$ \\~
                        \item $j(x)=\dfrac{x+2}{x-3} \quad \quad D_j=\left\{-2, -1, 0, 1, 2\right\}$ \\~
                    \end{itemize}
            \end{exampleblock}}
        \end{frame}





    \subsection{Linearna funkcija}

        \begin{frame}
            \frametitle{Predpis linearne funkcije}
        \end{frame}


        %%% naloge

        \begin{frame}
            \only<2->{\begin{exampleblock}{Naloga}
                Ugotovite, ali je dana funkcija linearna. Linearnim funkcijam določite smerni koeficient in začetno vrednost.
                \vskip-0.5em
                \begin{columns}
                    \column{0.5\textwidth}
                    \begin{itemize}
                        \item $f(x)=\dfrac{1}{7x}-\dfrac{3}{4}$ \\~
                        \item $g(x)=\frac{2}{3}-\pi x$ \\~
                        \item $h(x)=\dfrac{8+6x}{24}$ \\~
                        \item $i(x)=0.\overline{3}x+1$ \\~
                    \end{itemize}

                    \column{0.47\textwidth}
                    \begin{itemize}
                        \item $j(x)=\dfrac{x^2-3}{5}$ \\~
                        \item $k(x)=-\sqrt{2}x+\frac{2}{3}$ \\~
                        \item $l(x)=2$ \\~
                    \end{itemize}
                \end{columns}

            \end{exampleblock}}
        \end{frame}



        \begin{frame}
            \only<2->{\begin{exampleblock}{Naloga}
                Zapišite predpis linearne funkcije $f$, ki ima začetno vrednost $5$ in diferenčni količnik $-3$. \\~
            \end{exampleblock}}

            \only<3->{\begin{exampleblock}{Naloga}
                Dana je linearna funkcija $f(x)=3x-4$. Izračunaj $f(-2)$, $f(0)$; $f(5)$ in $f(\sqrt{2})$. \\~
            \end{exampleblock}}

            \only<4->{\begin{exampleblock}{Naloga}
                Zapišite predpis linearne funkcije, za katero je $u(-2)=10$ in $u(0)=2$. \\~
            \end{exampleblock}}

        \end{frame}


        \begin{frame}
            \only<2->{\begin{exampleblock}{Naloga}
                Ali je funkcija naraščajoča ali padajoča?.
                \vskip-0.5em
                \begin{columns}
                    \column{0.5\textwidth}
                    \begin{itemize}
                        \item $f(x)=3x+5$ \\~
                        \item $g(x)=-2x+7$ \\~
                        \item $h(x)=10-\frac{1}{2}x$ \\~
                        \item $i(x)=\dfrac{x-1}{2}$ \\~
                    \end{itemize}

                    \column{0.47\textwidth}
                    \begin{itemize}
                        \item $j(x)=\dfrac{5-2x}{3}$ \\~
                        \item $k(x)=\dfrac{-\sqrt{3}x+1}{3}$ \\~
                        \item $l(x)=-\dfrac{2-4x}{17}$ \\~
                    \end{itemize}
                \end{columns}

            \end{exampleblock}}
        \end{frame}


        \begin{frame}
            \only<2->{\begin{exampleblock}{Naloga}
                Izračunajte ničlo linearne funkcije.
                \vskip-1.5em
                \begin{columns}
                    \column{0.5\textwidth}
                    \begin{itemize}
                        \item $f(x)=6x+12$ \\~
                        \item $g(x)=5x+2$ \\~
                        \item $h(x)=3x-12$ \\~
                        \item $i(x)=-4x+8$ \\~
                        \item $j(x)=-3x+2$ \\~
                        \item $k(x)=-x-7$ \\~
                    \end{itemize}

                    \column{0.47\textwidth}
                    \begin{itemize}
                        \item $l(x)=\frac{}{4}x-\frac{1}{4}$ \\~
                        \item $m(x)=-\dfrac{2x+3}{6}$ \\~
                        \item $n(x)=\dfrac{1-4x}{2}$ \\~
                        \item $o(x)=\dfrac{\pi x+4}{3}$ \\~
                        \item $p(x)=\sqrt{2}x+1$ \\~
                        \item $r(x)=4$ 
                    \end{itemize}
                \end{columns}

            \end{exampleblock}}
        \end{frame}



    \subsection{Predpis linearne funkcije}

        \begin{frame}
            \frametitle{Predpis linearne funkcije}
        \end{frame}


    \subsection{Graf linearne funkcije}

        \begin{frame}
            \frametitle{Graf linearne funkcije}
        \end{frame}

        