    

        \begin{naloga}
            Rešite kvadratno enačbo.
            

                \begin{itemize}
                    \item $ x^2-14x+24=0 $ 
                    \item $ -x^2+10x+39=0 $ 
                    \item $ 2x^2+24x+70=0 $ 
                    \item $ \frac{1}{2}x^2+x-60=0 $ 
                    \item $ x^2-10x+25=0 $ 
                    \item $ x^2-9=0 $
                \end{itemize}


                \begin{itemize}
                    \item $ 3x^2-2x=2x^2-35-14x $ 
                    \item $ x^2-10x=36-x^2+4x $ 
                    \item $ x^2-10x=5x-2x^2 $ 
                    \item $ 70+x^2-x=2x^2-2x-2 $ 
                    \item $ 2x^2-10x=5x+x^2 $ 
                    \item $ 3x^2-4x=25-4x+2x^2 $
                \end{itemize}
            
        \end{naloga}
    


    
            \begin{naloga}
                Rešite kvadratno enačbo.
                
                    \begin{itemize}
                        \item $ 4x^2+5x-6=0 $ 
                        \item $ 12x^2+11x+2=0 $ 
                        \item $ 3x^2+1x-8=0 $ 
                        \item $ x^2-6x+2=0 $
                    \end{itemize}

            \end{naloga}

            \begin{naloga}
                Rešite enačbo
                
                    \begin{itemize}
                        \item $ 2x^3-5x^2-3x=0 $ 
                        \item $ \left(2x-1\right)^2-5\left(2x-1\right)+6=0 $ 
                        \item $ \dfrac{1}{x}-\dfrac{1}{x+2}=\dfrac{2}{15} $ 
                    \end{itemize}

            \end{naloga}


    


    

        \begin{naloga}
                Izračunajte ničli kvadratne funkcije in jo zapišite v faktorizirani obliki.
                
                    \begin{itemize}
                        \item $ f(x)=2x^2-x-1 $ 
                        \item $ g(x)=4x^2+2x+2 $ 
                        \item $ h(x)=-3x^2-4x+4 $ 
                        \item $ i(x)=8x^2-2x+3 $ 
                    \end{itemize}

            \end{naloga}


    


    

            \begin{naloga}
                V splošni obliki zapišite predpis kvadratne funkcije, ki:
                                
                    \begin{itemize}
                        \item ima ničli $x_1=-2$ in $x_2=3$ ter začetno vrednost $f(0)=-12$.
                        \item ima ničli $x_1=1$ in $x_2=3$, največja vrednost, ki jo zavzame je $5$.
                        \item ima ničli $x_1=-7$ in $x_2=1$, $x=1$ pa preslika v $y=4$.
                        \item ima dvojno ničlo $x_{1,2}=-3$ in začetno vrednost $i(0)=3$.

                    \end{itemize}

            \end{naloga}


            \begin{naloga}
                Zapišite enačbo parabole, ki:
                
                    \begin{itemize}
                        \item seka abscisno od v $x_1=-1$ in $x_2=4$, ordinatno os pa pri $8$.
                        \item seka abscisno od v $x_1=-1$ in $x_2=5$, teme pa leži na premici $y=9$.
                        \item seka abscisno od v $x_1=4$ in $x_2=7$, gre skozi točko $A(2,20)$.
                        \item seka abscisno od v $x_1=-2$ in $x_2=-6$, zaloga vrednosti pa je $(-\infty,2]$.
                    \end{itemize}

            \end{naloga}

    


    

            \begin{naloga}
                V faktorizirani obliki zapišite kvadratno funkcijo, ki ima:
                                
                    \begin{itemize}
                        \item teme v točki $T(7,-3)$ in ničlo $x_1=6$.
                        \item teme v točki $T(1,9)$ in ničlo $x_1=-2$.
                        \item teme v točki $T(3,-4)$ in ničlo $x_1=-1$.

                    \end{itemize}

            \end{naloga}


            \begin{naloga}
                V temenski obliki zapišite kvadratno funkcijo, ki ima:
                
                    \begin{itemize}
                        \item ničli $x_1=-5$ in $x_2=3$, teme pa v točki $T(x,32)$.
                        \item ničli $x_1=-\frac{1}{2}$ in $x_2=\frac{5}{2}$, teme pa v točki $T(x,-9)$.
                        \item ničli $x_1=-4$ in $x_2=2$, teme pa v točki $T(x,18)$.
                    \end{itemize}

            \end{naloga}

    


    

            \begin{naloga}
                Dana je družina kvadratnih funkcij.
                Za katero vrednost parametra $m$ ima funkcija eno dvojno ničlo?
                Izračunajte tudi ničlo.
                                
                    \begin{itemize}
                        \item $f(x)=4x^2+(m+1)x+1$ 
                        \item $g(x)=-2x^2+mx-x-18$
                        \item $h(x)=-x^2+mx-x+m-1$

                    \end{itemize}

            \end{naloga}


            \begin{naloga}
                Dana je družina parabol.
                Za katero vrednost parametra $n$ se parabola dotika abscisne osi. 
                Izračunajte dotikališče.
                
                    \begin{itemize}
                        \item $y=2x^2+(n-3)x+2$
                        \item $y=-4x^2+mx-2x-1$
                    \end{itemize}

            \end{naloga}

    