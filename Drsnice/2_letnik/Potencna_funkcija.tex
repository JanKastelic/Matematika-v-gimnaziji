\section{Potenčna funkcija}

\begin{frame}
    \sectionpage
\end{frame}

\begin{frame}
    \tableofcontents[currentsection, hideothersubsections]
\end{frame}


%%%%%%%%%%%%%%%%%%%%%%%%%%%%%%%%%%%%%%%%%%%

\subsection{Potenčna funkcija z naravnim eksponentom}

        \begin{frame}
            \frametitle{Potenčna funkcija z naravnim eksponentom}
        \end{frame}


    %%%%%%% naloge

            \begin{frame}
                \only<2->{\begin{exampleblock}{Naloga}
                    Katere izmed točk $(1,27)$, $(-1,9)$, $(10,157)$ ležijo na grafu funkcije $f(x)=2\left(x-3\right)^4-5$?
                    \\~
                \end{exampleblock}}


                \only<3->{\begin{exampleblock}{Naloga}
                    Dana je funkcija $f(x)=x^3$. Zapišite predpis za funkcijo $g$, katere graf je premaknjen:
                    \begin{itemize}
                        \item za $2$ v levo in za $3$ navzgor;
                        \item za $3$ v desno in za $2$ navzgor;
                        \item Za $1$ v levo in za $5$ navzdol;
                        \item za $4$ v desno in za $1$ nvazdol.
                    \end{itemize}
                    ~
                \end{exampleblock}}

            \end{frame}


            \begin{frame}

                \only<2->{\begin{exampleblock}{Naloga}
                    Dana je funkcija $f(x)=\left(x+3\right)^3+1$. Zapišite predpis za funkcijo $g$, katere graf je premaknjen:
                    \begin{itemize}
                        \item za $2$ v levo in za $3$ navzgor;
                        \item za $3$ v desno in za $2$ navzgor;
                        \item Za $1$ v levo in za $5$ navzdol;
                        \item za $4$ v desno in za $1$ nvazdol;
                        \item za $1$ v desno in za $3$ navzdol;
                        \item za $5$ v levo in za $4$ navzdol.
                    \end{itemize}
                    ~
                \end{exampleblock}}

            \end{frame}


            \begin{frame}

                \only<2->{\begin{exampleblock}{Naloga}
                    Graf funkcije $g$ smo dobili s togim premikom grafa funkcije $f(x)=x^2$. Zapišite vektor premika.
                    Narišite graf. V kateri točki ima funkcija $g$ teme?
                    \begin{itemize}
                        \item $g(x)=\left(x-3\right)^2+1$ \\~
                        \item $g(x)=\left(x-2\right)^2-1$ \\~
                        \item $g(x)=\left(x+3\right)^2+4$ \\~
                        \item $g(x)=\left(x+1\right)^2-5$ \\~
                    \end{itemize}
                    ~
                \end{exampleblock}}

            \end{frame}


            \begin{frame}

                \only<2->{\begin{exampleblock}{Naloga}
                    Z grafa funkcije $f(x)=\left(x+a\right)^n+b$ razberite vrednosti parametrov $a$, $b$ in $n$.
                    
                    \begin{figure}[H]
                        \centering
                    \end{figure}

                \end{exampleblock}}

            \end{frame}


            \begin{frame}

                \only<2->{\begin{exampleblock}{Naloga}
                    Narišite graf funkcije $f$, potem pa v isti koordinatni sistem še graf funkcije $g$.

                    \begin{itemize}
                        \item $f(x)?x^3$, $g(x)=\frac{1}{2}x^3$ \\~
                        \item $f(x)=x^2$, $g(x)=-2x^2$ \\~
                        \item $f(x)=x^4$, $g(x)=-x^4$ \\~
                        \item $f(x)=x^3$, $g(x)=\left|2x^3\right|$ \\~
                    \end{itemize}
                    ~
                \end{exampleblock}}

            \end{frame}


            \begin{frame}

                \only<2->{\begin{exampleblock}{Naloga}
                    Z grafa funkcije $f(x)=a\left(x-p\right)^2+q$ razberite vrednosti parametrov $a$, $p$ in $q$.
                    
                    \begin{figure}[H]
                        \centering
                    \end{figure}

                \end{exampleblock}}

            \end{frame}


            \begin{frame}

                \only<2->{\begin{exampleblock}{Naloga}
                    Z grafa funkcije $f(x)=a\left(x-p\right)^3+q$ razberite vrednosti parametrov $a$, $p$ in $q$.
                    
                    \begin{figure}[H]
                        \centering
                    \end{figure}

                \end{exampleblock}}

            \end{frame}


            \begin{frame}
                \only<2->{\begin{exampleblock}{Naloga}
                    Izračunajte presečišče grafa dane funkcije $f$ in dane premice.

                    \begin{itemize}
                        \item $f(x)=\left(x-3\right)^2-2$ in $y=-2x+4$
                        \item $f(x)=2\left(x-1\right)^2+4$ in $y=6$
                        \item $f(x)=-\frac{1}{2}x^2+3$ in $y=x-1$
                    \end{itemize}
                    ~
                \end{exampleblock}}


                \only<3->{\begin{exampleblock}{Naloga}
                    Izračunajte presečišče grafov danih funkcij $f$ in $g$.

                    \begin{itemize}
                        \item $f(x)=\left(x-3\right)^2$ in $g(x)=x^2+3$
                        \item $f(x)=\left(x-3\right)^2-2$ in $g(x)=\left(x-4\right)^2+1$
                        \item $f(x)=-x^2+2$ in $g(x)=\left(x-1\right)^2+1$
                    \end{itemize}
                    ~
                \end{exampleblock}}

            \end{frame}


            \begin{frame}

                \only<2->{\begin{exampleblock}{Naloga}
                    Naj bo prvič funkcija $f$ dana s predpisom $f(x)=x^2$, drugič pa s $f(x)=x^3$.
                    Zapišite predpis funkcije $g$ za oba primera in narišite oba grafa.
                    
                    \begin{columns}

                    \column{0.48\textwidth}
                    \begin{itemize}
                        \item $g(x)=f(x-2)$ \\~
                        \item $g(x)=f(x+1)$ \\~
                        \item $g(x)=f(x)+1$ \\~
                        \item $g(x)=f(x)-2$ \\~
                        \item $g(x)=f(x+1)-3$ \\~
                    \end{itemize}

                    \column{0.48\textwidth}
                    \begin{itemize}
                        \item $g(x)=-f(x)+1$ \\~
                        \item $g(x)=-f(x-2)+1$ \\~
                        \item $g(x)=\left|f(x)-1\right|$ \\~
                        \item $g(x)=2f(x)$ \\~
                        \item $g(x)=f(|x|)+1$ \\~
                    \end{itemize}
                    \end{columns}
                    
                \end{exampleblock}}

            \end{frame}




%%%%%%%%%%%%%%%%%%%%%%%%%%%%%%%%%%%%%%%%%%%%

\subsection{Potenčna funkcija z negativnim celim eksponentom}

        \begin{frame}
            \frametitle{Potenčna funkcija z negativnim celim eksponentom}
        \end{frame}


    %%%%%%% naloge




%%%%%%%%%%%%%%%%%%%%%%%%%%%%%%%%%%%%%%%%%%%%%

\subsection{Modeliranje s potenčno funkcijo}

        \begin{frame}
            \frametitle{Modeliranje s korensko in potenčno funkcijo}
        \end{frame}
