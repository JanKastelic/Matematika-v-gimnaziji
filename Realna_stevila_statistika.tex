\section{Realna števila, statistika}

\begin{frame}
    \sectionpage
\end{frame}

\begin{frame}
    \tableofcontents[currentsection, hideothersubsections]
\end{frame}

    \subsection{Realna števila}

        \begin{frame}
            \frametitle{Realna števila}
        \end{frame}

    \subsection{Kvadratni in kubični koren}

        \begin{frame}
            \frametitle{Kvadratni in kubični koren}
        \end{frame}

        \begin{frame}
            \begin{exampleblock}{Naloga 563}
                Izračunaj in rezultat delno koreni.
                \begin{description}
                    \item<2->[(b)] $\displaystyle 4\sqrt{8}-\left(2\sqrt{5}+3\sqrt{8}\right)\sqrt{10}$
                    \item<3->[(č)] $\displaystyle\left(5\sqrt{3}+2\sqrt{27}\right)\left(\sqrt{75}-4\sqrt{12}+\sqrt{147}\right)$
                    \item<4->[(g)] $\displaystyle 8\sqrt{3}\left(\sqrt{2}-1\right)-\left(\sqrt{5}+2\sqrt{6}\right)\left(4-2\sqrt{2}\right)$  
                    \item<5->[(j)] $\displaystyle\left(2-4\sqrt{3}\right)\cdot 3\sqrt{2}-\left(2\sqrt{2}-3\sqrt{3}\right)^2$
                    \item<6->[(l)] $\displaystyle\left(3-2\sqrt{2}\right)^3-\left(\sqrt{8}-5\sqrt{2}\right)\left(-3\sqrt{2}\right)$
                    \item<7->[(o)] $\displaystyle\sqrt{300}-\sqrt{5-2\sqrt{6}}\cdot\sqrt{5+2\sqrt{6}}+\sqrt{5^4}$
                    \item<8->[(r)] $\displaystyle\sqrt{5\sqrt{3}-5}\cdot\sqrt{2\sqrt{3}+2}-\left(\sqrt{5}\right)^3$  
                    \item<9->[(u)] $\displaystyle\left(\sqrt{17}-3\right)\sqrt{26+6\sqrt{17}}-\sqrt{2}\left(\sqrt{2}+\sqrt{6}\right)$
                \end{description}
            \end{exampleblock}
        \end{frame}

    \subsection{Intervali}

        \begin{frame}
            \frametitle{Intervali}
        \end{frame}

    \subsection{Absolutna vrednost}

        \begin{frame}
            \frametitle{Absolutna vrednost}
        \end{frame}

    \subsection{Sistem linearnih enačb}

        \begin{frame}
            \frametitle{Sistem linearnih enačb}
        \end{frame}

    \subsection{Obravnavanje linearnih enačb, neenačb, sistemov}

        \begin{frame}
            \frametitle{Obravnavanje linearnih enačb, neenačb, sistemov}
        \end{frame}

    \subsection{Absolutna in relativna napaka}

        \begin{frame}
            \frametitle{Absolutna in relativna napaka}
        \end{frame}

    \subsection{Sredine}

        \begin{frame}
            \frametitle{Sredine}
        \end{frame}

    \subsection{Razpršenost podatkov}

        \begin{frame}
            \frametitle{Razpršenost podatkov}
        \end{frame}

    \subsection{Prikazi}
        
        \begin{frame}
            \frametitle{Prikazi}
        \end{frame}