\begin{priprava}{2., 3}{}{Ničle in začetna vrednost funkcije}{Funkcija}{frontalna}{drsnice, projekcija, tabla}


    \section{Ničla in začetna vrednost funkcije}

    \subsubsection*{Ničla funkcije}
        \textbf{Ničla} funkcije $f:\mathcal{X}\to\mathcal{Y}$ je tista vrednost $x_0\in\mathcal{X}$ neodvisne spremenljivke, 
        pri kateri je vrednost funkcije $f$ enaka $0$: $f(x_0)=0$.

        ~

        Ničle funkcije $f$ poiščemo tako, da rešimo enačbo $f(x)=0$. \\
        Ničle so le tiste izmed vrednosti, ki ležijo v definicijskem območju $D_f$ funkcije $f$.

        
    \subsubsection*{Začetna vrednost}
        \textbf{Začetna vrednost} funkcije $f:\mathcal{X}\to\mathcal{Y}$ je funkcijska vrednost pri $x=0$, to je $f(0)$.

        ~

        Začetna vrednost obstaja le, če je $0$ v definicijskem območju funkcije $f$: $0\in D_f$.

        
~\\~\\


%%%% naloge


    \begin{naloga}
        Izračunajte ničle funkcij.
                \begin{multicols}{2}

            \begin{itemize}
                \item $f(x)=\frac{4}{5}-6x$ 
                \item $g(x)=x^2-7x+12$ 
                \item $h(x)=\dfrac{3x+6}{5}$ 
                \item $i(x)=x^2-9$ 
                \item $j(x)=x^2+1$ 
                \item $k(x)=x^2-3x^2-4x+12$ 
                \item $l(x)=\sqrt{x+7}$ 
                \item $m(x)=\dfrac{3}{x}$ 
            \end{itemize}
        \end{multicols}
    \end{naloga}


    \begin{naloga}
        Izračunajte začetne vrednosti funkcij.
        \begin{multicols}{2}
            \begin{itemize}
                \item $f(x)=\frac{4}{5}-6x$ 
                \item $g(x)=x^2-7x+12$ 
                \item $h(x)=\dfrac{3x+6}{5}$ 
                \item $i(x)=x^2-9$ 
                \item $j(x)=x^2-3x^2-4x+12$ 
                \item $k(x)=\sqrt{x+7}$ 
                \item $l(x)=\dfrac{3}{x}$ 
                \item $m(x)=\dfrac{x^3-2x^2-4}{x^4+2x^3+3}$ 
            \end{itemize}
        \end{multicols}
    \end{naloga}



\end{priprava}