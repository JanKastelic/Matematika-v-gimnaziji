\begin{priprava}{4}{}{Surjektivnost in injektivnost}{Funkcija}{frontalna}{drsnice, projekcija, tabla}




        \section{Surjektivnost in injektivnost}

            \subsection*{Surjektivnost}
                Funkcija $f:\mathcal{X}\to\mathcal{Y}$ je \textbf{surjektivna}, če je zaloga vrednosti $Z_f$ funkcije enaka njeni kodomeni $\mathcal{Y}$ -- vsak element kodomene $\mathcal{Y}$ je slika vsaj enega elementa iz domene $\mathcal{X}$.

                $$\forall y\in\mathcal{Y}. \exists x\in\mathcal{X}\ni:f(x)=y$$

            \subsection*{Injektivnost}
                Funkcija $f:\mathcal{X}\to\mathcal{Y}$ je \textbf{injektivna}, če se dva poljubna različna originala iz domene $\mathcal{X}$ preslikata v različni sliki v kodomeni $\mathcal{Y}$ -- vsak element kodomene $\mathcal{Y}$ je slika kvečjemu enega elementa iz domene $\mathcal{X}$.

                $$\forall x,y\in\mathcal{X}: f(x)=f(y)\Rightarrow x=y$$


~

            Funkcija $f:\mathcal{X}\to\mathcal{Y}$ je \textbf{bijektivna}, če je injektivna in surjektivna hkrati -- vsak element iz kodomene $\mathcal{Y}$ je slika natanko enega elementa domene $\mathcal{X}$.


            ~\\~\\
            %%%%%% naloge

            \begin{naloga}
                Zapišite in narišite grafe funkcij ter zapišite začetne vrednosti in ničle funkcije.
                %Določite, kje je funkcija naraščajoča oziroma padajoča, ter 
                Preverite surjektivnost in injektivnost funkcije.
                    \begin{itemize}
                        \item $f(x)=x \quad \quad D_f=\mathbb{R}$ 
                        \item $g(x)=-2x+1 \quad \quad D_g=\mathbb{R}$ 
                        \item $h(x)=x^2-1 \quad \quad D_h=\mathbb{R}$ 
                        \item $i(x)=\dfrac{1}{x^2} \quad \quad D_i=\left\{-2, -1, -\frac{1}{2}, \frac{1}{2}, 1, 2\right\}$ 
                        \item $j(x)=\dfrac{x+2}{x-3} \quad \quad D_j=\left\{-2, -1, 0, 1, 2\right\}$ 
                    \end{itemize}
            \end{naloga}                    




\end{priprava}