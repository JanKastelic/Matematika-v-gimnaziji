\begin{priprava}{3}{}{Graf funkcije}{Funkcija}{frontalna}{drsnice, projekcija, tabla}


        \section{Graf funkcije}

            \textbf{Graf} $\Gamma_f$ funkcije $f:\mathcal{X}\to\mathcal{Y}$ je množica urejenih parov $(x,y)\in\mathcal{X}\times\mathcal{Y}$, 
                kjer element $x$ preteče celotno definicijsko območje $D_f$ funkcije, element $y$ pa je slika pripadajočega $x$, torej $y=f(x)$.
            $$ \Gamma_f=\left\{(x,y)\in\mathcal{X}\times\mathcal{Y}; x\in D_f \land y=f(x)\right\} $$
        
~

                Urejene pare iz množice $\Gamma_f$ lahko upodobimo v koordinatnem sistemu. 
               
               ~

                Vsakemu elementu $(x,f(x))$ iz zgornje množice pripada natanko ena točka v koordinatnem sistemu, 
                katere abscisa je enaka $x$, ordinata pa je njegova slika $f(x)$.

                ~

                V ničli graf funkcije seka abscisno os, v začetni vrednosti pa ordinatno os.


~\\~\\
            %%%%%% naloge

            \begin{naloga}
                Zapišite in narišite grafe funkcij ter zapišite začetne vrednosti in ničle funkcije.
                % Določite, kje je funkcija naraščajoča oziroma padajoča, ter preverite surjektivnost in injektivnost.
                    \begin{itemize}
                        \item $f(x)=x \quad \quad D_f=\mathbb{R}$ 
                        \item $g(x)=-2x+1 \quad \quad D_g=\mathbb{R}$ 
                        \item $h(x)=x^2-1 \quad \quad D_h=\mathbb{R}$ 
                        \item $i(x)=\dfrac{1}{x^2} \quad \quad D_i=\left\{-2, -1, -\frac{1}{2}, \frac{1}{2}, 1, 2\right\}$ 
                        \item $j(x)=\dfrac{x+2}{x-3} \quad \quad D_j=\left\{-2, -1, 0, 1, 2\right\}$ 
                    \end{itemize}
            \end{naloga}                    




\end{priprava}