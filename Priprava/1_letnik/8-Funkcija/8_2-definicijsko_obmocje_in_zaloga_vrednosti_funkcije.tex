\begin{priprava}{1., 2}{}{Definicijsko območje in zaloga vrednosti funkcije}{Funkcija}{frontalna}{drsnice, projekcija, tabla}

    \section{Definicijsko območje in zaloga vrednosti}

    \subsubsection*{Definicijsko območje}
        \textbf{Definicijsko območje} preslikave ali funkcije $f:\mathcal{X}\to\mathcal{Y}$ je množica vseh originalov, ki jih v danem primeru opazujemo. 
        Oznaka: $D_f$.                

~

        Za definicijsko območje navadno vzamemo največjo možno množico, za katero je predpis funkcije veljaven/definiran.

        
    \subsubsection*{Zaloga vrednosti}
        \textbf{Zaloga vrednosti} preslikave ali funkcije $f:\mathcal{X}\to\mathcal{Y}$ je množica vseh slik oziroma funkcijskih vrednosti.
        Oznaka: $Z_f$.

~

        Zaloga vrednosti $Z_f$ je podmnožica kodomene $\mathcal{Y}$: $Z_f\subseteq \mathcal{Y}$.


~\\~\\


%%%% naloge

    \begin{naloga}
        Funkcijo $f: A\to B$ predstavite s tabelo. Izračunajte, kam posamezna funkcija preslika $x=1$.
        \begin{itemize}
            \item $A=\left\{-2, -1, 0, 1, 2, 3\right\}$, $B=\left\{0, 1, 2, 3, 4, 5\right\}$, $f(x)=|x|+1$ 
            \item $A=\left\{1, 2, 3, 4, 5\right\}$, $B=\mathbb{N}$, $f(x)=2x+1$ 
            \item $A=B=\left\{\frac{1}{3}, \frac{1}{2}, 1, 2, 3\right\}$, $f(x)=\dfrac{1}{x}$ 
        \end{itemize} 
    \end{naloga}
   
    \begin{naloga}
        Tabelirajte funkcijo $g(x)=2x+|x|$ od $-3$ do $3$ s korakom $1$. 
    \end{naloga}


    \begin{naloga}
        Zapišite definicijska območja funkcij.
        \begin{multicols}{2}
            \begin{itemize}
                \item $f(x)=\dfrac{-7}{x+1}$ 
                \item $g(x)=\dfrac{1}{(x+2)(x+6)}$ 
                \item $h(x)=\dfrac{3x^2+1}{5}$ 
                \item $i(x)=\sqrt{x-2}$ 
                \item $j(x)=x^3-\frac{2}{3}$ 
                \item $k(x)=\sqrt{x^2+7}$ 
                \item $l(x)=\dfrac{3}{x}$ 
                \item $m(x)=\dfrac{x^2+1}{x^2-5x-6}$ 
            \end{itemize}
        \end{multicols}
    \end{naloga}




\end{priprava}