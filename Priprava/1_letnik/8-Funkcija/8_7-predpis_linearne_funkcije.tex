\begin{priprava}{5., 6., 7}{}{Predpis linearne funkcije}{Funkcija}{frontalna}{drsnice, projekcija, tabla}




        \section{Predpis linearne funkcije}
        
                \textbf{Linearna funkcija} je realna funkcija realne spremenljivke, podana s predpisom 
                $$f(x)=kx+n;\quad k,n\in\mathbb{R},$$
                kjer je $k$ \textbf{diferenčni kvocient/smerni koeficient}, $n$ pa \textbf{začetna vrednost} $f(0)=n$.

                ~\\
                Glede na predznak smernega koeficienta $k$ je linearna funkcija:
                \begin{itemize}
                    \item naraščajoča, če je $k>0$;
                    \item konstanta, če je $k=0$ ali
                    \item padajoča, če je $k<0$.
                \end{itemize}

                ~\\~
        %%% naloge

        
            \begin{naloga}
                Ugotovite, ali je dana funkcija linearna. Linearnim funkcijam določite smerni koeficient in začetno vrednost.
                \begin{multicols}{2}
                \begin{itemize}
                        \item $f(x)=\dfrac{1}{7x}-\dfrac{3}{4}$ 
                        \item $g(x)=\dfrac{2}{3}-\pi x$ 
                        \item $h(x)=\dfrac{8+6x}{24}$ 
                        \item $i(x)=0.\overline{3}x+1$ 
                        \item $j(x)=\dfrac{x^2-3}{5}$ 
                        \item $k(x)=-\sqrt{2}x+\dfrac{2}{3}$ 
                        \item $l(x)=2$ 
                    \end{itemize}
                \end{multicols}
            \end{naloga}
        



        
            \begin{naloga}
                Zapišite predpis linearne funkcije $f$, ki ima začetno vrednost $5$ in diferenčni količnik~$-3$. 
            \end{naloga}

            \begin{naloga}
                Dana je linearna funkcija $p(x)=3x-4$. Izračunajte $p(-2)$, $p(0)$; $p(5)$ in $p(\sqrt{2})$. 
            \end{naloga}

            \begin{naloga}
                Zapišite predpis linearne funkcije, za katero je $u(-2)=10$ in $u(0)=2$. 
            \end{naloga}

        


        
            \begin{naloga}
                Ali je funkcija naraščajoča ali padajoča?
                \begin{multicols}{2}
                    \begin{itemize}
                        \item $f(x)=3x+5$ 
                        \item $g(x)=-2x+7$ 
                        \item $h(x)=10-\frac{1}{2}x$ 
                        \item $i(x)=\dfrac{x-1}{2}$ 
                        \item $j(x)=\dfrac{5-2x}{3}$ 
                        \item $k(x)=\dfrac{-\sqrt{3}x+1}{3}$ 
                        \item $l(x)=-\dfrac{2-4x}{17}$ 
                    \end{itemize}
                \end{multicols}
            \end{naloga}
        


        
            \begin{naloga}
                Izračunajte ničlo linearne funkcije.
                \begin{multicols}{2}
                    \begin{itemize}
                        \item $f(x)=6x+12$ 
                        \item $g(x)=5x+2$ 
                        \item $h(x)=3x-12$ 
                        \item $i(x)=-4x+8$ 
                        \item $j(x)=-3x+2$ 
                        \item $k(x)=-x-7$ 
                        \item $l(x)=\frac{3}{4}x-\frac{1}{4}$ 
                        \item $m(x)=-\dfrac{2x+3}{6}$ 
                        \item $n(x)=\dfrac{1-4x}{2}$ 
                        \item $o(x)=\dfrac{\pi x+4}{3}$ 
                        \item $p(x)=\sqrt{2}x+1$ 
                        \item $r(x)=4$ 
                    \end{itemize}
                \end{multicols}
            \end{naloga}
        


        
            \begin{naloga}
                Dana je linearna funkcija $f$. Zapišite predpis funkcije $g$ v obliki $g(x)=kx+n$.
                \begin{multicols}{2} 
                \begin{itemize}
                        \item $f(x)=2x-6, ~ g(x)=3f(x)$ 
                        \item $f(x)=5x-3; ~ g(x)=f(x+1)$ 
                        \item $f(x)=\dfrac{2x-5}{3}; ~ g(x)=f(1-x)$ 
                        \item $f(x)=\dfrac{10-4x}{7}; ~ g(x)=f(3x)$ 
                    \end{itemize}
                \end{multicols}
            \end{naloga}
        


        
            \begin{naloga}
                Dana je družina linearnih funkcij $f(x)=(2m-1)x+(3-m); ~m\in\mathbb{R}$.
                    \begin{itemize}
                        \item Za katero vrednost parametra $m$ ima funkcija diferenčni količnik enak $-5$? 
                        \item Za katero vrednost parametra $m$ je funkcija padajoča?
                        \item Za katero vrednost parametra $m$ je funkcija konstantna? 
                        \item Za katero vrednost parametra $m$ je funkcija naraščajoča? 
                        \item Za katero vrednost parametra $m$ je začetna vrednost enaka $2$? 
                        \item Za katero vrednost parametra $m$ ima funkcija ničlo $x=-4$? 
                    \end{itemize}
            \end{naloga}
        


        
            \begin{naloga}
                Taksist meri razdaljo, ki jo je prevozil. Vsak kilometer stane $2.5~€$, startnina pa $7~€$.
                Zapišite funkcijo, po kateri taksist izračuna znesek za plačilo, ko prebere število prevoženih kilometrov $x$. 
                Izračunajte, koliko bi plačali, če bi se peljali $12~km$. 
            \end{naloga}

            \begin{naloga}
                V bezenu je $12~l$ vode. V bazen po cevi vsako minuto pritečejo še $4~l$ vode.
                Zapišite funkcijo, s katero bomo lahko izračunali, koliko je vode v bazenu po pretečenih $x$ minutah.
                Izračunajte, koliko vode je v bazenu po $9$ minutah. 
            \end{naloga}


        



\end{priprava}