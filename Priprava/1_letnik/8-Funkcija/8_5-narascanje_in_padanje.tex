\begin{priprava}{4}{}{Naraščanje in padanje}{Funkcija}{frontalna}{drsnice, projekcija, tabla}



        \section{Naraščanje in padanje}

            \subsection*{Naraščajoča funkcija}

                Funkcija $f$ je na intervalu $(a,b)$ \textbf{naraščajoča}, če za poljubna $x_1,x_2\in(a,b)$, kjer je $x_1<x_2$, velja $f(x_1)\leq f(x_2)$.

                ~

                Funkcija $f$ je na intervalu $(a,b)$ \textbf{strogo naraščajoča}, če za poljubna $x_1,x_2\in(a,b)$, kjer je $x_1<x_2$, velja $f(x_1)<f(x_2)$.
            

            \subsection*{Padajoča funkcija}
                Funkcija $f$ je na intervalu $(a,b)$ \textbf{padajoča}, če za poljubna $x_1,x_2\in(a,b)$, kjer je $x_1<x_2$, velja $f(x_1)\geq f(x_2)$.
                
                ~

                Funkcija $f$ je na intervalu $(a,b)$ \textbf{strogo padajoča}, če za poljubna $x_1,x_2\in(a,b)$, kjer je $x_1<x_2$, velja $f(x_1)>f(x_2)$.


~\\~\\

            %%%%%% naloge

            \begin{naloga}
                Zapišite in narišite grafe funkcij ter zapišite začetne vrednosti in ničle funkcije.
                Določite, kje je funkcija naraščajoča oziroma padajoča. %, ter preverite surjektivnost in injektivnost.
                    \begin{itemize}
                        \item $f(x)=x \quad \quad D_f=\mathbb{R}$ 
                        \item $g(x)=-2x+1 \quad \quad D_g=\mathbb{R}$ 
                        \item $h(x)=x^2-1 \quad \quad D_h=\mathbb{R}$ 
                        \item $i(x)=\dfrac{1}{x^2} \quad \quad D_i=\left\{-2, -1, -\frac{1}{2}, \frac{1}{2}, 1, 2\right\}$ 
                        \item $j(x)=\dfrac{x+2}{x-3} \quad \quad D_j=\left\{-2, -1, 0, 1, 2\right\}$ 
                    \end{itemize}
            \end{naloga}                    




\end{priprava}