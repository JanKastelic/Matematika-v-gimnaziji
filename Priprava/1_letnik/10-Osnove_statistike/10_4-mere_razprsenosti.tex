\begin{priprava}{4}{}{Mere razpršenosti}{Osnove statistike}{frontalna, delo v dvojicah/individualno}{drsnice, projekcija, računalniki}

    \section{Mere razpršenosti}

        

            
    Informacijo o \textbf{porazdelitvi} oziroma \textbf{razpršenosti} podatkov lahko izračunamo s pomočjo: 
    variacijskega razmika, interkvartilnega ranga, variance in standarnega odklona.

\subsubsection{Variacijski razmik}
    \textbf{Variacijski razmik} $R$ je razlika med maksimalno in minimalno vrednostjo statistične spremenljivke:
    $$R=x_{max}-x_{min}.$$



    Variacijski razmik je zelo odvisen od ekstremnih vrednosti, posebno osamelcev, 
    zato ga uporabljamo le v kombinaciji z drugimi merami razpršenosti.





\subsubsection{Interkvartilni rang}
    \textbf{Interkvartilni rang} oziroma \textbf{medčetrtinski razmik} $IR$ je razlika med vrednostjo prvega in tretjega kvartila:
    $$IR=Q_3-Q_1.$$

~

    \textbf{Osamelec} je podatek, katerega vrednost je za več kot $3$-kratnik interkvartilnega ranga~$IR$ nad tretjim kvartilom $Q_3$ ali pod prvim kvartilom $Q_1$. \\
    Podatek je ``pogojno osamelec'', če je njegova vrednosz za več kot $1.5$-kratnik interkvartilnega ranga~$IR$ nad tretjim kvartilom $Q_3$ ali pod prvim kvartilom $Q_1$.


~   

    Interkvartilni rang je mera razpršenosti, ki ni občutljiva na osamelce.





\subsubsection{Varianca}
    \textbf{Varianca} $\sigma^2$ predstavlja aritmetično sredino kvadratov odmikov vrednosti statistične spremenljivke od aritmetične sredine:
    $$\sigma^2=\dfrac{(x_1-\overline{x})^2+(x_2-\overline{x})^2+\cdots+(x_n-\overline{x})^2}{N}=\dfrac{1}{N}\sum_{i=1}^n(x_i-\overline{x})^2.$$


% 
%     Večja kot je varianca, bolj so podatki razpršeni.
% 

\subsubsection{Standardni odklon}
    \textbf{Standardni odklon} $\sigma$ izračunamo kot koren variance:
    $$\sigma=\sqrt{\dfrac{(x_1-\overline{x})^2+(x_2-\overline{x})^2+\cdots+(x_n-\overline{x})^2}{N}}=\sqrt{\dfrac{1}{N}\sum_{i=1}^n(x_i-\overline{x})^2}.$$
    Predstavlja povprečje odmikov vrednosti statistične spremenljivke od aritmetične sredine.


~\\~

%%% naloge



\begin{naloga}
 
    V preglednici so predstavljene cene treh izdelkov v trgovini po posameznih mesecih leta 2019. 

     \begin{table}[H]
         \centering
         \begin{tabular}{||c|c|c|c|c|c|c|c|c|c|c|c||} 
         \hhline{|t:============:t|}
         \rowcolor[rgb]{0.843,0.718,0.718} 
         Izdelek  & Jan & Feb & Mar & Apr & Maj & Jun & Jul & Avg & Sep & Okt & Nov    \\ 
         \hhline{|:============:|}
         Kruh  & $3.35$ & $3.29$ & $3.34$ & $3.38$ & $3.38$ & $3.37$ & $3.38$ & $3.55$ & $3.53$ & $3.54$ & $3.49$ \\ 
         \hhline{|:============:|}
         Jagode & $8.73$ & $7.18$ & $5.52$ & $4.48$ & $5.72$ & $5.64$ & $6.49$ & $6.58$ & $7.15$ & $7.58$ & $8.34$ \\ 
         \hhline{|:============:|}
         Cvetača & $2.04$ & $2.17$ & $1.58$ & $1.75$ & $2.13$ & $1.85$ & $1.93$ & $1.87$ & $1.81$ & $1.99$ & $1.80$ \\ 
         \hhline{|b:============:b|}
         \end{tabular}
     \end{table}

     Izračunajte povprečno ceno in standardni odklon cene vsakega izdelka.

 
    \end{naloga}





\begin{naloga}

V preglednici je prikazano število rojstev v Sloveniji po letih. 

 \begin{table}[H]
     \centering
     \begin{tabular}{||c|c|c|c|c|c|c|c|c|c||} 
     \hhline{|t:==========:t|}
     \rowcolor[rgb]{0.843,0.718,0.718} 
     Leto   & $2013$ & $2014$ & $2015$ & $2016$ & $2017$ & $2018$ & $2019$ & $2020$ & $2021$    \\ 
     \hhline{|:==========:|}
     Število  & $21111$ & $21165$ & $20641$ & $20345$ & $20241$ & $19585$ & $19328$ & $18767$ & $18989$ \\ 
     \hhline{|b:==========:b|}
     \end{tabular}
 \end{table}

 Izračunajte povprečno število rojstev in standardni odklon.

\end{naloga}


 \begin{naloga}

    Pridobili smo podatke (urejene po velikosti): $1, 13, 14, 15, 15, 15, 17, 18, 18, 19, 19, 19$, $19, 20$ in $40$.
    \begin{itemize}
        \item Opišite razpršenost podatkov $R$, $IR$, $Q_1$, $Q_3$, $\sigma$, $\overline{x}$.
        \item Največjo in najmanjšo vrednost (v tem primeru sta to osamelca) odstranimo. Kako se spremeni razpršenost podatkov? 
    \end{itemize}
    
\end{naloga}





\end{priprava}