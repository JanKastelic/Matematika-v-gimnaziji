\begin{priprava}{1}{}{Osnovni pojmi statistike}{Osnove statistike}{frontalna}{drsnice, projekcija}

    \section{Osnovni pojmi statistike}

        

            
                \textbf{Populacija} je množica, ki jo statistično proučujemo. 
                Element populacije imenujemo \textbf{statistična enota}. 
            
                ~
            
                \textbf{Vzorec} je podmnožica populacije, katere elementi predstavljajo največjo možno mero značilnosti celotne množice. 
                Vzorec izberemo, kadar je celotna populacija prevelika množica, da bi analizirali vse njene elemente. 
                
                \begin{itemize}
                    \item \textbf{Reprezentativen vzorec} je vzorec, ki je izbran tako, da predstavlja značilnosti celotne populacije.
                    \item \textbf{Slučajni vzorec} je vzorec, ki je izbran naključno -- vsi elementi populacije imajo enako možnost, da bodo izbrani.
                \end{itemize}

                \textbf{Numerus} je število elementov vzorca. Oznaka $N$. 
            
                ~
            
                \textbf{Statistična spremenljivka/podatek/znak} je vrednost ali lastnost, ki jo proučujemo.
            

                Vrste statističnih spremenljivk:
                \begin{itemize}
                    \item \textbf{opisne/kvalitativne} statistične spremenljivke
                    \item \textbf{vrstne/ordinalne} statistične spremenljivke
                    \item \textbf{številske/kvantitivne} statistične spremenljivke
                \end{itemize}
                
            

                Številske statistične spremenljivke:
                \begin{itemize}
                    \item \textbf{diskretne} številske spremenljivke -- zavzamejo lahko posamezne vrednosti
                    \item \textbf{zvezne} številske spremenljivke -- zavzamejo lahko vsako vrednost z nekega intervala
                \end{itemize}

            


                ~\\~\\


        %%%% naloge

        
            \begin{naloga}
                Zapišite, kaj je v danem primeru populacija, vzorec, statistična enota, spremenljivka in 
                ugotovite ali je spremenljivka opisna ali numerična in, če je numerična, ugotovite, ali je zvezna ali diskretna.
                \begin{itemize}
                        \item Na spletni strani je anketa z vprašanjem ``Ali imate doma pomivalni stroj?''. Nanjo je odgovorilo $254$ ljudi. 
                        \item V televizijski oddaji gledalci glasujejo za dva kandidata. 
                        \item Razrednik svojih $28$ dijakov vpraša, kolikšna je oddaljenost njihovega doma do šole.
                        \item Maturant piše seminarsko nalogo z naslovom ``Uporaba TikTok-a med srednješolci''. Pridobil je odgovore $369$ srednješolcev, ki so odgovarjali na vprašanje ``Ali~uporabljaš~TikTok?'' 
                        \item Znanstveniki pri raziskavi spremljajo, koliko jajc znesejo kokoši na slovenskih farmah na mesec.
                \end{itemize}
            \end{naloga}
        

        



            \begin{naloga}
                    Slovenija ima več kot $6000$ naselij. Statistični urad Republike Slovenije je januarja 2024 naredil analizo naselij glede na število prebivalcev. 
                    Rezultati so prikazani v tabeli. 

                    \begin{table}[H]
                        \centering
                        \begin{tabular}{||c|c||} 
                        \hhline{|t:==:t|}
                        \rowcolor[rgb]{0.843,0.718,0.718} 
                        velikostni razred naselja  & število naselij   \\ 
                        \hhline{|:==:|}
                        $0$ & $57$    \\ 
                        \hline
                        $1-24$ & $719$    \\ 
                        \hline
                        $25-49$ & $851$    \\ 
                        \hline
                        $50-99$ & $1256$     \\
                        \hline
                        $100-199$ & $1444$     \\
                        \hline
                        $200-499$ & $1109$     \\
                        \hline
                        $500-999$ & $359$     \\
                        \hline
                        $1000-4999$ & $199$     \\
                        \hline
                        $5000-9999$ & $23$     \\
                        \hline
                        $10000-49999$ & $16$     \\                    
                        \hline
                        $50000+$ & $2$     \\
                        \hhline{|b:==:b|}
                        \end{tabular}
                    \end{table}
                
                    Zapišite, kaj je v danem primeru populacija, statistična enota, spremenljivka in ugotovite ali je spremenljivka opisna ali numerična in, 
                    če je numerična, ugotovite ali je zvezna ali diskretna.
            \end{naloga}



\end{priprava}