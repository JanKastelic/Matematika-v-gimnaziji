\begin{priprava}{2}{}{Urejanje in grupiranje podatkov}{Osnove statistike}{frontalna, delo v dvojicah/individualno}{drsnice, projekcija, računalniki}

    \section{Urejanje in grupiranje podatkov}

        

            
                Podatke, pridobljene v posamezni raziskavi, moramo najprej urediti. \\
                Če podatkov ni veliko, jih uredimo po velikosti v \textbf{ranžirno vrsto}, sicer jih združujemo v skupine, \textbf{frekvenčne razrede}.
            

            
                Podatek z največjo vrednostjo označimo z $x_{max}$, podatek z najnižjo vrednostjo pa $x_{min}$.
            
                ~
            
                \textbf{Frekvenca} $f$ statističnega znaka je posamezno število diskretnih statističnih enot iste vrednosti.
            
                ~
            
                \textbf{Frekvenčni razred} je skupina podatkov iz vzorca. Frekvenčni razredi so navadno enako široki,
                in skupaj zajamejo celoten razpon podatkov. Za zvezen nabor podatkov za frekvenčne razrede izberemo intervale (navadno oblike $[a,b)$).
            

        

        
            
                \textbf{Širina frekvenčnega razreda} $d_k$ je razlika med zgornjo ($z_k$) in spodnjo ($s_k$) mejo frekvenčnega razreda:
                $$d_k=z_k-s_k.$$
            

            
                Če so razredi enako široki, določimo njihovo širino kot kvocient med celotnim razponom podatkov $x_{max}-x_{min}$ in številom razredov.
            

            
                \textbf{Sredina frekvenčnega razred} $x_k$ je aritmetična sredina zogrnje in spodnje meje razreda: 
                $$x_k=\dfrac{z_k+s_k}{2}.$$
            
        

        
            
                Grupirane podatke predstavimo s \textbf{frekvenčno preglednico/porazdelitvijo}.
            

            
                Za podatke v frekvenčnih preglednicah računamo:
                \begin{itemize}
                    \item \textbf{(absolutno) frekvenco} $f_k$ -- število podatkov z vrednostmi v danem frekvenčnem razredu;
                    \item \textbf{relativno frekvenco} $f_k'$ -- delež celote, ki ga predstavlja število podatkov v danem frekvenčnem razredu;
                    \item \textbf{(absolutno) kumulativno frekvenco} $F_k$ -- število podatkov, katerih vrednosti zavzemajo manjšo vrednost od zgornje meje danega frekvenčnega razreda;
                    \item \textbf{relativno kumulativno frekvenco} $F_k'$ -- delež celote, ki ga predstavlja število podatkov v danem in vseh manjših frekvenčnih razredih.
                \end{itemize}
            
        
                ~\\~


                %%%% naloge
        
        
                    \begin{naloga}
                        Na šoli analizirajo količino prevzetih obrokov v jedilnici. Rezultati so zbrani v tabeli. 
        
                            \begin{table}[H]
                                \centering
                                \begin{tabular}{||c|c||} 
                                \hhline{|t:==:t|}
                                \rowcolor[rgb]{0.843,0.718,0.718} 
                                Oddelek  & Število prevzetih obrokov   \\ 
                                \hhline{|:==:|}
                                1.a & $12$    \\ 
                                \hline
                                1.b & $14$    \\ 
                                \hline
                                1.c & $20$    \\ 
                                \hline
                                2.a & $17$     \\
                                \hline
                                2.b & $16$     \\
                                \hline
                                2.c & $9$     \\
                                \hline
                                3.a & $13$     \\
                                \hline
                                3.b & $16$     \\
                                \hline
                                3.c & $14$     \\
                                \hline
                                4.a & $21$     \\                    
                                \hline
                                4.b  & $8$     \\
                                \hline
                                4.c  & $12$     \\
                                \hhline{|b:==:b|}
                                \end{tabular}
                            \end{table}
                        
                        Analizirajte podatke s frekvenčno preglednico.
                        Podatke razdelite v razrede $5-9$, $10-14$, $15-19$, $20$ in več.
                    \end{naloga}

                

                
                
                    \begin{naloga}
                        Dijaki 3.~a oddelka so zapisovali svoje pribljubljene barve. \\
                        Zapisali so jih: modra, rdeča, rdeča, zelena, rumena, rdeča, modra, zelena, modra, modra, rumena, rdeča, zelena, modra, rumena, rumena, zelena, rdeča. \\
                        Analizirajte rezultate s frekvenčno preglednico. 
                    \end{naloga}

                    \begin{naloga}
                        Lokostrelec si beleži rezultate treningov. \\
                        Vrednosti so bile: $10.3$, $10.4$, $9.9$, $9.7$, $10.2$, $8.9$, $9.4$, $10.1$, $9.0$, $10.3$, $9.5$, $10.6$. \\
                        Analizirajte rezultate s frekvenčno preglednico. 
                    \end{naloga}
                

                
        
                   \begin{naloga}
                    
                    V frekvenčni preglednici so zbrani podatki o številu sorojencev dijakov 2.~b oddelka.
                    Dopolnite preglednico.
    
                        \begin{table}[H]
                            \centering
                            \begin{tabular}{||c|c|c|c|c||} 
                            \hhline{|t:=====:t|}
                            \rowcolor[rgb]{0.843,0.718,0.718} 
                            Število sorojencev  & $f_k$ & $f_k'$ & $F_k$ & $F_k'$   \\ 
                            \hhline{|:=====:|}
                            0 & $5$  & & &  \\ 
                            \hline
                            1 & & $25~\%$ & &    \\ 
                            \hline
                            2 & & & &    \\ 
                            \hline
                            3 & & $10~\%$ & &   \\
                            \hline
                            skupaj & $20$ & $100~\%$ & / & /    \\
                            \hhline{|b:=====:b|}
                            \end{tabular}
                        \end{table}
                    \end{naloga}
    
            



\end{priprava}