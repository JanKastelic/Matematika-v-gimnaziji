\begin{priprava}{9., 10., 11}{}{Reševanje enačb}{Realna števila}{frontalna}{drsnice, projekcija, tabla}
    
    \section{Reševanje enačb}

        
                \textbf{Enačba} je enakost dveh izrazov, pri čemer vsaj v enem nastopa \textbf{neznanka}, ki je ponavadi označena s črko $x$.

                \textbf{Rešitev enačbe} je vsaka vrednost neznanke, za katero sta vrednosti leve in desne strani enačbe enaki.
            
~

                Enačbo rešujemo tako, da jo preoblikujemo v ekvivalentno enačbo, iz katere preberemo rešitve.

                Ekvivalentno enačbo dobimo, če:
                \begin{itemize}
                    \item na obeh straneh enačbe prištejemo isto število ali izraz;
                    \item obe strani enačbe množimo z istim neničelnim številom ali izrazom.
                \end{itemize}
            
        
                ~~\\
        
                \textbf{Linearna enačba} je enačba oblike $ax+b=0;~a,b\in\mathbb{R}$.

                Rešujemo jo tako, da jo preoblikujemo v ekvivalentno enačbo, ki ima na eni strani samo neznanko.
            
~~

                \textbf{Razcepna enačba} je enačba, v kateri nastopajo potence neznanke (na primer $x^2$, $x^3$) in jo je mogoče zapisati kot produkt (linearnih) faktorjev.

                Preoblikujemo jo v ekvivalentno enačbo, ki ima vse člene na eni strani neenačaja, na drugi pa $0$. 
                Izraz (neničelna stran) razstavimo, kolikor je mogoče, in preberemo rešitve.
            
~~

                \textbf{Racionalna enačba} je enačba, v kateri nastopajo neznake (tudi) v imenovalcu, pri tem smo pozorni na obstoj ulomkov. 
                Nato enačbo preoblikujemo v ekvivalentno enačbo.
            

        ~~~\\



        %%% naloge

        
            \begin{naloga}
                Rešite enačbe.
                \begin{itemize}
                        \item $3(2a-1)-5(a-2)=9$ 
                        \item $2(y-2)+3(1-y)=7$ 
                        \item $3(3-2(t-1))=3(5-t)$ 
                        \item $-(2-x)+3(x+1)=x-5$ 
                \end{itemize}
            \end{naloga}
        


        
            \begin{naloga}
                Rešite enačbe.
                \begin{itemize}
                        \item $\dfrac{1}{5}-\dfrac{x-1}{2}=\dfrac{7}{10}$ 
                        \item $\dfrac{a-1}{3}+\dfrac{a+2}{6}=\dfrac{1}{2}$ 
                        \item $2\dfrac{2}{3}-\dfrac{3t+1}{6}=0$ 
                        \item $\left(\dfrac{2}{b+1}\right)^{-1}+\dfrac{b-1}{4}=b+3$ 
                \end{itemize}
            \end{naloga}
        



        
            \begin{naloga}
                Rešite razcepne enačbe.
                \begin{itemize}
                        \item $x^2-3x=-2$ 
                        \item $(x+2)^3-(x-1)^3=8x^2+x+2$ 
                        \item $x^4=16x^2$ 
                        \item $(x^2-4x+5)^2-(x^2+4x+1)^2-78=2x^2(x+30)-18(x+1)^3$ 
                        \item $x^3-4x^2+4=x$ 
                        \item $x^5=3x^4-2x^3$ 
                \end{itemize}
            \end{naloga}
        


        
            \begin{naloga}
                Rešite enačbe.
                \begin{itemize}
                        \item $\dfrac{x-1}{x+2}=\dfrac{x+1}{x-3}$ 
                        \item $\dfrac{1}{a-1}-\dfrac{3}{a}=\dfrac{2}{a-1}$ 
                        \item $\dfrac{x-3}{x-2}+\dfrac{x+4}{x+1}=\dfrac{2x^2}{x^2-x-2}$ 
                        \item $\dfrac{1}{3a-1}+\dfrac{1}{3a+1}=\dfrac{a-1}{9a^2-1}$ 
                \end{itemize}
            \end{naloga}
        


        
            \begin{naloga}
                Neznano število smo delili s $4$ in dobljenemu količniku prišteli $1$. 
                Dobili smo enako, kot če bi istemu številu prišteli $10$. Izračunajte neznano število.
                
            \end{naloga}

            \begin{naloga}
                Kvadrat neznanega števila je za $4$ manjši od njegovega štirikratnika. Izračunajte neznano število.
                
            \end{naloga}

        


        
            \begin{naloga}
                Avtomobil vozi s povprečno hitrostjo $50~\frac{km}{h}$, kolesar s povprečno hitrostjo $20~\frac{km}{h}$.
                Avtomobil gre iz Lendave v Ormož (približno $50~km$), kolesar vozi v obratno smer. 
                Koliko časa pred avtomobilom mora na pot kolesar, da se bosta srečala na polovici poti?
                
            \end{naloga}

            \begin{naloga}
                Vsota števk dvomestnega števila je $3$. Če zamenjamo njegovi števki, dobimo za $9$ manjše število. Katero število je to?
                
            \end{naloga}

        


        
            \begin{naloga}
                Andreja je bila ob rojstvu hčere Eve stara $38$ let. Čez koliko let bo Andreja stara trikrat toliko kot Eva?
                
            \end{naloga}

            \begin{naloga}
                Prvi delavec sam pozida steno v  $10$ urah, drugi v $12$ urah, tretji v $8$ urah. 
                Delavci skupaj začnejo zidati steno. Po dveh urah tretji delavec odide, pridruži pa se četrti delavec. 
                Skupaj s prvim in drugim delavcem nato končajo steno v eni uri. V kolikšnem času četrti delavec pozida steno?
                
            \end{naloga}

        



    
\end{priprava}