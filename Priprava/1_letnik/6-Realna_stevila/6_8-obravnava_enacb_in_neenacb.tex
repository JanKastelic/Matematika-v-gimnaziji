\begin{priprava}{17}{}{Obravnava enačb in neenačb}{Realna števila}{frontalna}{drsnice, projekcija, tabla}
    
    \section{Obravnava enačb in neenačb}

        

            
            Kadar v enačbi oziroma neenačbi poleg neznake $x$ nastopajo tudi druge črke, na primer $a, b, c, k, l ...$, 
            le-te označujejo števila, ki imajo poljubno realno vrednost. Imenujemo jih \textbf{parametri}.

~
            
            Vrednost parametrov vpliva na rešitev enačbe oziroma neenačbe, zato moramo enačbo reševati glede na vrednosti parametrov.
            Temu postopku rečemo \textbf{obravnava enačbe} oziroma \textbf{obravnava neenačbe}.

        

~~~
        %%%% naloge

        
            \begin{naloga}
                Obravnavajte enačbe.
                \begin{itemize}
                        \item $2(ax-3)+3=ax$ 
                        \item $-4x-b(x-2)^2=3-bx^2-7b$ 
                        \item $3(a-2)(x-2)=a^2(x-1)-4x+7$ 
                        \item $(b-3)^2x-3=4x-3b$ 
                \end{itemize}
            \end{naloga}
        

        
            \begin{naloga}
                Obravnavajte neenačbe.
                \begin{itemize}
                        \item $a(x-2)\leq 4$ 
                        \item $mx+4>m^2-2x$ 
                        \item $a(a-3x+1)\geq a(x-4)+a^2x$ 
                        \item $(k-1)^2x\leq kx+2(k+1)+5x$ 
                \end{itemize}
            \end{naloga}

    
\end{priprava}