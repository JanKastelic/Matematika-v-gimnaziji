\begin{priprava}{1}{}{Realna števila}{Realna števila}{frontalna}{drsnice, projekcija, tabla}
    
    \section{Realna števila}

        

                Med poljubnima dvema racionalnima številoma $\frac{x}{y}, \frac{z}{w}\in\mathbb{Q}$ je vsaj še eno racionalno število
                 -- aritmetična sredina teh dveh števil $\frac{1}{2}\left(\frac{x}{y}+\frac{z}{w}\right)$.

            
                % $$\frac{x}{y}<\frac{z}{w},\ y,w\neq 0 \quad \Rightarrow \quad \frac{x}{y}<\frac{1}{2}\left(\frac{x}{y}+\frac{z}{w}\right)<\frac{z}{w}$$
            
                
            
                Med poljubnima racionalnima številoma je neskončno mnogo racionalnih števil in pravimo, da je množica $\mathbb{Q}$ \textbf{povsod gosta}. 
            
                ~
            
                Množici $\mathbb{Q}$ in $\mathbb{Z}$ imata enako moč -- sta števno neskončni ($m(\mathbb{Q})=m(\mathbb{Z})=\aleph_0$).
            
        

                ~
        
                \textbf{Iracionalna števila} $\mathbb{I}$ so vsi kvadratni koreni števil, ki niso popolni kvadrati, tretji koreni, ki niso popolni kubi, ..., 
                število $\pi$, Eulerjevo število $e$ ... 
            
                ~
            
                Množici racionalnih in iracionalnih števil sta disjunktni: $\mathbb{Q}\cap\mathbb{I}=\emptyset$.
            
                
                ~

                \textbf{Realna števila} so množica števil, ki jo dobimo kot unijo racionalnih in iracionalnih števil: $\mathbb{R}=\mathbb{Q}\cup\mathbb{I}$.
            

            
                Množica realnih števil je močnejša od množice racionalnih števil. Pravimo, da je (neštevno) neskončna.
            

             ~

                Množico realnih števil lahko, glede na predznak števil, razdelimo na tri množice:
                \begin{itemize}
                    \item \textcolor{green}{množico negativnih realnih števil $\mathbf{\mathbb{R}^-}$},
                    \item množico z elementom nič: $\mathbf{\{0\}}$ in
                    \item \textcolor{red}{množico pozitivnih realnih števil: $\mathbf{\mathbb{R}^+}$}.
                \end{itemize}
                $$ \mathbb{R}=\textcolor{green}{\mathbb{R}^-}\cup\{0\}\cup\textcolor{red}{\mathbb{R}^+} $$
            
                \vskip-1em
                \begin{figure}[H]
                \centering
                \begin{tikzpicture}
                    % \clip (0,0) rectangle (14.000000,10.000000);
                    {\footnotesize
                    
                    % Drawing segment A B
                    \draw [line width=0.016cm] (1.000000,1.500000) -- (4.460000,1.500000);%
                    \draw [line width=0.016cm] (4.540000,1.500000) -- (8.000000,1.500000);%
                    
                    % Marking point 0 by circle
                    \draw [line width=0.016cm] (4.500000,1.500000) circle (0.040000);%
                    \draw (4.500000,1.500000) node [anchor=south] { $0$ };%
                    
                    
                    % Changing color 255 0 0
                    \definecolor{r255g0b0}{rgb}{1.000000,0.000000,0.000000}%
                    \color{r255g0b0}% 
                    
                    % Marking point \mathbb{Q}^+
                    \draw (6.250000,1.500000) node [anchor=south] { $\mathbb{R}^+$ };%
                    
                    % Drawing segment B 0
                    \draw [line width=0.016cm] (8.000000,1.500000) -- (4.540000,1.500000);%
                    }

                    
                    % Changing color 0 255 0
                    \definecolor{r0g255b0}{rgb}{0.000000,1.000000,0.000000}%
                    \color{r0g255b0}% 
                    
                    % Marking point \mathbb{Q}^-
                    \draw (2.750000,1.500000) node [anchor=south] { $\mathbb{R}^-$ };%
                    
                    % Drawing segment A 0
                    \draw [line width=0.016cm] (1.000000,1.500000) -- (4.460000,1.500000);%
                    

                    % Changing color 0 0 0
                    \definecolor{r0g0b0}{rgb}{0.000000,0.000000,0.000000}%
                    \color{r0g0b0}% 
                    
                    % Marking point \mathbb{Q}
                    \draw (1.500000,2.000000) node  { $\mathbb{R}$ };%
                    \color{black}
                    
                    \end{tikzpicture}
                    
                \end{figure}
            

            
            
                Vsaki točki na številski premici ustreza natanko eno realno število in obratno, 
                vsakemu realnemu številu ustreza natanko ena točka na številski premici.
            

            
                Številsko premico, ki upodablja realna števila, imenujemo tudi \textbf{realna os}.
            


        
                ~~        
            
                Z relacijo \textit{biti manjši ali enak} je množica $\mathbb{R}$ \textbf{linearno urejena}, 
                to pomeni, da veljajo:

                \begin{itemize}
                    \item \textbf{refleksivnost}: $\forall x\in\mathbb{R}: x\leq x$;
                    \item \textbf{antisimetričnost}: $\forall x,y\in\mathbb{R}: x\leq y \land y\leq x \Rightarrow x=y$;
                    \item \textbf{tranzitivnost}: $\forall x,y,z\in\mathbb{R}: x\leq y \land y\leq z \Rightarrow x\leq z$;
                    \item \textbf{stroga sovisnost}: $\forall x,y\in\mathbb{R}: x\leq y \lor y\leq x$.
                \end{itemize}
            
                ~

            
                Za realcijo urejenosti na množici $\mathbb{R}$ veljajo še naslednje lastnosti:

                \begin{itemize}
                    \item \textbf{monotonost vsote}: $x<y \Rightarrow x+z<y+z$ oziroma $x\leq y \Rightarrow x+z\leq y+z$;
                    \item $x<y \land z>0 \Rightarrow xz<yz$ in $x\leq y \land z>0 \Rightarrow x z\leq y z$;
                    \item $x<y \land z<0 \Rightarrow xz>yz$ in $x\leq y \land z<0 \Rightarrow x z\geq y z$.
                \end{itemize}

            

        

    
\end{priprava}