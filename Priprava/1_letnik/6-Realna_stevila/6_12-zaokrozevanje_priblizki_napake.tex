\begin{priprava}{26}{}{Zaokroževanje, približki, napake}{Realna števila}{frontalna}{drsnice, projekcija, tabla}
    
    \section{Zaokroževanje, približki, napake}

    
    \subsubsection*{Pravila zaokroževanja}
        \begin{itemize}
            \item Zadnjo števko pustimo enako, če je prva izbrisana števka manjša od $5$;
            \item zadnjo števko povečamo za $1$, če je prva izbrisana števka $5$ ali več.
        \end{itemize}
    

        ~~

    
        Zaokroževanje na \textbf{$n$ decimalnih mest} pomeni: 
        opustiti vse decimalke od $n$-tega mesta dalje in zaokrožiti.
        Primer: $\sqrt{2}\doteq 1.41$ (na $2$ decimalni mesti).
    
        ~~
    
        Zaokroževanje na \textbf{$n$ mest} pomeni, 
        da ima število v svojem zapisu $n$ števk, 
        pri pogoju, da ne štejemo ničel na začetku in na koncu.
        Primer: $\sqrt{2}\doteq 1.41$ (na $3$ mesta).
    
        ~
    
        Pri zapisu uporabimo $\doteq$, kar označuje, da smo rezultat zapisali približno in ne natančno.
    




    \subsubsection*{Absolutna in relativna napaka}
        Naj bo $x$ točna vrednost in $X$ njen \textbf{približek}.

        \textbf{Absolutna napaka} približka je $\left\lvert X-x\right\rvert$; 
        \textbf{relativna napaka} je $\dfrac{\left\lvert X-x\right\rvert}{x}$.
    
        ~~
    
        Absolutno napako zapišemo tudi $X=x\pm\epsilon$, kar pomeni, da se absolutna napaka približka $X$ razlikuje od točne vrednosti $x$ kvečjemu za $\epsilon$.
    

~~\\

%%%%%% naloge


    \begin{naloga}
        Na kartonski škatli je oznaka velikosti $50 \pm 3 ~cm$.
        Koliko je največja in koliko najmanjša velikost škatle, ki ustreza tej oznaki? 
        Ali je lahko relativna napaka velikosti $8~\%$?            
    \end{naloga}
    
    \begin{naloga}
        Pri $200~m$ vrvi smemo narediti $7~\%$ napako.
        Ali je lahko takšna vrv dolga $230~m$?
        Kako dolgi bosta najkrajša in najdaljša vrv, ki še ustrezata?            
    \end{naloga}
    
    \begin{naloga}
        V EU morajo biti banane za prodajo dolge najmanj $14~cm$. 
        V trgovino dobijo novo pošiljko banan, ki jih izmerijo, da so dolžine $13.7~cm$. 
        Njihov meter ima $5~\%$ odstopanje. 
        Ali lahko prodajajo takšne banane?            
    \end{naloga}



\end{priprava}