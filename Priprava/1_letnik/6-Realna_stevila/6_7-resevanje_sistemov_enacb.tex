\begin{priprava}{14., 15., 16}{}{Reševanje sistemov enačb}{Realna števila}{frontalna}{drsnice, projekcija, tabla}
    
    \section{Reševanje sistemov enačb}

        

            \subsection{Sistem dveh linearnih enačb z dvema neznankama}

                \textbf{Sistem dveh linearnih enačb z dvema neznankama} ali \textbf{sistem $\mathbf{2\times 2}$} je v splošnem oblike:
                    $$\begin{aligned}
                            a_1x+b_1y&=c_1 \\ a_2x+b_2y&=c_2
                        \end{aligned}$$
                $x$ in $y$ sta \textbf{neznanki}, $a_i,b_i,c_i\in\mathbb{R}$ so \textbf{koeficienti}.
            
~
            
                \textbf{Rešitev sistema} je \textbf{urejen par} števil $(x,y)$, ki zadoščajo obema enačbama.
            

            
                    Sistem $2\times 2$ ima lahko eno rešitev, nima rešitve ali ima neskončno rešitev.
            

        

        
            
                Sistem lahko rešujemo s primerjalnim načinom, zamenjalnim načinom ali z metodo nasprotnih koeficientov.
            
        
            \subsubsection*{Primerjalni način}
                Iz obeh enačb izrazimo isto neznanko, nato njuni vrednosti enačimo.
            

            \subsubsection*{Zamenjalni način}
                Iz ene enačbe izrazimo eno izmed neznank (preverimo, če je kateri od koeficientov pri neznankah enak $1$ -- takšno neznanko hitro izrazimo) in izraženo vrednost vstavimo v drugo enačbo.
            

            \subsubsection*{Metoda nasprotnih koeficientov}
                Eno ali obe enačbi pomnožimo s takimi števili, da bosta pri eni izmed neznank koeficienta nasprotni števili, nato enačbi seštejemo.
                Ostane ena enačba z eno neznanko.
            

~~~\\
        

        %%% naloge

        
            \begin{naloga}
                Rešite sisteme enačb.
                \begin{itemize}
                    
                        \item $\begin{aligned}
                            2x+y&=9 \\ x-3y&=8
                        \end{aligned}$ 
                        \item $\begin{aligned}
                            x-y&=5 \\ y-x&=3
                        \end{aligned}$ 
                        \item $\begin{aligned}
                            2x-3y&=5 \\ -4x+6y&=-10
                        \end{aligned}$ 
                        \item $\begin{aligned}
                            3x-y&=5 \\ 6x-10&=2y
                        \end{aligned}$ 
                    

                \end{itemize}
            \end{naloga}
        


        
            \begin{naloga}
                Z zamenjalnim načinom rešite sisteme enačb.
                \begin{itemize}
                    
                        \item $\begin{aligned}
                            2x+5y&=-2 \\ x-3y&=-1
                        \end{aligned}$ 
                        \item $\begin{aligned}
                            \frac{x}{2}-y&=3 \\ y+x&=-2
                        \end{aligned}$ 
                        \item $\begin{aligned}
                            3x-2y&=1 \\ x+y&=\frac{7}{6}
                        \end{aligned}$ 
                        \item $\begin{aligned}
                            0.5x+0.2y&=2 \\ \frac{3}{2}x-\frac{2}{5}y&=1
                        \end{aligned}$ 
                    

                \end{itemize}
            \end{naloga}
        


        
            \begin{naloga}
                Z metodo nasprotnih koeficientov rešite sisteme enačb.
                \begin{itemize}
                    
                        \item $\begin{aligned}
                            2x+3y&=3 \\ -4x+3y&=0
                        \end{aligned}$ 
                        \item $\begin{aligned}
                            4x-3y&=-2 \\ -8x+y&=-1
                        \end{aligned}$ 
                        \item $\begin{aligned}
                            3x-2y&=2 \\ 2x-3y&=-2
                        \end{aligned}$ 
                        \item $\begin{aligned}
                            x-y&=-5 \\ 0.6x+0.4y&=7
                        \end{aligned}$ 
                    

                \end{itemize}
            \end{naloga}
        


        
            \begin{naloga}
                V bloku je $26$ stanovanj. Vsako stanovanje ima $2$ ali $3$ sobe. Koliko je posameznih vrst stanovanj, če je v bloku $61$ sob?
                
            \end{naloga}

            \begin{naloga}
                Kmet ima v ogradi $20$  živali. Če so v ogradi le race in koze, koliko je posameznih živali, če smo našteli $50$ nog? 
                
            \end{naloga}

        


        
            \begin{naloga}
                Razredničarka na sladoled pelje svojih $30$ dijakov. Naročili so lahko $2$ ali $3$ kepice sladoleda. Koliko dijakov je naročilo dve in koliko tri kepice sladoleda,
                če razredničarka ni jedla sladoleda, plačala pa je $79$ kepic sladoleda?
                
            \end{naloga}

            \begin{naloga}
                Babica ima dvakrat toliko vnukinj kot vnukov. Vnukinjam je podarila po tri bombone, vnukom pa po štiri bombone.
                Koliko vnukinj in vnukov ima, če je podarila $70$ bombonov?
                
            \end{naloga}

        


~~~\\
      
            
            \subsection{Sistem treh linearnih enačb s tremi neznankami}

                \textbf{Sistem treh linearnih enačb z tremi neznankami} ali \textbf{sistem $\mathbf{3\times 3}$} je v splošnem oblike:
                    $$\begin{aligned}
                            a_1x+b_1y+c_1z&=d_1 \\ a_2x+b_2y+c_2z&=d_2 \\ a_3x+b_3y+c_3z&=d_3
                        \end{aligned}$$
                $x$, $y$ in $z$ so \textbf{neznanke}, $a_i,b_i,c_i\in\mathbb{R}$ so \textbf{koeficienti}.
            
~
            
                \textbf{Rešitev sistema} je \textbf{urejena trojka} števil $(x,y,z)$, ki zadoščajo vsem trem enačbam.
            

            
                    Sistem $3\times 3$ rečujemo z istimi postopki kot sisteme $2\times 2$, le da postopek ponovimo večkrat.
            

        
~~~\\


        
            \begin{naloga}
                Z metodo nasprotnih koeficientov rešite sisteme enačb.
                \begin{itemize}
                    
                        \item $\begin{aligned}
                            2x+y-3z&=5 \\ x+2y+2z&=1 \\ -x+y+z&=-4
                        \end{aligned}$ 
                        \item $\begin{aligned}
                            x-2y+6z&=5 \\ -x+3z&=-1 \\ 4y-3z&=-3
                        \end{aligned}$ 
                        \item $\begin{aligned}
                            x+y-z&=0 \\ x-y-3z&=2 \\ 2x+y-3z&=1
                        \end{aligned}$ 
                        \item $\begin{aligned}
                            2x-4y+z&=3 \\ 4x-y+2z&=4 \\ -8x+2y-4z&=7
                        \end{aligned}$ 
                    

                \end{itemize}
            \end{naloga}
        

    
\end{priprava}