\begin{priprava}{18}{}{Sklepni račun}{Realna števila}{frontalna}{drsnice, projekcija, tabla}
    
    \section{Sklepni račun}

    

        
            Pri sklepnem računu obravnavamo situacije, v katerih nastopata dve količini,
            ki sta premo sorazmerni ali obratno sorazmerni.
        

        \subsubsection*{Premo sorazmerje}
        Količini $x$ in $y$ sta \textbf{premo sorazmerni}, če obstaja takšno neničelno število $k\in\mathbb{R}^*$, da je $x=k\cdot y$.
        

        \subsubsection*{Obratno sorazmerje}
        Količini $x$ in $y$ sta \textbf{obratno sorazmerni}, če obstaja takšno neničelno število $k\in\mathbb{R}^*$, da je $x=\dfrac{k}{y}$.
        

    
~~~~\\

    %%% naloge

    
        \begin{naloga}
            Delavec v štirih urah zasluži $10~€$. Koliko zasluži v dvanajstih urah?            
        \end{naloga}

        \begin{naloga}
            Tiskalnik v sedmih minutah natisne $42$ strani. Koliko časa potrebuje za $108$ strani?            
        \end{naloga}

        \begin{naloga}
            Tri čebele v treh dneh oprašijo devetsto cvetov. Koliko cvetov v šestih dneh opraši šest čebel?            
        \end{naloga}
 
        \begin{naloga}
            Kolesar od Ljubljane do Geometrijskega središča Slovenije potuje dve uri s hitrostjo $20~km/h$. 
            Kako hitro bi moral peljati, da bi pot prevozil v eni uri in petnajstih minutah?            
        \end{naloga}
        
        \begin{naloga}
            En računalnik za pripravo posebnih efektov filma potrebuje $14$ ur.
            Koliko časa bi potrebovali trije taki računalniki, za pripravo posebnih efektov za šest filmov?            
        \end{naloga}

        \begin{naloga}
            Sedem pleskarjev pleska hišo $15$ dni. Po petih dneh dva delavca premestijo na drugo delovišče.
            Koliko časa bodo preostali delavci pleskali hišo?            
        \end{naloga}

    

    
\end{priprava}