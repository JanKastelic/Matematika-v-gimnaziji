\begin{priprava}{19., 20., 21}{}{Odstotni račun}{Realna števila}{frontalna}{drsnice, projekcija, tabla}
    
    \section{Odstotni račun}  
        
    Količine pri odstotnem računu so povezane s sklepnim računim, in sicer so v premem sorazmerju.

~\\
    \textbf{Odstotek} (ali procent) $\text{\%}$ celote definiramo kot stotino celote,
    \textbf{odtisoček} (ali promil) $\permil$ kot tisočino celote.

    $$ 1~\text{\%}=\dfrac{1}{100} \quad \quad {1~\permil=\dfrac{1}{1000}}$$



    \textbf{Relativni delež} je kvocient med deležem in osnovo: $r=\dfrac{d}{o}$.



~~~\\
%%%%%%% naloge

\begin{multicols}{2}
\begin{naloga}
    Zapišite z okrajšanim ulomkom oziroma odstotkom.
    \begin{multicols}{2}
    \begin{itemize}
            \item $12~\%$ 
            \item $20~\%~a$ 
            \item $250~\%$ 
            \item $0.5~\%~b$ 
            \item $12~\permil$ 
            \item $\frac{3}{4}a$ 
            \item $\frac{7}{20}x$ 
            \item $\frac{31}{10}y$ 
            \item $0.8 z$ 
            \item $\frac{25}{8}m$
    \end{itemize} 
\end{multicols}
\end{naloga}




\begin{naloga}
    Izračunajte.
    \begin{itemize}
            \item Koliko je $20~\%$ od $10~kg$? 
            \item Koliko je $25~\%$ od $20~€$? 
            \item Koliko je $10~\%$ od $1~l$? 
            \item Koliko je $250~\%$ od $12~g$? 
            \item Koliko je $1~\permil$ od $2350~kg$? 
            \item Koliko je $17~\permil$ od $100~m$? 
    \end{itemize} ~
\end{naloga}

\end{multicols}
    
    
        \begin{naloga}
            Pri ekološki pridelavi kmet pridela $3$ tone pšenice na hektar. 
            Zaradi toče je bil letošnji pridelek le $2450~kg$ pšenice.
            Za koliko odstotkov se je zmanjšala količina pridelka zaradi toče? 
        \end{naloga}

        \begin{naloga}
            V $5~kg$ raztopine je $120~g$ soli. Koliko odstotna je ta raztopina? 
        \end{naloga}

        \begin{naloga}
            V tovarni čevljev so povečali proizvodnjo s $1250$ parov tedensko na $1700$ parov.
            Koliko odstotno je to povečanje? 
        \end{naloga}
    

    
        \begin{naloga}
            Kokoši nesnice znesejo $270$ jajc letno. 
            Koliko odstotna je njihova nesnost? 
        \end{naloga}

        \begin{naloga}
            V trgovini stane izdelek $120~€$. Koliko stane po:
            \begin{itemize}
                \item $5~\%$ podražitvi,
                \item $20~\%$ pocenitvi?
            \end{itemize}
        \end{naloga}

        \begin{naloga}
            Jošt je natipkal besedilo na list A4 formata v pisava Arial, velikosti $12$, in ugotovil, da je bilo na strani $3150$ znakov s presledki.
            Če bi pisavo zmanjšal na velikost $10$, bi na stran prišlo $28~\%$ več znakov. Koliko? 
        \end{naloga}
    

    
    
        \begin{naloga}
            Dizelsko gorivo je stalo v Sloveniji $1.421~€$, v Italijo $1.748~€$, v Avstriji pa $1.751~€$.
            Za koliko odstotkov je bilo gorivo v Italiji dražje od goriva v naši državi in za koliko odstotkov je bilo
            naše gorivo cenejše od goriva v Avstriji? 
        \end{naloga}

        \begin{naloga}
            Prenočitvene zmogljivosti na Bledu so $8880$ ležišč. Pred prvomajskimi prazniki so se turistični delavci pohvalili,
            da je zasedenost kapacitet $90~\%$. Koliko turistov še lahko sprejmejo na nočitev? 
        \end{naloga}

        \begin{naloga}
            Maksov avto porabi $5.6~l$ goriva na prevoženih $100~km$. 
            Z varčno vožnjo lahko zniža porabo za $15~\%$.
            Koliko kilometrov bo tako prevozil s polnim rezervoarjem, ki drži $55~l$. 
        \end{naloga}
    


    
        \begin{naloga}
            Kavču, ki je stal $652~€$, so ceno znižali za $10~\%$, na sejmu pa so ponudili na to ceno še $12~\%$ sejemskega popusta.
            Koliko bomo odšteli za kavč na sejmu? Za koliko odstotkov je cena na sejmu nižja od prvotne cene kavča? 
        \end{naloga}

        
        \begin{naloga}
            Servis so najprej podražili za $10~\%$, potem pa se je ena skodelica okrušila in so ga pocenili za $30~\%$.
            Koliko je servis stal na začetku, če ga danes lahko kupimo za $115.5~€$? 
        \end{naloga}

        \begin{naloga}
            Izdelek je imel napako, zato so ga pocenili za $20~\%$. Ko so napako skoraj v celoti odpravili, so ga podražili za $20~\%$.
            Kolikšna je bila začetna cena izdelka, če po popravilu stane $192~€$? 
        \end{naloga}

    


    
        \begin{naloga}
            Koliko vode moramo priliti $3~kg$ $45~\%$ raztopine, da bomo koncentracijo znižali na $20~\%$? 
        \end{naloga}

        \begin{naloga}
            Kolikšno koncentracijo raztopine dobimo, če zmešamo $2~kg$ $60~\%$ raztopine in $3~kg$ $40~\%$ raztopine? 
        \end{naloga}

        \begin{naloga}
            Koliko $kg$ $12~\%$ raztopine moramo priliti k $30~kg$ $24~\%$ raztopine, da bomo dobili raztopino z $20~\%$ koncentracijo? 
        \end{naloga}
    


    
\end{priprava}