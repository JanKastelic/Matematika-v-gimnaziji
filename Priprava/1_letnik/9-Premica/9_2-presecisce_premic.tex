\begin{priprava}{4., 5., 6}{}{Enačba premice}{Premica}{frontalna}{drsnice, projekcija, tabla}


\section{Presečišče premic}

                Dve premici v ravnini se lahko \textbf{sekata} ali sta \textbf{vzporedni}. \\ 
                Glede na to dobimo različne rešitve sistemov dveh linearnih enačb z dvema neznankama.
                $$\begin{aligned}
                    a_1x+b_1y+c_1&=0 \\ a_2x+b_2y+c_2&=0
                \end{aligned}$$
                
                \begin{itemize}
                    \item Če se premici sekata, dobimo kot rešitev sistema urejen par $(x,y)$ oziroma točko $T(x,y)$, v kateri se sekata.
                    \item Če sta premici vzporedni imamo dve možnosti:
                    \begin{itemize}
                        \item sistem ima neskončno mnogo (premico) rešitev, če premici sovpadata (sta identični),
                        \item sistem nima rešitve, če sta premici različni.
                    \end{itemize}
                \end{itemize}

~\\~\\
    %%%% naloge

            
            \begin{naloga}
                Izračunajte presečišče premic, rezultat preverite s sliko.
                \begin{multicols}{2}   
                \begin{itemize}
                        \item $\begin{aligned}
                            2x-3x-3&=0 \\ x&=3
                        \end{aligned}$ 
                        \item $\begin{aligned}
                            y&=3x+3 \\ y&=\dfrac{x}{2}+3
                        \end{aligned}$ 
                        \item $\begin{aligned}
                            x+3y-9&=0 \\ x-3y-3&=0
                        \end{aligned}$ 
                        \item $\begin{aligned}
                            \dfrac{x}{3}-\dfrac{y}{6}&=1 \\ \dfrac{x}{2}+\dfrac{y}{5}&=1
                        \end{aligned}$ 

                \end{itemize}
            \end{multicols}
            \end{naloga}

            
            \begin{naloga}
                Zapišite enačbo premice, ki gre skozi presečišče premic $y=2x+1$ in $y=-\frac{1}{2}x+6$ ter seka ordinatno os pri $y=4$.
            \end{naloga}

            \begin{naloga}
                Zapišite enačbo premice, ki gre skozi presečišče premic $y=3x+1$ in $y=-x+5$ ter ima smerni koeficient $k=2$.
            \end{naloga}

            \begin{naloga}
                Zapišite implicitno enačbo premice, ki gre skozi presečišče premic $2x-y-13=0$ in $2x+3y-1=0$ ter seka abscisno os pri $x=\frac{7}{2}$.
            \end{naloga}
        
            
            \begin{naloga}
                Zapišite enačbo premice, ki gre skozi presečišče premic $3x+4y-11=0$ in $2x-7y+41=0$ ter je vzporedna ordinatni osi.
            \end{naloga}

            \begin{naloga}
                Zapišite eksplicitno enačbo premice, ki gre skozi presečišče premic $5x-7y+3=0$ in $2x+y-14=0$ ter je vzporedna premici z enačbo $3x-2y+1=0$.
            \end{naloga}

            \begin{naloga}
                Izračunajte smerni koeficient $k$ tako, da se premici z enačbama $y=2x+6$ in $y=kx+\frac{5}{2}$ sekata na simetrali sodih kvadrantov.
            \end{naloga}
            
            \begin{naloga}
                Stranice trikotnika ležijo na premicah z enačbami $x+y=0$, $3x-2y=0$ in $x-4y+10=0$. 
                Izračunajte oglišča trikotnika ter njegov obseg in ploščino.
                Premice in trikotnik narišite v pravokotnem koordinatnem sistemu.
            \end{naloga}

            \begin{naloga}
                Dani sta dve oglišči $A$ in $B$ trikotnika $\triangle ABC$, orientacija in ploščina. Izračunajte kooridnati tretjega oglišča $C$, če leži na dani premici.
                \begin{itemize}
                    \item $A(-6,1)$, $B(2,-1)$; \\ pozitivna orientacija, $S=25$; \\ $C$ leži na $y=-2x+4$
                    \item $A(-4,0)$, $B(4,2)$; \\ pozitivna orientacija, $S=7$; \\ $C$ leži na $y=5-2x$
                \end{itemize}
            \end{naloga}



\end{priprava}