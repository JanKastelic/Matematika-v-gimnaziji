\begin{priprava}{1}{}{Ulomki in racionalna števila}{Racionalna števila}{frontalna}{drsnice, projekcija, tabla}

    \section{Ulomki in racionalna števila}

        
            
                \textbf{Ulomek} $\dfrac{x}{y}$ je zapis, ki predstavlja zapis deljenja 
                $$x:y=\dfrac{x}{y};\quad y\neq 0\land x,y\in\mathbb{Z}.$$
                Število/izraz $x$ imenujemo \textbf{števec}, $y$ pa \textbf{imenovalec}, med njima je \textbf{ulomkova črta}.
            
                ~
            
                Ulomek $\dfrac{x}{0}$ ni definiran (nima pomena), saj z $0$ ne moremo deliti.
            
                ~
            
                \textbf{Algebrski ulomek} je ulomek, v katerem v števcu in/ali imenovalcu nastopajo algebrski izrazi.
            
                ~
            
                Vsako celo število $x\in\mathbb{Z}$ lahko zapišemo z ulomkom: $x=\dfrac{x}{1}$.
            
                ~
            
                \textbf{Ničelni ulomek} je ulomek oblike $\dfrac{0}{y}=0; y\neq 0$.
            
                ~
            
                V ulomku, kjer v števcu ali imenovalcu nastopa negativno število, upoštevamo enakost 
                $$-\dfrac{x}{y}=\dfrac{-x}{y}=\dfrac{x}{-y}.$$
            
                ~
            
                Vsakemu neničelnemu ulomku $\dfrac{x}{y}$ lahko priredimo njegovo \textbf{obratno vrednost}:
                $$\left(\dfrac{x}{y}\right)^{-1}=\dfrac{y}{x}; \quad x,y\in\mathbb{Z}\setminus\{0\}.$$
            

        


        
            \subsection*{Racionalna števila}

            
                Množica racionalnih števil $\mathbb{Q}$ je sestavljena iz vseh ulomkov (kar pomeni, da vsebuje tudi vsa naravna in cela števila).

                $$\mathbb{Q}=\left\{\frac{x}{y}; \ x\in\mathbb{Z}, y\in\mathbb{Z}\setminus\{0\}\right\}$$
            
                \begin{figure}[H]
                \centering
                \begin{tikzpicture}
                    % \clip (0,0) rectangle (14.000000,10.000000);
                    {\footnotesize
                    
                    % Drawing segment A B
                    \draw [line width=0.016cm] (1.000000,1.500000) -- (4.460000,1.500000);%
                    \draw [line width=0.016cm] (4.540000,1.500000) -- (8.000000,1.500000);%
                    
                    % Marking point 0 by circle
                    \draw [line width=0.016cm] (4.500000,1.500000) circle (0.040000);%
                    \draw (4.500000,1.500000) node [anchor=south] { $0$ };%
                    
                    
                    % Changing color 255 0 0
                    \definecolor{r255g0b0}{rgb}{1.000000,0.000000,0.000000}%
                    \color{r255g0b0}% 
                    
                    % Marking point \mathbb{Q}^+
                    \draw (6.250000,1.500000) node [anchor=south] { $\mathbb{Q}^+$ };%
                    
                    % Drawing segment B 0
                    \draw [line width=0.016cm] (8.000000,1.500000) -- (4.540000,1.500000);%
                    }

                    
                    % Changing color 0 255 0
                    \definecolor{r0g255b0}{rgb}{0.000000,1.000000,0.000000}%
                    \color{r0g255b0}% 
                    
                    % Marking point \mathbb{Q}^-
                    \draw (2.750000,1.500000) node [anchor=south] { $\mathbb{Q}^-$ };%
                    
                    % Drawing segment A 0
                    \draw [line width=0.016cm] (1.000000,1.500000) -- (4.460000,1.500000);%
                    

                    % Changing color 0 0 0
                    \definecolor{r0g0b0}{rgb}{0.000000,0.000000,0.000000}%
                    \color{r0g0b0}% 
                    
                    % Marking point \mathbb{Q}
                    \draw (1.500000,2.000000) node  { $\mathbb{Q}$ };%
                    \color{black}
                    
                    \end{tikzpicture}
                \end{figure}
                    
            

            
                Glede na predznak razdelimo racionalna števila v tri množice:
                \begin{itemize}
                    \item \textcolor{green}{množico negativnih racionalnih števil $\mathbf{\mathbb{Q}^-}$},
                    \item množico z elementom nič: $\mathbf{\{0\}}$ in
                    \item \textcolor{red}{množico pozitivnih racionalnih števil: $\mathbf{\mathbb{Q}^+}$}.
                \end{itemize}
                $$ \mathbb{Q}=\textcolor{green}{\mathbb{Q}^-}\cup\{0\}\cup\textcolor{red}{\mathbb{Q}^+} $$
            
            

            % 
            %     Množica racionalnih števil je povsod gosta, saj lahko med poljubnima racionalnima številoma vedno najdemo racionalno število (posledično je med dvema racionalnima številoma neskončno mnogo racionalnih števil).
            % 

        

        
            
                Ulomka $\dfrac{x}{y}$ in $\dfrac{w}{z}$ sta enaka/enakovredna natanko takrat, ko je $xz=wy$; $y,z\neq 0$.
                $$\dfrac{x}{y}=\dfrac{w}{z}\Leftrightarrow xz=wy; \quad y,z\neq 0$$
            

            
                Enaka/enakovredna ulomka sta različna zapisa za isto racionalno število.
            
        ~\\



%%% naloge

        \begin{multicols}{2}

            \begin{naloga}
                Za katere vrednosti $x$ ulomek ni definiran?
                \begin{itemize}
                    \item $\frac{x-2}{x+1}$ 
                    \item $\frac{2}{x-5}$ 
                    \item $\frac{x+2}{3}$ 
                    \item $\frac{13}{2x-5}$ 
                \end{itemize}
            \end{naloga}
        

        
            \begin{naloga}
                Za katere vrednosti $x$ ima ulomek vrednost enako $0$?
                \begin{itemize}
                    \item $\frac{x-2}{x+1}$ 
                    \item $\frac{2}{x-5}$ 
                    \item $\frac{x+2}{3}$ 
                    \item $\frac{13}{2x-5}$ 
                \end{itemize}
            \end{naloga}
        

        
            \begin{naloga}
                Ali imata ulomka isto vrednost?
                \begin{itemize}
                    \item $\frac{2}{3}$ in $\frac{10}{15}$ 
                    \item $\frac{-1}{2}$ in $\frac{1}{-2}$ 
                    \item $\frac{4}{5}$ in $\frac{-8}{-10}$ 
                    \item $\frac{5}{8}$ in $\frac{8}{5}$ 
                \end{itemize}
            \end{naloga}
        

        
            \begin{naloga}
                Za kateri $x$ imata ulomka isto vrednost?
                \begin{itemize}
                    \item $\frac{x+1}{2}$ in $\frac{3}{4}$ 
                    \item $\frac{4}{2x-1}$ in $\frac{1}{3}$ 
                    \item $\frac{x+1}{2}$ in $\frac{x-1}{-3}$ 
                    \item $\frac{x+1}{x-2}$ in $\frac{2}{5}$ 
                \end{itemize}
            \end{naloga}
        

        
            \begin{naloga}
                Ali ulomka predstavljata isto vrednost?
                \begin{itemize}
                    \item $\left(\frac{1}{2}\right)^{-1}$ in $-\frac{1}{2}$ 
                    \item $\left(\frac{2}{3}\right)^{-1}$ in $\frac{3}{2}$ 
                    \item $ 1\frac{3}{7}$ in $\left(\frac{7}{10}\right)^{-1}$ 
               \end{itemize}
            \end{naloga}
        

        
            \begin{naloga}
                Ali ulomka predstavljata isto vrednost?
                \begin{itemize}
                    \item $ 2\cdot\frac{3}{4}$ in $\frac{3}{2}$ 
                    \item $ 2\frac{3}{4}$ in $\frac{3}{2}$ 
                    \item $\left(1\frac{2}{5}\right)^{-1}$ in $ 1\frac{5}{2}$ 
                    \item $\left(1\frac{2}{5}\right)^{-1}$ in $\frac{5}{7}$ 
               \end{itemize}
            \end{naloga}
        

        
            \begin{naloga}
                Zapišite s celim delom oziroma z ulomkom.
                \begin{itemize}
                            \item $\frac{14}{5}$ 
                            \item $-\frac{5}{2}$ 
                            \item $\frac{4}{3}$ 
                            \item $\frac{110}{17}$ 
                            \item $ 3\frac{5}{8}$ 
                            \item $ 2\frac{9}{2}$                  
               \end{itemize}
            \end{naloga}
            ~\\~~\\

        \end{multicols}
        


\end{priprava}