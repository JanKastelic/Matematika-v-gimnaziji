\begin{priprava}{2., 3}{}{Razširjanje in krajšanje ulomkov}{Racionalna števila}{frontalna}{drsnice, projekcija, tabla}

    \section{Razširjanje in krajšanje ulomkov}

        

            \subsection*{Razširjanje ulomka}
                Ulomek ohrani svojo vrednost, če števec in imenovalec pomnožimo z istim neničelnim številom oziroma izrazom.
                Temu postopku pravimo \textbf{razširjanje ulomka}.

                $$\dfrac{x}{y}=\dfrac{x\cdot z}{y\cdot z}; \quad x\in\mathbb{Z} \land y,z\in\mathbb{Z}\setminus\{0\}$$
            

            
                Ko ulomke seštevamo ali odštevamo, jih razširimo na \textbf{najmanjši skupni imenovalec}, 
                ki je najmanjši skupni večkratnik vseh imenovalcev.
            

        

        
            \subsection*{Krajšanje ulomka}
                Vrednost ulomka se ne spremeni, če števec in imenovalec delimo z istim neničelnim številom oziroma izrazom.
                Temu postopku rečemo \textbf{krajšanje ulomka}.

                $$\dfrac{x\cdot z}{y\cdot z}=\dfrac{x}{y}; \quad x\in\mathbb{Z}\land y,z\in\mathbb{Z}\setminus\{0\} $$
            
                ~

                Ulomek $\dfrac{x}{y}$ je \textbf{okrajšan}, če je $(x,y)=1$, torej če sta števec in imenovalec tuji števili.
            
        
            ~\\

        %%% naloge

        
            \begin{naloga}
                Razširite ulomke na najmanjši skupni imenovalec.
                \begin{itemize}
                            \item $\frac{1}{3}$, $\frac{3}{5}$ in $\frac{5}{6}$ 
                            \item $\frac{2}{7}$, $1$ in $\dfrac{1}{2}$ 
                            \item $\frac{5}{6}$, $\frac{1}{2}$ in $-\frac{2}{3}$ 
                            \item $\frac{1}{5}$, $-\frac{1}{2}$ in $\frac{-1}{3}$ 
                            \item $\frac{2}{-1}$, $\frac{3}{2}$ in $\frac{1}{-3}$ 
                            \item $\frac{3}{-4}$, $\frac{-1}{2}$ in $-\frac{2}{5}$ 
                \end{itemize}
            \end{naloga}
        

        
            \begin{naloga}
                Razširite ulomke na najmanjši skupni imenovalec.
                \begin{itemize}
                            \item $\frac{1}{x-1}$, $\frac{1}{x+1}$ in $1$ 
                            \item $\frac{2}{x}$, $\frac{1}{x-3}$ in $\frac{1}{(x-3)^2}$ 
                            \item $\frac{3}{x^2-4x}$, $\frac{1}{x}$ in $\frac{2}{x-4}$ 
                            \item $\frac{4}{x-4}$, $\frac{2}{x-2}$ in $\frac{1}{x^2-6x+8}$ 
                            \item $\frac{2}{x-1}$ in $\frac{3}{1-x}$ 
                            \item $\frac{1}{2-x}$, $\frac{2}{x+2}$ in $\frac{3}{x^2-4}$ 
                \end{itemize}
            \end{naloga}
        

        
            \begin{naloga}
                Okrajšajte ulomek.
                \begin{itemize}
                    \item $\frac{100}{225}$ 
                    \item $\frac{34}{51}$ 
                    \item $\frac{121}{3}$ 
                    \item $\frac{45}{75}$ 
                \end{itemize}
            \end{naloga}
        

        
            \begin{naloga}
                Okrajšajte ulomek.
                \begin{itemize}
                            \item $\frac{x^2-4}{x^2+2x}$ 
                            \item $\frac{x^3+8}{2x+4}$ 
                            \item $\frac{x^3-1}{x^2-4x+3}$ 
                            \item $\frac{x^3-2x^2-x+2}{x^2-3x+2}$ 
                            \item $\frac{x^2-9}{3-x}$ 
                            \item $\frac{x-4}{16-x^2}$ 
                \end{itemize}
            \end{naloga}
        


\end{priprava}