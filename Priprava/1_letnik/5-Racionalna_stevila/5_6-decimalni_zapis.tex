\begin{priprava}{11}{}{Decimalni zapis}{Racionalna števila}{frontalna}{drsnice, projekcija, tabla}

    \section{Decimalni zapis}

    Vsako racionalno število lahko zapišemo na dva načina:
    \begin{itemize}
        \item z \textbf{ulomkom} in 
        \item z \textbf{decimalnim zapisom}.
    \end{itemize}

    ~

    \textbf{Decimalni zapis} sestavljajo tri komponente:
    \begin{itemize}
        \item \textbf{celi del},
        \item \textbf{decimalna pika} oziroma \textbf{decimalna vejica} in 
        \item \textbf{ulomljeni del}.
    \end{itemize}

    ~

    Decimalni zapis racionalnega števila (zapisanega z ulomkom) dobimo tako, 
    da števec ulomka delimo z njegovim imenovalcem.



    \subsection*{Končen decimalni zapis}
    
    \textbf{Končen decimalni zapis} dobimo pri \textbf{desetiških}/\textbf{decimalnih ulomkih}. 
    
    To so ulomki, katerih imenovalec se lahko razširi na potenco števila $10$, takšni imenovalci so oblike $2^n\cdot 5^m$.

    

    \subsection*{Neskončen periodičen decimalni zapis}
    
    \textbf{Neskončen periodičen decimalni zapis} dobimo pri \textbf{nedesetiških}/\textbf{nedecimalnih ulomkih}. 
    
    To so ulomki, katerih imenovalca ne moremo razširiti na potenco števila $10$.

    ~

    Najmanjšo skupino števk, ki se pri neskončnem periodičnem decimalnem zapisu ponavlja, imenujemo \textbf{perioda}.
    Označujemo jo s črtico nad to skupino števk.

    Glede na število števk, ki v njej nastopajo, določimo njen \textbf{red}.

    

%%% naloge

    ~\\

    \begin{multicols}{2}
        

\begin{naloga}
    Zapišite z decimalnim zapisom.
    \begin{itemize}
                \item $\dfrac{3}{8}$ 
                \item $\dfrac{2}{125}$ 
                \item $\dfrac{6}{25}$ 
                \item $\dfrac{5}{6}$ 
                \item $\dfrac{4}{9}$ 
                \item $\dfrac{4}{15}$ 
                \item $\dfrac{1}{7}$ 
                \item $\dfrac{11}{13}$ 
   \end{itemize}
\end{naloga}



\begin{naloga}
    Periodično decimalno število zapišite z okrajšanim ulomkom.
    \begin{itemize}
                \item $0.\overline{24}$ 
                \item $0.\overline{9}$ 
                \item $1.\overline{2}$ 
                \item $1.0\overline{3}$ 
                \item $1.00\overline{12}$ 
    \end{itemize}
\end{naloga}




\begin{naloga}
    Izračunajte.
    \begin{itemize}
                \item $2.3+4.8$ 
                \item $11.3+2.35$ 
                \item $0.94+0.24$ 
                \item $5.6-2.9$ 
                \item $0.2-1.25$ 
                \item $12.5-20.61$ 
    \end{itemize}
\end{naloga}




\begin{naloga}
    Izračunajte.
    \begin{itemize}
                \item $0.1\cdot 2.44$ 
                \item $1.2\cdot 0.4$ 
                \item $11\cdot 0.002$ 
                \item $0.5\cdot 0.04$ 
                \item $0.3: 5$ 
                \item $12.5: 0.05$ 
                \item $2: 0.02$ 
                \item $0.15: 0.3$ 
    \end{itemize}
\end{naloga}




\begin{naloga}
    Izračunajte.
    \begin{itemize}
                \item $\left(0.24 + 0.06\right):5 - 1.2$ 
                \item $12:\left(1.2- 0.2\cdot 3\right)+1.2$ 
                \item $\left(2-0.3:\left(0.025 + 0.035\right)\right)\cdot 0.11$ 
                \item $\left(1-0.2:\left(0.03+0.02\right)\right)\cdot 1.5$ 
                \item $0.3\cdot\left(1.2-0.6\cdot\left(0.04+0.06\right)\right)$ 
    \end{itemize}
\end{naloga}

\end{multicols}

\end{priprava}