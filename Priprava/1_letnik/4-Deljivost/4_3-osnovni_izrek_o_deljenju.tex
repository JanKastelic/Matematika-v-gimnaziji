\begin{priprava}{4}{}{Osnovni izrek o deljenju}{Deljivost}{frontalna}{drsnice, projekcija, tabla}


    \section{Osnovni izrek o deljenju}

        

    \subsection*{Osnovni izrek o deljenju}
        Za poljubni naravni števili $\mathbf{m}$ (\textbf{deljenec}) in $\mathbf{n}$ (\textbf{delitelj}), $m\geq n$, 
        obstajata natanko določeni nenegativni števili $\mathbf{k}$ (\textbf{količnik}/\textbf{kvocient}) in $\mathbf{r}$ (\textbf{ostanek}), 
        da velja:
        $$m=k\cdot n+r; \quad  0\leq r<n; \quad m,n\in\mathbb{N}; k,r\in\mathbb{N}_0.$$
    

    
        Če je ostanek pri deljenju enak $0$, je število $m$ \textbf{večkratnik} števila $n$. 
        Tedaj je število $m$ deljivo s številom $n$. Pravimo, da $n$ deli število $m$: $n\mid m$.

~\\~

    \begin{naloga}
        Določite, katera števila so lahko ostanki pri deljenju naravnega števila $n$ s:
        \begin{itemize}
            \item številom $3$; 
            \item številom $7$; 
            \item številom $365$. 
        \end{itemize}
    \end{naloga}

    \begin{naloga}
        Zapišite prvih nekaj naravnih števil, ki dajo:
        \begin{itemize}
            \item pri deljenju s $4$ ostanek $3$; 
            \item pri deljenju s $7$ ostanek $4$; 
            \item pri deljenju z $9$ ostanek $4$. 
        \end{itemize}
    \end{naloga}

    \begin{naloga}
        Zapišite naravno število, ki da:
        \begin{itemize}
            \item pri deljenju s $7$ količnik $5$ in ostanek $3$; 
            \item pri deljenju z $10$ količnik $9$ in ostanek $1$; 
            \item pri deljenju s $23$ količnik $2$ in ostanek $22$. 
        \end{itemize}
    \end{naloga}

    \begin{naloga}
        Zapišite množico vseh naravnih števil $n$, ki dajo:
        \begin{itemize}
            \item pri deljenju z $2$ ostanek $1$; 
            \item pri deljenju z $2$ ostanek $0$; 
            \item pri deljenju s $5$ ostanek $2$. 
        \end{itemize}
    \end{naloga}



    \begin{naloga}
        Katero število smo delili s $7$, če smo dobili kvocient $3$ in ostanek $5$? 
    \end{naloga}

    \begin{naloga}
        S katerim številom smo delili število $73$, če smo dobili kvocient $12$ in ostanek $1$? 
    \end{naloga}

    \begin{naloga}
        Marjeta ima čebulice tulipana, ki jih želi posaditi v več vrst. 
        V vsaki od $3$ vrst je izkopala po $8$ jamic, potem pa ugotovila, da ji bosta $2$ čebulici ostali.
        Koliko čebulic ima Marjeta?  
    \end{naloga}



    \begin{naloga}
        Če neko število delimo z $8$, dobimo ostanek $7$. Kolikšen je ostanek, če to isto število delimo s $4$? 
    \end{naloga}

    \begin{naloga}
        Če neko število delimo s $24$ dobimo ostanek $21$. Kolikšen je ostanek, če to isto število delimo s $3$? 
    \end{naloga}
            
 


\end{priprava}