\begin{priprava}{2., 3}{}{Kriteriji deljivosti}{Deljivost}{frontalna}{drsnice, projekcija, tabla}


    \section{Kriteriji deljivosti}
    
    \subsubsection*{Deljivost z $2$}
        Število je deljivo z $2$ natanko takrat, ko so enice števila deljive z $2$.

    \subsubsection*{Deljivost s $3$}
        Število je deljivo s $3$ natanko takrat, ko je vsota števk števila deljiva s $3$.

    \subsubsection*{Deljivost s $4$ oziroma $25$}
        Število je deljivo s $4$ oziroma $25$ natanko takrat, ko je dvomestni konec števila deljiv s $4$ oziroma~$25$.

    \subsubsection*{Deljivost s $5$}
        Število je deljivo s $5$ natanko takrat, ko so enice števila enake $0$ ali $5$.

    \subsubsection*{Deljivost s $6$}
        Število je deljivo s $6$ natanko takrat, ko je deljivo z $2$ in s $3$ hkrati.

    \subsubsection*{Deljivost z $8$ oziroma s $125$}
        Število je deljivo z $8$ oziroma s $125$ natanko takrat, ko je trimestni konec števila deljiv z~$8$ oziroma s $125$.

    \subsubsection*{Deljivost z $9$}
        Število je deljivo z $9$ natanko takrat, ko je vsota števk števila deljiva z $9$.

    \subsubsection*{Deljivost z $10$ oziroma $10^n$}
        Število je deljivo z $10$ natanko takrat, ko so enice števila enake $0$.
        \\Število je deljivo z $10^n$ natanko takrat, ko ima število na zadnjih $n$ mestih števko $0$.

    \subsubsection*{Deljivost z $11$}
        Število je deljivo z $11$ natanko takrat, ko je alternirajoča vsota števk tega števila deljiva z $11$.

    \subsubsection*{Deljivost s $7$}
        Algoritem za preverjanje deljivosti s $7$:
        \begin{enumerate}
            \item vzamemo enice danega števila in jih pomnožimo s $5$,
            \item prvotnemu številu brez enic prištejemo dobljeni produkt,
            \item vzamemo enice dobljene vsote in jih pomnožimo s $5$,
            \item produkt prištejemo prej novo dobljenemu številu ...     
        \end{enumerate}
        Postopek ponavljamo, dokler ne dobimo dvomestnega števila -- 
        če je to deljivo s $7$, je prvotno število deljivo s $7$. 
        

~\\~

    \begin{naloga}
        S katerimi od števil $2$, $3$, $4$, $5$, $6$, $7$, $8$, $9$, $10$, $11$ so deljiva naslednja števila?
        \begin{itemize}
            \item $84742$ 
            \item $393948$ 
            \item $12390$ 
            \item $19401$ 
        \end{itemize}
    \end{naloga}



    \begin{naloga}
        Določite vse možnosti za števko $a$, da je število $\overline{65833a}$:
        \begin{itemize}
            \item deljivo s $3$, 
            \item deljivo s $4$, 
            \item deljivo s $5$, 
            \item deljivo s $6$. 
        \end{itemize}
    \end{naloga}



    \begin{naloga}
        Določite vse možnosti za števko $b$, da je število $\overline{65b90b}$:
        \begin{itemize}
            \item deljivo z $2$, 
            \item deljivo s $3$, 
            \item deljivo s $6$, 
            \item deljivo z $9$, 
            \item deljivo z $10$. 
        \end{itemize}
    \end{naloga}



    \begin{naloga}
        Določite vse možnosti za števki $c$ in $d$, da je število $\overline{115c1d}$ deljivo s $6$.
        
    \end{naloga}

    \begin{naloga}
        Določite vse možnosti za števki $e$ in $f$, da je število $\overline{115e1f}$ deljivo z $8$.
        
    \end{naloga}





    \begin{naloga}
        Pokažite, da za vsako naravno število $n$ $12$ deli $n^4-n^2$.
        
    \end{naloga}

    \begin{naloga}
        Preverite, ali je število $8641 969$ deljivo s $7$.
        
    \end{naloga}
            
 


\end{priprava}