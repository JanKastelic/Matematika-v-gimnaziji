\begin{priprava}{6}{}{Osnovni izrek aritmetike}{Deljivost}{frontalna}{drsnice, projekcija, tabla}


    \section{Osnovni izrek aritmetike}
    
    
        Vsako naravno število lahko enolično/na en sam način (do vrstnega reda faktorjev natančno) zapišemo kot produkt potenc s praštevilskimi osnovami:
        $$ n=p_1^{k_1}\cdot p_2^{k_2}\cdot\ldots\cdot p_l^{k_l};  p_i\in\mathbb{P}\land n, k_i\in\mathbb{N}.$$

    

        Zapis naravnega števila kot produkt potenc s praštevilskimi osnovami imenujemo tudi \textbf{praštevilski razcep}.
                

        ~\\
        \begin{naloga}
            Zapišite število $8755$ kot produkt samih praštevil in njihovih potenc. 
        \end{naloga}

        \begin{naloga}
            Razcepite število $3520$ na prafaktorje. 
        \end{naloga}

        \begin{naloga}
            Zapišite praštevilski razcep števila $38250$. 
        \end{naloga}

        \begin{naloga}
            Zapišite praštevilski razcep števila $3150$. 
        \end{naloga}
    
        \begin{naloga}
            Razcepite število $66$ na prafaktorje in zapišite vse njegove delitelje. 
        \end{naloga}

        \begin{naloga}
            Razcepite število $204$ na prafaktorje in zapišite vse njegove delitelje. 
        \end{naloga}
    
        \begin{naloga}
            Zapišite vse izraze, ki delijo dani izraz.
            \begin{itemize}
                \item $x^2+x-1$ 
                \item $x^3-x^2-4x+4$ 
                \item $x^3-27$ 
            \end{itemize}
        \end{naloga}
    
            
 


\end{priprava}