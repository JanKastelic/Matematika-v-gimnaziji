\begin{priprava}{5}{}{Praštevila in sestavljena števila}{Deljivost}{frontalna}{drsnice, projekcija, tabla}


    \section{Praštevila in sestavljena števila}
                
    Glede na število deliteljev, lahko naravna števila razdelimo na tri skupine:
    \begin{itemize}
        \item \textbf{število $1$} -- število, ki ima samo enega delitelja (samega sebe);
        \item \textbf{praštevila} -- števila, ki imajo natanko dva delitelja ($1$ in samega sebe);
        \item \textbf{sestavljena števila} -- števila, ki imajo več kot dva delitelja.
    \end{itemize}
    
    $$ \mathbb{N}=\{1\}\cup \mathbb{P}\cup \{sestavljena~števila\} $$


    Praštevil je neskončno mnogo.
    \\

    Število $n$ je praštevilo, če ni deljivo z nobenim praštevilom, manjšim ali enakim $\sqrt{n}$.
    \\


\textbf{Eratostenovo sito:}
    \begin{longtable}{|c|c|c|c|c|c|c|c|c|c|}
        \hline
        1 & 2 & 3 & 4 & 5 & 6 & 7 & 8 & 9 & 10 \\
        \hline
        11 & 12 & 13 & 14 & 15 & 16 & 17 & 18 & 19 & 20 \\
        \hline
        21 & 22 & 23 & 24 & 25 & 26 & 27 & 28 & 29 & 30 \\
        \hline
        31 & 32 & 33 & 34 & 35 & 36 & 37 & 38 & 39 & 40 \\
        \hline
        41 & 42 & 43 & 44 & 45 & 46 & 47 & 48 & 49 & 50 \\
        \hline
        51 & 52 & 53 & 54 & 55 & 56 & 57 & 58 & 59 & 60 \\
        \hline
        61 & 62 & 63 & 64 & 65 & 66 & 67 & 68 & 69 & 70 \\
        \hline
        71 & 72 & 73 & 74 & 75 & 76 & 77 & 78 & 79 & 80 \\
        \hline
        81 & 82 & 83 & 84 & 85 & 86 & 87 & 88 & 89 & 90 \\
        \hline
        91 & 92 & 93 & 94 & 95 & 96 & 97 & 98 & 99 & 100 \\
        \hline
        \end{longtable}
    

    ~\\~

\begin{naloga}
    Preverite, ali so števila $103, 163, 137, 197, 147, 559$ praštevila.
\end{naloga}
            
 


\end{priprava}