\begin{priprava}{7., 8}{}{Največji skupni delitelj in najmanjši skupni večkratnik}{Deljivost}{frontalna}{drsnice, projekcija, tabla}


    \section{Največji skupni delitelj in najmanjši skupni večkratnik}

        

    \textbf{Največji skupni delitelj} števil $m$ in $n$ je največje število od tistih, ki delijo števili $m$ in $n$. 
    Oznaka: $D(m,n)$.           
    \\

    \textbf{Najmanjši skupni večkratnik} števil $m$ in $n$ je najmanjše število od tistih, ki so deljiva s številoma $m$ in $n$. 
    Oznaka: $v(m,n)$.
    \\


    Števili $m$ in $n$, katerih največji skupni delitelj je $1$, sta \textbf{tuji števili}.
    \\



\textbf{Računanje $D$ in $v$ s prafaktorizacijo števil}
    \begin{itemize}
        \item Števili $m$ in $n$ prafaktoriziramo.
        \item Za $D(m,n)$ vzamemo potence, ki so skupne obema številom v prafaktorizaciji.
        \item Za $v(m,n)$ vzamemo vse potence, ki se pojavijo v prafaktorizaciji števil, z največjim eksponentom.
    \end{itemize}                
~


    Za poljubni naravni števili $m$ in $n$ velja zveza $\mathbf{D(m,n)\cdot v(m,n)=m\cdot n}$.
\\

\textbf{Evklidov algoritem}

    V tem algoritmu zapored uporabljamo osnovni izrek o deljenju. 
    \\ Najprej ga uporabimo na danih dveh številih.
    \\ V naslednjem koraku deljenec postane prejšnji delitelj, delitelj pa prejšnji ostanek. 
    \\ V vsakem koraku imamo manjša števila, zato se algoritem konča v končno mnogo korakih.
    \\ ~\\ Največji skupni delitelj danih števil $m$ in $n$ je zadnji od $0$ različen ostanek pri deljenju v Evklidovem algoritmu.




~\\

\begin{multicols}{2}

\begin{naloga}
    Izračunajte največji skupni delitelj in najmanjši skupni večkratnik danih parov števil.
    \begin{itemize}
        \item $6$ in $8$ 
        \item $36$ in $48$ 
        \item $550$ in $286$ 
        \item $6120$ in $4158$ 
    \end{itemize}
\end{naloga}

\begin{naloga}
    Preverite, ali sta števili $522$ in $4025$ tuji števili. 
\end{naloga}

\begin{naloga}
    Izračunajte največji skupni delitelj in najmanjši skupni večkratnik treh števil.
    \begin{itemize}
        \item $1320$, $6732$ in $297$ 
        \item $372$, $190$ in $11264$ 
    \end{itemize}
\end{naloga}

\begin{naloga}
    Z Evklidovim algoritmom izračunajte največji skupni delitelj parov števil.
    \begin{itemize}
        \item $754$ in $3146$ 
        \item $4446$ in $6325$ 
    \end{itemize}
\end{naloga}

\begin{naloga}
    Izračuanjte število $b$, če velja: $D(78 166, b)=418$ in $v(78 166, b)=1 485 154$. 
\end{naloga}

\begin{naloga}
    Določite največji skupni delitelj izrazov.
    \begin{itemize}
        \item $x^3-5x^2-24x$ in $x^2-64$ 
        \item $x^2+3x+10$, $x^3-4x$ in $x^3-8$ 
        \item $x^2-25$ in $x^3-27$ 
    \end{itemize}
\end{naloga}

\begin{naloga}
    Določite najmanjši skupni večkratnik izrazov.
    \begin{itemize}
        \item $x^2-64$ in $x+8$ 
        \item $x$, $8-x$ in $x^2-64$ 
        \item $x^2+3x-10$, $2x$ in $x^2+5x$ 
    \end{itemize}
\end{naloga}
~\\~\\
\end{multicols}

\begin{naloga}
    Velika Janezova terasa je dolga $1035~cm$ in široka $330~cm$. Janez bi jo rad sam tlakoval s kvadratnimi vinilnimi ploščami.
    Ker ni najbolj vešč tega dela, bo kupil tako velike plošče, da mu jih ne bo treba rezati.
    Koliko so največ lahko velik kvadratne plošče? Koliko plošč bo potreboval za tlakovanje? 
\end{naloga}

\begin{naloga}
    Neca gre v knjižnico vsake $14$ dni, Nace pa vsakih $10$ dni. V knjižnici se srečata v ponedeljek 1. marca.
    Čez koliko dni se bosta naslednjič srečala? Na kateri dan in datum?                     
\end{naloga}

        
            
 


\end{priprava}