\begin{priprava}{7., 8}{}{Moč množice}{Osnove logike in teorije množic}{frontalna}{drsnice, projekcija, tabla}


        
    \section{Moč množice}

    Število elementov v množici predstavlja \textbf{moč množice}.
    Oznaka: $\mathbf{m(\mathcal{A})}$ ali $\mathbf{|\mathcal{A}|}$.

    ~

    Množica je lahko:
    \begin{itemize}
        \item \textbf{končna množica} -- vsebuje končno mnogo elementov: $\mathbf{m(\mathcal{A})=n}$;
        \item \textbf{neskončna množica} -- vsebuje neskončno mnogo elementov: $\mathbf{m(\mathcal{A})=\infty}$.
    \end{itemize}

~

    Če ima množica toliko elementov, kot jih ima množica naravnih števil, je ta števno 
    neskončna.
    Njeno moč pišemo kot: $m(\mathcal{A})=\aleph_0$.

    ~

    Za množici, ki imata isto moč, rečemo, da sta \textbf{ekvipolentni} oziroma \textbf{ekvipotentni}.



~

\begin{naloga}
    Naštejte elemente množice in zapišite njeno moč, če je $\mathcal{U}=\mathbb{N}$.
    \begin{itemize}
        \item $\mathcal{A}=\{x; x\mid 24\}$
        \item $\mathcal{B}=\{x; 3<x\leq 7\}$
        \item $\mathcal{C}=\{x; x=4k\land k\in\mathbb{N}\land k\leq 5\}$
        \item $\mathcal{D}=\{x; x=3k+2\land k\in\mathbb{N}\land (4<k\leq 8)\}$
    \end{itemize}
\end{naloga}

\begin{naloga}
    Naj bo $\mathcal{U}=\mathbb{N}$. Zapišite množico tako, da naštejete njene elemente.
    Določite še njeno moč.
    \begin{itemize}
        \item Množica vseh deliteljev števila 18.
        \item Množica praštevil, ki so manjša od 20.
        \item Množica večkratnikov števila $5$, ki so večji od $50$ in manjši ali enaki $70$.
    \end{itemize}
\end{naloga}




\begin{naloga}
    Zapišite množico s simboli.
    \begin{itemize}
        \item Množica vseh sodih naravnih števil.
        \item Množica vseh naravnih števil, ki dajo pri deljenju s $7$ ostanek $5$.
    \end{itemize}
\end{naloga}

\begin{naloga}
    Podane so množice tako, da so našteti njihovi elementi. Množice zapišite s simboli.
    \begin{itemize}
        \item $\mathcal{A}=\{1,2,3,6\}$
        \item $\mathcal{B}=\{6, 12, 18, 24, 30\}$
        \item $\mathcal{C}=\{10, 12, 14, 16, 18, 20\}$
        \item $\mathcal{D}=\{2, 4, 8, 16, 32, 64, 128, 256, 512, 1024\}$
        \item $\mathcal{E}=\{1,4,9,16,25,36,49\}$
    \end{itemize}
\end{naloga}



\end{priprava}