\begin{priprava}{0}{}{}{Osnove logike in teorije množic}{}{}
    
    \chapter{Osnove logike in teorije množic}

    \Large{Pregled vsebine poglavja in predvidenega števila ur:}

    \begin{table}[H]
        \centering
        \begin{tabular}{||c|c||} 
        \hhline{|t:==:t|}
        \rowcolor[rgb]{0.843,0.718,0.718} 
        Učna enota  & Predvideno število ur   \\ 
        \hhline{|:==:|}
        Izjave & $0.5$    \\ 
        \hline
        Logične operacije & $5$    \\ 
        \hline
        Pomen izjav v matematiki & $0.5$    \\ 
        \hline
        Množice & $0.5$     \\
        \hline
        Moč množice & $1$     \\
        \hline
        Podmnožice & $1$    \\ 
        \hline
        Operacije z množicami & $4.5$    \\ 
        \hhline{|:==:|}
        Skupaj & $13$     \\
        \hhline{|b:==:b|}
        \end{tabular}
    \end{table}


    
\end{priprava}