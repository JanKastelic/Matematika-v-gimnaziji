\begin{priprava}{6}{}{Pomen izjav v matematiki}{Osnove logike in teorije množic}{frontalna}{drsnice, projekcija, tabla}


    \section{Pomen izjav v matematiki}
              
    \textbf{Aksiomi} so najpreprostejše izjave, ki so očitno pravilne in zato njihove 
    pravilnosti ni treba dokazovati.
 
    ~
  
    \textbf{Izreki} ali \textbf{teoremi} so izjave, ki so pravilne, vendar pa njihova 
    pravilnost ni očitna. 
    Pravilnost izreka (teorema) moramo potrditi z dokazom, ki temelji na aksiomih in na 
    preprostejših že prej dokazanih izrekih.
 
    ~
  
    \textbf{Definicije} so izjave, s katerimi uvajamo nove pojme. Najpreprostejših pojmov 
    v matematiki ne opisujemo z definicijami (to so pojmi kot npr.: število, premica ipd.); 
    vsak nadaljnji pojem pa moramo definirati, zato da se nedvoumno ve, o čem govorimo.
 


\end{priprava}