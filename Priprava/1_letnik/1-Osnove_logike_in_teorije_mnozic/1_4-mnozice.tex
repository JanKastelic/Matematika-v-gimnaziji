\begin{priprava}{7}{}{Množice}{Osnove logike in teorije množic}{frontalna}{drsnice, projekcija, tabla}


     
    \section{Množice}
        
    \textbf{Množica} je skupek elementov, ki imajo neko skupno lastnost.


    Množica je določena, če:
    \begin{itemize}
        \item lahko naštejemo vse njene elemente ali
        \item poznamo pravilo/skupno lastnost, ki pove, kateri elementi so v množici.
    \end{itemize}

    Označujemo jih z velikimi črkami ($\mathcal{A}, \mathcal{B}, \mathcal{C} \dots$ ali  
    $A, B, C \dots$ ). 
    \newline

    \textbf{Univerzalna množica} ali \textbf{univerzum} ($\mathcal{U}$) je množica 
    vseh elementov, ki v danem primeru nastopajo oziroma jih opazujemo.
    \newline

    \textbf{Element množice} je objekt v množici. 

    Označujemo jih z malimi črkami ($a, b, c \dots$). 

    Elemente množice zapisujemo v zavitem oklepaju (npr. $\mathcal{A}=\left\{a, b, c\right\}$).

    Element je lahko vsebovan v množici (npr. $a\in\mathcal{A}$) ali pa 
    v množici ni vsebovan (npr. $d\notin\mathcal{A}$).
    \newline

    \textbf{Prazna množica} ($\mathbf{\emptyset, \left\{\right\}}$) je množica, 
    ki ne vsebuje nobenega elementa.



\end{priprava}