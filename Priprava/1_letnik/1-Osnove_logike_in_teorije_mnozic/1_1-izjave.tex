\begin{priprava}{1}{}{Izjave}{Osnove logike in teorije množic}{frontalna}{drsnice, projekcija, tabla}


    \section{Izjave}

    \textbf{Matematična izjava} je vsaka smiselna poved, za katero 
    lahko določimo resničnost oziroma pravilnost.

     
    Matematična izjava lahko zavzame dve logični vrednosti:
    \begin{itemize}
        \item izjava je \textbf{resnična}/\textbf{pravilna}, 
            oznaka $\mathbf{R}$/$\mathbf{P}$/$\mathbf{1}$/$\mathbf{\top}$;
        \item izjava je \textbf{neresnična}/\textbf{nepravilna}, 
            oznaka $\mathbf{N}$/$\mathbf{0}$/$\mathbf{\bot }$.
    \end{itemize}                

     
    Izjave označujemo z velikimi tiskanimi črkami ($A$, $B$, $C$ ...).
 

    ~\\


 \begin{naloga}
    Ali so naslednje povedi izjave?
    \begin{itemize}   
        \item Danes sije sonce.
        \item Koliko je ura?
        \item Piramida je geometrijski lik.
        \item Daj mi jabolko.
        \item Število $12$ deli število $3$.
        \item Število $3$ deli število $10$.
        \item Ali si pisal matematični test odlično?
        \item Matematični test si pisal odlično.
        \item Ali je $10~dl$ isto kot $1~l$?
        \item Število $41$ je praštevilo.
    \end{itemize}
\end{naloga}
 



 \begin{naloga}
    Spodnjim izjavam določite logične vrednosti.
    \begin{itemize}   
        \item $A$: Najvišja gora v Evropi je Mont Blanc.
        \item $B$: Število je deljivo s $4$ natanko takrat, ko je vsota števk deljiva s $4$.
        \item $C$: Ostanek pri deljenju s $4$ je lahko $1$, $2$ ali $3$.
        \item $D$: Mesec februar ima 28 dni.
        \item $E$: Vsa praštevila so liha števila.
        \item $F$: Število $1$ je naravno število.
        \item $G$: Praštevil je neskončno mnogo.
    \end{itemize}
\end{naloga}

~


  \subsection{Enostavne in sestavjene izjave}
    
    Izjave delimo med:
    \begin{itemize}
        \item \textbf{elementarne}/\textbf{enostavne izjave} -- ne moremo 
            jih razstaviti na bolj enostavne;
        \item \textbf{sestavljene izjave} -- sestavljene iz elementarnih izjav, 
            ki jih med seboj povezujejo \textbf{logične operacije} (imenovane 
            tudi izjavne povezave oziroma~ logična vezja).
    \end{itemize}
 

   
    Vrednost sestavljene izjave izračunamo glede na vrednosti elementarnih 
    izjav in izjavnih povezav med njimi.
 
   
    Pravilnost sestavljenih izjav nazorno prikazujejo 
    \textbf{resničnostne}/\textbf{pravilnostne tabele}.
 


\end{priprava}