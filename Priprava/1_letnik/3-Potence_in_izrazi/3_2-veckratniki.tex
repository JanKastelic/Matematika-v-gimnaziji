\begin{priprava}{6}{}{Večkratniki}{Potence in izrazi}{frontalna}{drsnice, projekcija, tabla}


    \section{Večkratniki}

        
            
            
    \textbf{Večkratnik} ali tudi \textbf{$k$-kratnik} števila $x$ je vsota $k$ enakih sumandov $x$:
        $${k\cdot x=\underbrace{x+x+\ldots+x}_\text{$k$ sumandov}}.$$



    Vse večkratnike števila $x$ dobimo tako, da število $x$ zapored pomnožimo z vsemi celimi števili:
    $$\left\{\ldots,-5x, -4x, -3x, -2x, -x, 0, x, 2x, 3x, 4x, 5x, \ldots\right\}=\left\{kx;\ k,x\in\mathbb{Z}\right\}=x\mathbb{Z}. $$



    Število $\mathbf{k}$ je \textbf{koeficient} števila oziroma spremenljivke $x$.


    

\end{priprava}