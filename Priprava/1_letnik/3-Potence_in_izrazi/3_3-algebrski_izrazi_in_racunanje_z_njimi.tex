\begin{priprava}{6., 7., 8., 9., 10}{}{Algebrski izrazi in računanje z njimi}{Potence in izrazi}{frontalna}{drsnice, projekcija, tabla}


    \section{Algebrski izrazi}
    
        
    
            
    \textbf{Algebrski izraz} ali \textbf{izraz} je smiseln zapis sestavljen iz:
    \begin{itemize}
        \item števil,
        \item spremenljivk/parametrov, ki predstavljajo števila in jih označujemo s črkami,
        \item oznak računskih operacij in funkcij, ki jih povezujejo,
        \item oklepajev, ki določajo vrstni red računanja. 
    \end{itemize}
~


    Če v izraz namesto spremenljivk vstavimo konkretna števila in izračunamo rezultat, dobimo \textbf{vrednost izraza} (pri dani izbiri spremenljivk).

~


    Dva matematična izraza sta \textbf{enakovredna}, če imata pri katerikoli izbiri spremenljivk vedno enako vrednost.





\section{Računanje z algebrskimi izrazi}





    Pri poenostavljanju izrazov veljajo vsi računski zakoni, ki veljajo za računanje s števili.

    \begin{multicols}{2}

    \subsubsection*{Komutativnost seštevanja}
        $$ \mathbf{x+ y=y+ x}$$
    

        \subsubsection*{Asociativnost seštevanja}
        $$ \mathbf{(x+ y)+ z=x+ (y+ z)}$$
    


        \subsubsection*{Komutativnost množenja}
       $$ \mathbf{x\cdot y=y\cdot x}$$
    

       \subsubsection*{Asociativnost množenja}
        $$ \mathbf{(x\cdot y)\cdot z=x\cdot (y\cdot z)}$$
    
    \end{multicols}


    \subsubsection*{Distributivnost seštevanja in množenja}
        $$ (x+y)\cdot z=\mathbf{x\cdot z+y\cdot z} $$
    




    Če v distributivnostnem zakonu zamenjamo levo in desno stran, dobimo pravilo o \textbf{izpostavljanju skupnega faktorja}: $xz+yz=(x+y)z$.


\subsection{Seštevanje in izpostavljanje izrazov}
    Med seboj lahko seštevamo samo člene, ki se razlikujejo kvečjemu v koeficientu. To naredimo tako, da seštejemo koeficienta.
    $$mx^2+ny+kx^2+ly=mx^2+kx^2+ny+ly=(m+k)x^2+(n+l)y $$


\subsection{Množenje izrazov}
    Dva izraza zmnožimo tako, da vsak člen prvega izraza zmnožimo z vsakim členom drugega izraza. Potem pa seštejemo podobne člene.
    $$(x+y)(z+w)=xz+xw+yz+yw $$


~\\
\begin{multicols}{2}
    

\begin{naloga}
    Poenostavite.
    \begin{itemize}
        \item $3a+2b-a+7b$ 
        \item $2a^2b-ab^2+3a^2b$ 
        \item $5a^4-(2a)^4+(-3a^2)^2-3(a^2)^2$ 
        \item $3(a-2(a+b))-2(b-a(-2)^2)$ 
    \end{itemize}
\end{naloga}

~

\begin{naloga}
    Zapišite izraz.
    \begin{itemize}
        \item Kvadrat razlike števil $x$ in $y$. 
        \item Razlika kvadratov števil $x$ in $y$. 
        \item Razlika petkratnika $m$ in kvadrata števila $3$. 
        \item Kub razlike sedemkratnika števila $x$ in trikratnika števila $y$. 
    \end{itemize}
\end{naloga}



\begin{naloga}
    Izpostavite skupni faktor.
    \begin{itemize}
        \item $3x+12y^2$ 
        \item $m^3+8mp$ 
        \item $22a^3-33ab$ 
        \item $kr^2-rk^2$ 
        \item $4u^2v^3-6uv^2$ 
        \item $12a^2b-8(ab)^2-(2ab)^4$ 
    \end{itemize}
\end{naloga}



\begin{naloga}
    Izpostavite skupni faktor.
    \begin{itemize}
        \item $3x(x+1)+5(x+1)$ 
        \item $(a-1)(a+1)+(a-1)$ 
        \item $4(m-1)-(1-m)(a+b)$ 
        \item $3(c-2)+5c(2-x)$ 
        \item $(-y+x)3a-(y-x)b$ 
    \end{itemize}
\end{naloga}



\begin{naloga}
    Izpostavite skupni faktor.
    \begin{itemize}
        \item $5^{11}-5^{10}+5^9$ 
        \item $2\cdot 3^8+5\cdot 3^6$ 
        \item $4\cdot 5^{10}-10\cdot 5^8-8\cdot 5^9$ 
        \item $7^5-7^6+7\cdot 7^4$ 
    \end{itemize}
\end{naloga}



\begin{naloga}
    Izpostavite skupni faktor.
    \begin{itemize}
        \item $3^n-2\cdot 3^{n+1}+3^{n+2}$ 
        \item $2^{k+2}-2^k$ 
        \item $5\cdot 3^m+2\cdot 3^{m+1}$ 
        \item $2^{n-3}+3\cdot 2^{n-2}-2^{n-1}$ 
        \item $3\cdot 5^{n+1}-5^{n+2}+4\cdot 5^{n+3}$ 
        \item $7^n+2\cdot 7^{n-1}-3\cdot 7^{n+1}$ 
    \end{itemize}
\end{naloga}



\begin{naloga}
    Izpostavite skupni faktor in izračunajte.
    \begin{itemize}
        \item $2^{2n}+4^n+(2^n)^2$ 
        \item $5^{2n+1}-25^n+3\cdot 5^{2n-1}$ 
        \item $5\cdot 2^{3n}-3\cdot 8^{n-1}$ 
        \item $49^n-2\cdot 7^{2n-1}$ 
    \end{itemize}
\end{naloga}




\begin{naloga}
    Izpostavite skupni faktor.
    \begin{itemize}
        \item $4a^n+6a^{n+1}$ 
        \item $b^n+b^{n+1}-2b^{n-1}$ 
        \item $a^{n-3}+5a^n$ 
        \item $3x^{n+1}-15x^n+18x^{n-1}$ 
    \end{itemize}
\end{naloga}



\begin{naloga}
    Zmnožite.
    \begin{itemize}
        \item $(x-3)(x+2)$ 
        \item $(2m+3)(5m-1)$ 
        \item $(1-a)(1+a)$ 
        \item $(x-3y)(2x+y)$ 
        \item $(m-2k)(3m-k)$ 
    \end{itemize}
\end{naloga}



\begin{naloga}
    Zmnožite.
    \begin{itemize}
        \item $(a+b-1)(a-b)$ 
        \item $(2x+y)(3x-4y+5)$ 
        \item $(m+2n-k)(m+2n+k)$ 
    \end{itemize}
\end{naloga}



\begin{naloga}
    Zmnožite.
    \begin{itemize}
        \item $(x^2-3)(x^3+2)$ 
        \item $(3x^2-y)(5y^4-7x^3)$ 
        \item $(u^3-1)(u^3+1)$ 
        \item $(a^5b^2-4b)(3a^7+2a^2b)$ 
        \item $(a-b)(a^2+ab+b^2)$ 
        \item $(z+w)(z^2-zw+w^2)$ 
    \end{itemize}
\end{naloga}



\begin{naloga}
    Poenostavite.
    \begin{itemize}
        \item $(2x-y)(3+y)+(y-4)(y+4)-2xy+3(y-2x+5)$ 
        \item $(x-y)(x+y)-(x^2+xy+y^2)(x-y)-(1-x)x^2+(-y)y^2$ 
        \item $2ab+(a-3b^2)(a+3b^2)+2^3(-b^2)^2-(a-b)(b-a)-2a^3$  
    \end{itemize}
\end{naloga}

~\\

\end{multicols}
    

\end{priprava}