\begin{priprava}{4., 5}{}{Osnovni računski zakoni v $\mathbb{Z}$}{Naravna in cela števila}{frontalna}{drsnice, projekcija, tabla}


    \section{Osnovni računski zakoni v $\mathbb{Z}$}



    \subsection*{Komutativnost seštevanja}
    $$ \mathbf{x+y=y+x}$$
    Vsota ni odvisna od vrstnega reda seštevanja.
 

    \subsection*{Asociativnost seštevanja}
    $$ \mathbf{(x+y)+z=x+(y+z)}$$
    Vsota več kot dveh sumandov ni odvisna od združevanja po dveh sumandov.
 

    \subsection*{Komutativnost množenja}
    $$ \mathbf{x\cdot y=y\cdot x}$$
    Produkt ni odvisna od vrstnega reda faktorjev.
 

    \subsection*{Asociativnost množenja}
    $$ \mathbf{(x\cdot y)\cdot z=x\cdot (y\cdot z)}$$
    Produkt več kot dveh sumandov ni odvisen od združevanja faktorjev.
 
    \subsection*{Distributivnost seštevanja in množenja ter odštevanja in množenja}
    $$ \mathbf{x\cdot z+y\cdot z = (x+y)\cdot z} $$
    $$ \mathbf{x\cdot z-y\cdot z = (x-y)\cdot z} $$

    Če to beremo iz desne proti levi, rečemu tudi \textit{pravilo izpostavljanja skupnega faktorja}.

    ~\\

    \begin{multicols}{2}
        

    \begin{naloga}
        Izračunajte.
        \begin{itemize}
            \item $17-13-2+10$  
            \item $50+11-32-14$  
            \item $3+((5+2(7-9))\cdot 2-1)$  
            \item $(2-5(6-10))\cdot(5-2(7-5))$  
            \item $9(11-3)+7(10-15)$  
            \item $8+9(11-18)-2\cdot 5$  
        \end{itemize}
    \end{naloga}

    \begin{naloga}
        Spretno izračunajte.
        \begin{itemize}
            \item $7\cdot 8-12\cdot 8$  
            \item $5\cdot 18+9\cdot 5-5\cdot 2$  
            \item $8\cdot(4-9)\cdot 2$  
            \item $5\cdot 3\cdot (12-8)$  
            \item $(15-6)(12-3\cdot 4)$  
        \end{itemize}
    \end{naloga}
    ~
\end{multicols}


    \begin{naloga}
        Rešite besedilne naloge.
        \begin{itemize}
            \item V hotelu imajo na voljo osemnajst enoposteljnih, štiriintrideset dvoposteljnih in petindevetdeset triposteljnih sob.
                Koliko ljudi lahko še prespi v hotelu, če je v njem že sto triinštirideset gostov?      
            \item Pohod na bližnji hrib traja tri ure. Koliko minut moramo še hoditi, če smo na poti že $145$ minut?       
            \item S Ptuja in iz Postojne (razdalja med njima je približno $190~km$) sočasno odpeljeta dva motorista drug proti drugemu.
                En vozi povprečno $40~km/h$, drugi pa $5~km/h$ manj. Kolikšna bo razdalja med njima po dveh urah vožnje?      
        \end{itemize}
    \end{naloga}

    \begin{naloga}
        Zapišite enačbe in jih poenostavite.
        \begin{itemize}
            \item Razlika petkratnka $a$ in $b$ je enaka trikratniku vsote štirikratnika $a$ in petkratnika $b$.   
            \item Vsota $x$ in dvakratnika $y$ je enaka razliki petkratnika $x$ in dvanajstkratnika $y$.    
        \end{itemize}
    \end{naloga}


\end{priprava}