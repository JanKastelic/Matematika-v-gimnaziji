\begin{priprava}{6}{}{Urejenost naravnih in celih števil}{Naravna in cela števila}{frontalna}{drsnice, projekcija, tabla}


    \section{Urejenost naravnih in celih števil}


    
        Številska množica je \textbf{urejena}, kadar lahko po velikosti primerjamo njena poljubna elementa.
    

    
        Pri urejanju števil uporabljamo naslednje znake:
        \begin{table}[H]
            \centering
            \addtolength{\tabcolsep}{6pt}
            \renewcommand{\arraystretch}{1.4}                
            \begin{tabular}{||c|c||} 
                \hhline{|t:==:t|}
                        $\mathbf{<}$ & manjše / manj  \\ 
                \hline
                        $\mathbf{>}$ & večje / več   \\ 
                \hline
                        $\mathbf{\leq}$ & manjše ali enako / največ   \\ 
                \hline
                        $\mathbf{\geq}$ & večje ali enako / vsaj, najmanj \\  
                \hline
                        $\mathbf{=}$ & enako \\
                \hhline{|b:==:b|}
            \end{tabular}
        \end{table}
    



    
        Za poljubni števili $x,y\in\mathbb{Z}$ velja natanko ena izmed naslednjih možnosti: $x>y$, $x<y$ ali $x=y$.
    \newline

            Slika števila $x$ leži na številski premici desno od slike števila $y$:
        $$\mathbf{x>y \Leftrightarrow x-y>0}$$
    

            Slika števila $x$ leži na številski premici levo od slike števila $y$:
        $$\mathbf{x<y \Leftrightarrow x-y<0}$$
    

            Slika števila $x$ sovpada s sliko števila $y$:
        $$\mathbf{x=y \Leftrightarrow x-y=0}$$
    
        ~

    
        Velja pa tudi:
        $$x\leq y \Leftrightarrow x-y\leq 0 $$
        $$x\geq y \Leftrightarrow x-y\geq 0 $$


    \subsubsection*{Pozitivna in negativna števila}
        V množici $\mathbb{Z}$ so pozitivna tista števila, ki so večja od števila $0$ 
        in njihove slike ležijo desno od izhodišča, 
        negativna pa tista števila, ki so manjša od števila $0$ 
        in njihove slike ležijo levo od izhodišča.
    
        Vsako pozitivno celo število (vsako naravno število) je večje od katerega koli negativnega celega števila.
    



    


        \subsection{Linearna urejenost}

    
        Z relacijo \textit{biti manjši ali enak} je množica $\mathbb{Z}$ \textbf{linearno urejena}, 
        to pomeni, da veljajo naslednje lastnosti: refleksivnost, antisimetričnost, tranzitivnost, stroga sovisnost.
    
        \begin{multicols}{2}
    
        \subsubsection*{Refleksivnost}
        $$\forall x\in\mathbb{Z}: x\leq x$$
    

        \subsubsection*{Antisimetričnost}
        $$\forall x,y\in\mathbb{Z}: x\leq y \land y\leq x \Rightarrow x=y$$
    

        \subsubsection*{Tranzitivnost}
        $$\forall x,y,z\in\mathbb{Z}: x\leq y \land y\leq z \Rightarrow x\leq z$$
    

        \subsubsection*{Stroga sovisnost}
        $$\forall x,y\in\mathbb{Z}: x\leq y \lor y\leq x$$

        \end{multicols}



        \subsection{Lastnosti relacij $\leq$ in $<$}
        \subsubsection*{Monotonost vsote}
        $$x<y \Rightarrow x+z<y+z \quad \quad x\leq y \Rightarrow x+z\leq y+z$$
        Če na obeh straneh neenakosti prištejemo isto število, se neenakost ohrani.
    \newline

    
        $$x<y \land z>0 \Rightarrow x\cdot z<y\cdot z \quad \quad x\leq y \land z>0 \Rightarrow x\cdot z\leq y\cdot z$$
        Pri množenju neenakosti z negativnim številom se znak neenakosti ohrani.
        \newline

    
        $$x<y \land z<0 \Rightarrow x\cdot z>y\cdot z \quad \quad x\leq y \land z<0 \Rightarrow x\cdot z\geq y\cdot z$$
        Pri množenju neenakosti z negativnim številom se znak neenakosti obrne.
        \newline ~
        \newline

    
        Obravnavane lastnosti veljajo tudi za relaciji $\geq$ in $>$.
        \newline ~
        \newline ~
        \newline





        \begin{naloga}
            Uredite števila $3, -2, 5, -1, 0, -7, 6, -6$ po velikosti in jih predstavite na številski premici.
        \end{naloga}

        \begin{naloga}
            Uredite števila $104, -27, 35, -107, 36, -26, 25, -28, 81$ po velikosti.
        \end{naloga}

        \begin{naloga}
            Gladina Mrtvega morja leži v depresiji na $-423~m$ nadmorske višine, njegova največja globina pa je $378~m$.
            Kolikšna je najmanjša nadmorska višina dna Mrtvega morja?
        \end{naloga}

        \begin{naloga}
            Za katera cela števila $x$ ima izraz $3x-5(x+2)$ večjo ali enako vrednost od izraza $4-(12+x)$?
        \end{naloga}



\end{priprava}