\begin{priprava}{2., 3}{}{Osnovni računski zakoni}{Naravna in cela števila}{frontalna}{drsnice, projekcija, tabla}


    \section{Osnovni računski zakoni}

    \subsection*{Komutativnost seštevanja -- zakon o zamenjavi členov}
       $$ \mathbf{x+y=y+x}$$
       Vsota ni odvisna od vrstnega reda seštevanja.
    

       \subsection*{Asociativnost seštevanja -- zakon o poljubnem združevanju členov}
       $$ \mathbf{(x+y)+z=x+(y+z)}$$
       Vsota več kot dveh sumandov ni odvisna od združevanja po dveh sumandov.
    

       \subsection*{Komutativnost množenja -- zakon o zamenjavi faktorjev}
       $$ \mathbf{x\cdot y=y\cdot x}$$
       Produkt ni odvisna od vrstnega reda faktorjev.
    

       \subsection*{Asociativnost množenja -- zakon o poljubnem združevanju faktorjev}
       $$ \mathbf{(x\cdot y)\cdot z=x\cdot (y\cdot z)}$$
       Produkt več kot dveh sumandov ni odvisen od združevanja faktorjev.
    
       \subsection*{Distributivnost -- zakon o razčlenjevanju}
       $$ \mathbf{x\cdot z+y\cdot z = (x+y)\cdot z} $$
       Če to beremo iz desne proti levi, rečemu tudi \textit{pravilo izpostavljanja skupnega faktorja}.
       \newline ~
       \newline


\begin{multicols}{2}
    \begin{naloga}
       Izračunajte.
       \begin{itemize}
           \item $(1+2\cdot 7)+3\cdot(2\cdot 2+7)$ 
           \item $3\cdot(2+3\cdot 5)\cdot(2+1)$ 
           \item $7+(2+6\cdot 3)+(8+4\cdot 5)$ 
           \item $11\cdot 4+(12-6)\cdot 5$ 
           \item $8+2\cdot(3+7)-15$ 
           \item $37-5\cdot(10-3)$ 
       \end{itemize}
    \end{naloga}

~

    \begin{naloga}
       Hitro izračunajte.
       \begin{itemize}
           \item $45+37+15$ 
           \item $108+46-28$
           \item $5\cdot 13\cdot 8$
           \item $4\cdot 7\cdot 25$
           \item $(7+3)\cdot 2\cdot 5$
           \item $15\cdot(4+6)\cdot 2$
           \item $3\cdot 5+7\cdot 5$
           \item $8\cdot 12+6\cdot 8$
       \end{itemize}
    \end{naloga}

\end{multicols}

    \begin{naloga}
       Zapišite račun glede na besedilo in izračunajte.
       \begin{itemize}
           \item Produktu števil $12$ in $27$ odštejte razliko števil $19$ in $11$. 
           \item Vsoti produkta $4$ in $12$ ter produkta $5$ in $16$ odštejte $8$. 
           \item Vsoto števil $42$ in $23$ pomnožite z razliko števil $58$ in $29$. 
           \item Produkt števil $14$ in $17$ pomnožite z vsoto števil $5$ in $16$. 
       \end{itemize}
    \end{naloga}



    \begin{naloga}
       Rešite besedilno nalogo.
       \begin{itemize}
           \item V trgovini kupimo tri litre mleka in štiri čokoladne pudinge v prahu. Če stane liter mleka $95$ centov,
               čokoladni puding v prahu pa $24$ centov, koliko moramo plačati? 
           \item Manca bo kuhala rižoto za štiri otroke in šest odraslih. Za otroško porcijo rižote zadošča $45~g$ riža,
               za odraslo pa $75~g$. Koliko riža mora dati kuhati za rižoto? 
       \end{itemize}
    \end{naloga}


\end{priprava}