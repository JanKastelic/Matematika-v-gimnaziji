\begin{priprava}{1., 2}{}{Operacije v množici $\mathbb{N}$}{Naravna in cela števila}{frontalna}{drsnice, projekcija, tabla}

    

 
    \section{Operacije v množici $\mathbb{N}$}

    \subsection{Seštevanje}
       Poljubnima naravnima številoma $x$ in $y$ priredimo \textbf{vsoto} $\mathbf{x+y}$.
       \newline

       Število $x$ oziroma $y$ imenujemo \textbf{seštevanec} ali \textbf{sumand} ali \textbf{člen}. 
       Število $x+y$ pa imenujemo \textbf{vsota} ali \textbf{summa}. 

    \begin{figure}[H]
       \centering

       \begin{tikzpicture}
           % \clip (0,0) rectangle (14.000000,10.000000);
           {\footnotesize
           
           % Drawing segment a b
           \draw [line width=0.016cm] (5.000000,0.500000) -- (5.000000,2.500000);%
           
           % Drawing segment b d
           \draw [line width=0.016cm] (5.000000,2.500000) -- (7.000000,2.500000);%
           
           % Drawing segment c d
           \draw [line width=0.016cm] (7.000000,0.500000) -- (7.000000,2.500000);%
           
           % Drawing segment c a
           \draw [line width=0.016cm] (7.000000,0.500000) -- (5.000000,0.500000);%
           
           % Drawing segment e f
           \draw [line width=0.016cm] (7.000000,1.500000) -- (9.000000,1.500000);%
           
           % Drawing arrow e f 1.00
           \draw [line width=0.016cm] (8.702567,1.539158) -- (9.000000,1.500000);%
           \draw [line width=0.016cm] (8.702567,1.539158) -- (8.900856,1.500000);%
           \draw [line width=0.016cm] (8.702567,1.460842) -- (9.000000,1.500000);%
           \draw [line width=0.016cm] (8.702567,1.460842) -- (8.900856,1.500000);%
           
           % Drawing segment g h
           \draw [line width=0.016cm] (2.500000,1.000000) -- (5.000000,1.000000);%
           
           % Drawing segment i j
           \draw [line width=0.016cm] (2.500000,2.000000) -- (5.000000,2.000000);%
           
           % Drawing arrow g h 1.00
           \draw [line width=0.016cm] (4.702567,1.039158) -- (5.000000,1.000000);%
           \draw [line width=0.016cm] (4.702567,1.039158) -- (4.900856,1.000000);%
           \draw [line width=0.016cm] (4.702567,0.960842) -- (5.000000,1.000000);%
           \draw [line width=0.016cm] (4.702567,0.960842) -- (4.900856,1.000000);%
           
           % Drawing arrow i j 1.00
           \draw [line width=0.016cm] (4.702567,2.039158) -- (5.000000,2.000000);%
           \draw [line width=0.016cm] (4.702567,2.039158) -- (4.900856,2.000000);%
           \draw [line width=0.016cm] (4.702567,1.960842) -- (5.000000,2.000000);%
           \draw [line width=0.016cm] (4.702567,1.960842) -- (4.900856,2.000000);%
           
           % Marking point {vsota}
           \draw (8.000000,1.500000) node [anchor=south] { ${vsota}$ };%
           
           % Marking point {summa}
           \draw (8.000000,1.500000) node [anchor=north] { ${summa}$ };%
           
           % Marking point {se�tevanec}
           \draw (3.750000,1.000000) node [anchor=south] { ${seštevanec}$ };%
           
           % Marking point {sumand}
           \draw (3.750000,1.000000) node [anchor=north] { ${sumand}$ };%
           
           % Marking point {se�tevanec}
           \draw (3.750000,2.000000) node [anchor=south] { ${seštevanec}$ };%
           
           % Marking point {sumand}
           \draw (3.750000,2.000000) node [anchor=north] { ${sumand}$ };%
           
           % Drawing segment x y
           \draw [line width=0.032cm] (6.000000,1.000000) -- (6.000000,2.000000);%
           
           % Drawing segment z w
           \draw [line width=0.032cm] (5.500000,1.500000) -- (6.500000,1.500000);%
           }
           \end{tikzpicture}
           
   \end{figure}


    


     
       Vsota naravnih števil je naravno število: $x, y \in \mathbb{N} \Rightarrow x+y \in \mathbb{N}$.

    




    \subsection{Množenje}
       Poljubnima naravnima številoma $x$ in $y$ priredimo \textbf{produkt} $\mathbf{x\cdot y}$.
       \newline

       Število $x$ oziroma $y$ imenujemo \textbf{množenec} ali \textbf{faktor}. 
       Število $x\cdot y$ pa imenujemo \textbf{zmnožek} ali \textbf{produkt}. 

        \begin{figure}[H]
           \centering

       \begin{tikzpicture}
           % \clip (0,0) rectangle (14.000000,10.000000);
           {\footnotesize
           
           % Drawing segment a b
           \draw [line width=0.016cm] (5.000000,0.500000) -- (5.000000,2.500000);%
           
           % Drawing segment b d
           \draw [line width=0.016cm] (5.000000,2.500000) -- (7.000000,2.500000);%
           
           % Drawing segment c d
           \draw [line width=0.016cm] (7.000000,0.500000) -- (7.000000,2.500000);%
           
           % Drawing segment c a
           \draw [line width=0.016cm] (7.000000,0.500000) -- (5.000000,0.500000);%
           
           % Drawing segment e f
           \draw [line width=0.016cm] (7.000000,1.500000) -- (9.000000,1.500000);%
           
           % Drawing arrow e f 1.00
           \draw [line width=0.016cm] (8.702567,1.539158) -- (9.000000,1.500000);%
           \draw [line width=0.016cm] (8.702567,1.539158) -- (8.900856,1.500000);%
           \draw [line width=0.016cm] (8.702567,1.460842) -- (9.000000,1.500000);%
           \draw [line width=0.016cm] (8.702567,1.460842) -- (8.900856,1.500000);%
           
           % Drawing segment g h
           \draw [line width=0.016cm] (2.500000,1.000000) -- (5.000000,1.000000);%
           
           % Drawing segment i j
           \draw [line width=0.016cm] (2.500000,2.000000) -- (5.000000,2.000000);%
           
           % Drawing arrow g h 1.00
           \draw [line width=0.016cm] (4.702567,1.039158) -- (5.000000,1.000000);%
           \draw [line width=0.016cm] (4.702567,1.039158) -- (4.900856,1.000000);%
           \draw [line width=0.016cm] (4.702567,0.960842) -- (5.000000,1.000000);%
           \draw [line width=0.016cm] (4.702567,0.960842) -- (4.900856,1.000000);%
           
           % Drawing arrow i j 1.00
           \draw [line width=0.016cm] (4.702567,2.039158) -- (5.000000,2.000000);%
           \draw [line width=0.016cm] (4.702567,2.039158) -- (4.900856,2.000000);%
           \draw [line width=0.016cm] (4.702567,1.960842) -- (5.000000,2.000000);%
           \draw [line width=0.016cm] (4.702567,1.960842) -- (4.900856,2.000000);%
           
           % Marking point {zmno�ek}
           \draw (8.000000,1.500000) node [anchor=south] { ${zmnožek}$ };%
           
           % Marking point {produkt}
           \draw (8.000000,1.500000) node [anchor=north] { ${produkt}$ };%
           
           % Marking point {mno�enec}
           \draw (3.750000,1.000000) node [anchor=south] { ${množenec}$ };%
           
           % Marking point {faktor}
           \draw (3.750000,1.000000) node [anchor=north] { ${faktor}$ };%
           
           % Marking point {mno�enec}
           \draw (3.750000,2.000000) node [anchor=south] { ${množenec}$ };%
           
           % Marking point {faktor}
           \draw (3.750000,2.000000) node [anchor=north] { ${faktor}$ };%
           
           % Drawing circle k
           \draw [line width=0.016cm] (6.000000,1.500000) circle (0.100000);%
           
           % Filling circle k
           \fill (6.000000,1.500000) circle (0.100000);%
           }
           \end{tikzpicture}
           
   \end{figure}

    
       Produkt naravnih števil je naravno število: $x, y \in \mathbb{N} \Rightarrow x\cdot y \in \mathbb{N}$.
       \newline

       Število $\mathbf{1}$ je \textbf{nevtralni element} za mmnoženje: $1\cdot x = x$.
    \newline


       Seštevanje in množenje sta \textit{dvočleni notranji operaciji} v množici naravnih števil $\mathbb{N}$.


   

    \subsection{Odštevanje}
       Številoma $x$ in $y$, pri čemer je $y$ večje od $x$ ($x>y$), priredimo \textbf{razliko} $\mathbf{x-y}$.                
       \newline

       Število $x$ imenujemo \textbf{zmanjševanec} ali \textbf{minuend}, število $y$  pa imenujemo \textbf{odštevanec} ali \textbf{subtrahend}. 
       Številu $x-y$ rečemo \textbf{razlika} ali \textbf{diferenca}. 

        \begin{figure}[H]
           \centering
           \begin{tikzpicture}
               % \clip (0,0) rectangle (14.000000,10.000000);
               {\footnotesize
               
               % Drawing segment a b
               \draw [line width=0.016cm] (5.000000,0.500000) -- (5.000000,2.500000);%
               
               % Drawing segment b d
               \draw [line width=0.016cm] (5.000000,2.500000) -- (7.000000,2.500000);%
               
               % Drawing segment c d
               \draw [line width=0.016cm] (7.000000,0.500000) -- (7.000000,2.500000);%
               
               % Drawing segment c a
               \draw [line width=0.016cm] (7.000000,0.500000) -- (5.000000,0.500000);%
               
               % Drawing segment e f
               \draw [line width=0.016cm] (7.000000,1.500000) -- (9.000000,1.500000);%
               
               % Drawing arrow e f 1.00
               \draw [line width=0.016cm] (8.702567,1.539158) -- (9.000000,1.500000);%
               \draw [line width=0.016cm] (8.702567,1.539158) -- (8.900856,1.500000);%
               \draw [line width=0.016cm] (8.702567,1.460842) -- (9.000000,1.500000);%
               \draw [line width=0.016cm] (8.702567,1.460842) -- (8.900856,1.500000);%
               
               % Drawing segment g h
               \draw [line width=0.016cm] (2.500000,1.000000) -- (5.000000,1.000000);%
               
               % Drawing segment i j
               \draw [line width=0.016cm] (2.500000,2.000000) -- (5.000000,2.000000);%
               
               % Drawing arrow g h 1.00
               \draw [line width=0.016cm] (4.702567,1.039158) -- (5.000000,1.000000);%
               \draw [line width=0.016cm] (4.702567,1.039158) -- (4.900856,1.000000);%
               \draw [line width=0.016cm] (4.702567,0.960842) -- (5.000000,1.000000);%
               \draw [line width=0.016cm] (4.702567,0.960842) -- (4.900856,1.000000);%
               
               % Drawing arrow i j 1.00
               \draw [line width=0.016cm] (4.702567,2.039158) -- (5.000000,2.000000);%
               \draw [line width=0.016cm] (4.702567,2.039158) -- (4.900856,2.000000);%
               \draw [line width=0.016cm] (4.702567,1.960842) -- (5.000000,2.000000);%
               \draw [line width=0.016cm] (4.702567,1.960842) -- (4.900856,2.000000);%
               
               % Marking point {razlika}
               \draw (8.000000,1.500000) node [anchor=south] { ${razlika}$ };%
               
               % Marking point {diferenca}
               \draw (8.000000,1.500000) node [anchor=north] { ${diferenca}$ };%
               
               % Marking point {od�tevanec}
               \draw (3.750000,1.000000) node [anchor=south] { ${odštevanec}$ };%
               
               % Marking point {subtrahend}
               \draw (3.750000,1.000000) node [anchor=north] { ${subtrahend}$ };%
               
               % Marking point {zmanj�evanec}
               \draw (3.750000,2.000000) node [anchor=south] { ${zmanjševanec}$ };%
               
               % Marking point {minuend}
               \draw (3.750000,2.000000) node [anchor=north] { ${minuend}$ };%
               
               % Drawing segment z w
               \draw [line width=0.032cm] (5.500000,1.500000) -- (6.500000,1.500000);%
               }
               \end{tikzpicture}
               
   \end{figure}


     
       Razlika je število, ki ga moramo prišteti številu $y$, da dobimo število $y$.
       $$ (x-y)+y=x $$


       Odštevanje ni notranja operacija v množici naravnih števil $\mathbb{N}$.
    

    \subsection{Vrstni red operacij}
       Prednost pri računanju imajo \textbf{oklepaji} (najprej najbolj notranji), nato sledi \textbf{množenje},
       na koncu pa imamo še \textbf{seštevanje} in \textbf{odštevanje}.
    

     
       Kadar v izrazu nastopajo enakovredne računske operacije, računamo od leve proti desni.
    

       ~\newline

       Pri množenju količin, ki so označene s črkovnimi oznakami, piko, ki označuje operacijo množenja ponavadi opustimo.
       $$ x\cdot y = xy$$
    




\end{priprava}