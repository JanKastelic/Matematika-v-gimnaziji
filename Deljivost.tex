\section{Deljivost}

\begin{frame}
    \sectionpage
\end{frame}

\begin{frame}
    \tableofcontents[currentsection, hideothersubsections]
\end{frame}

    \subsection{Relacija deljivosti}

        \begin{frame}
            \frametitle{Relacija deljivosti}

            \only<2->{\begin{alertblock}{}
                Naravno število $n$ je \textbf{delitelj} naravnega števila $n$ (\textbf{deljenec}), če obstaja naravno število $k$ (\textbf{kvocient}), da velja: $$\mathbf{n=k\cdot m}.$$
            \end{alertblock}}

            \only<3->{\begin{alertblock}{}
                Naravno število $m$ deli naravno število $n$, ko je število $n$ večkratnik števila $m$. $$m\mid n \Leftrightarrow n=k\cdot m;\quad m,n,k\in\mathbb{N}$$
            \end{alertblock}}

            \only<4->{\begin{block}{}
                    Število $m$ je delitelj samega sebe in vseh svojih večkratnikov.
            \end{block}}

            \only<5->{\begin{block}{}
                $1$ je delitelj vsakega naravnega števila.
            \end{block}}

            \only<6->{\begin{block}{}
                Če $d$ deli naravni števili $m$ in $n$, $n>m$, potem $d$ deli tudi vsoto in razliko števil $m$ in $n$.
            \end{block}}

        \end{frame}

        \begin{frame}
            \only<2->{\begin{block}{}
                Pri deljenju poljubnega naravnega števila $n$ z naravnim številom $m$ imamo dve možnosti: $n$ je deljivo z $m$ ali $n$ ni deljivo z $m$.
            \end{block}}

            \only<3->{\begin{block}{}
                Relacija deljivosti je:
                \only<4->{\begin{enumerate}
                    \item<4-> \textbf{refleksivna}: \only<5->{$$n\mid n;$$}
                    \item<6-> \textbf{antisimetrična}: \only<7->{$$m\mid n \wedge n\mid m \Rightarrow m=n;$$}
                    \item<8-> \textbf{tranzitivna}:  \only<9->{$$m\mid n \wedge n\mid o \Rightarrow m\mid o.$$}
                \end{enumerate}}
            \end{block}}

            \only<10->{\begin{block}{}
                Relacija s temi lastnostmi je relacija \textbf{delne urejenosti}, zato relacija deljivosti delno ureja množico $\mathbb{N}$.
            \end{block}}
        \end{frame}

        \begin{frame}
            \only<2->{\begin{exampleblock}{Naloga}
                Zapišite vse delitelje števil.
                \only<3->{\begin{itemize}
                    \item $6$ \\~
                    \item $16$ \\~
                    \item $37$ \\~
                    \item $48$ \\~
                    \item $120$ \\~\\~
                \end{itemize}}
            \end{exampleblock}}
        \end{frame}

        \begin{frame}
            \only<2->{\begin{exampleblock}{Naloga}
                Pokažite, da trditev velja.
                \only<3->{\begin{itemize}
                    \item Izraz $x-3$ deli izraz $x^2-2x-3$. \\~\\~\\~
                    \item Izraz $x+2$ deli izraz $x^3+x^2-4x-4$. \\~\\~\\~
                    \item Izraz $x-2$ deli izraz $x^3-8$. \\~\\~\\~
                \end{itemize}}
            \end{exampleblock}}
        \end{frame}

        \begin{frame}
            \only<2->{\begin{exampleblock}{Naloga}
                Pokažite, da trditev velja.
                \only<3->{\begin{itemize}
                    \item $19\mid \left(3^{21}-3^{20}+3^{18}\right)$ \\~
                    \item $7\mid \left(3\cdot 4^{11}+4^{12}+7\cdot 4^{10}\right)$ \\~
                    \item $14\mid \left(5\cdot 3^6+2\cdot 3^8-3\cdot 3^7\right)$ \\~
                    \item $25\mid \left(7\cdot 2^{23}-3\cdot 2^{24}+3\cdot 2^{25}-2^{22}\right)$ \\~
                    \item $11\mid \left(2\cdot 10^6+3\cdot 10^7+10^8\right)$ \\~
                    \item $35\mid \left(6^{32}-36^{15}\right)$ \\~
                \end{itemize}}
            \end{exampleblock}}
        \end{frame}

        \begin{frame}
            \only<2->{\begin{exampleblock}{Naloga}
                Pokažite, da trditev velja.
                \only<3->{\begin{itemize}
                    \item $3\mid \left(2^{2n+1}-5\cdot 2^{2n}+9\cdot 2^{2n-1}\right)$ \\~
                    \item $29\mid \left(5^{n+3}-2\cdot 5^{n+1}+7\cdot 5^{n+2}\right)$ \\~
                    \item $10\mid \left(3\cdot 7^{4n-1}-4\cdot 7^{4n-2}+7^{4n+1}\right)$ \\~
                    \item $10\mid \left(9^{3n-1}+9\cdot 9^{3n+1}+9^{3n}-9^{3n+2}\right)$ \\~
                    \item $5\mid \left(7\cdot 2^{4n-2}+3\cdot 4^{2n}-16^n\right)$ \\~\\~
                \end{itemize}}
            \end{exampleblock}}
        \end{frame}


        \begin{frame}
            \only<2->{\begin{exampleblock}{Naloga}
                Pokažite, da je za poljubno naravno število $u$ vrednost izraza $$(u+7)(7-u)-3(3-u)(u+5)$$ večkratnik števila $4$.
            \end{exampleblock}}
        \end{frame}

% \begin{frame}
%     \only<2->{\begin{exampleblock}{Naloga}
%         Poenostavite.
%         \only<3->{\begin{itemize}
%             \item a \\~
%         \end{itemize}}
%     \end{exampleblock}}
% \end{frame}

% \begin{frame}
%     \only<2->{\begin{exampleblock}{Naloga}
%         Poenostavite.
%         \only<3->{\begin{itemize}
%             \item a \\~
%         \end{itemize}}
%     \end{exampleblock}}
% \end{frame}


% \begin{frame}
%     \only<2->{\begin{exampleblock}{Naloga}
%         Poenostavite.
%         \only<3->{\begin{itemize}
%             \item a \\~
%         \end{itemize}}
%     \end{exampleblock}}
% \end{frame}

% \begin{frame}
%     \only<2->{\begin{exampleblock}{Naloga}
%         Poenostavite.
%         \only<3->{\begin{itemize}
%             \item a \\~
%         \end{itemize}}
%     \end{exampleblock}}
% \end{frame}



    \subsection{Kriteriji deljivost}

        \begin{frame}
            \frametitle{Kriteriji deljivosti}

            \only<2->{\begin{alertblock}{Deljivost z $2$}
                \only<3->{Število je deljivo z $2$ natanko takrat, ko so enice števila deljive z $2$.}
            \end{alertblock}}

            \only<4->{\begin{alertblock}{Deljivost s $3$}
                \only<5->{Število je deljivo s $3$ natanko takrat, ko je vsota števk števila deljiva s $3$.}
            \end{alertblock}}

            \only<6->{\begin{alertblock}{Deljivost s $4$ oziroma $25$}
                \only<7->{Število je deljivo s $4$ oziroma $25$ natanko takrat, ko je dvomestni konec števila deljiv s $4$ oziroma $25$.}
            \end{alertblock}}

            \only<8->{\begin{alertblock}{Deljivost s $5$}
                \only<9->{Število je deljivo s $5$ natanko takrat, ko so enice števila enake $0$ ali $5$.}
            \end{alertblock}}

        \end{frame}

        \begin{frame}
            \only<2->{\begin{alertblock}{Deljivost s $6$}
                \only<3->{Število je deljivo s $6$ natanko takrat, ko je deljivo z $2$ in s $3$ hkrati.}
            \end{alertblock}}

            \only<4->{\begin{alertblock}{Deljivost z $8$ oziroma s $125$}
                \only<5->{Število je deljivo z $8$ oziroma s $125$ natanko takrat, ko je trimestni konec števila deljiv z~$8$ oziroma s $125$.}
            \end{alertblock}}

            \only<6->{\begin{alertblock}{Deljivost z $9$}
                \only<7->{Število je deljivo z $9$ natanko takrat, ko je vsota števk števila deljiva z $9$.}
            \end{alertblock}}

            \only<8->{\begin{alertblock}{Deljivost z $10$ oziroma $10^n$}
                \only<9->{Število je deljivo z $10$ natanko takrat, ko so enice števila enake $0$.}
                \only<10->{Število je deljivo z $10^n$ natanko takrat, ko ima število na zadnjih $n$ mestih števko $0$.}
            \end{alertblock}}

        \end{frame}

        \begin{frame}
            \only<2->{\begin{alertblock}{Deljivost z $11$}
                \only<3->{Število je deljivo z $11$ natanko takrat, ko je alternirajoča vsota števk tega števila deljiva z $11$.}
            \end{alertblock}}

            \only<4->{\begin{alertblock}{Deljivost s $7$}
                \only<5->{Algoritem za preverjanje deljivosti s $7$:
                \begin{enumerate}
                    \item<6-> vzamemo enice danega števila in jih pomnožimo s $5$,
                    \item<7-> prvotnemu številu brez enic prištejemo dobljeni produkt,
                    \item<8-> vzamemo enice dobljene vsote in jih pomnožimo s $5$,
                    \item<9-> produkt prištejemo prej dobljenemu številu ...     
                \end{enumerate}
                \only<10->{Postopek ponavljamo, dokler ne dobimo dvomestnega števila -- 
                če je to deljivo s $7$, je prvotno število deljivo s $7$. 
                }}
            \end{alertblock}}
        \end{frame}

        \begin{frame}
            \only<2->{\begin{exampleblock}{Naloga}
                S katerimi od števil $2$, $3$, $4$, $5$, $6$, $7$, $8$, $9$, $10$, $11$ so deljiva naslednja števila?
                \only<3->{\begin{itemize}
                    \item $84742$ \\~\\~
                    \item $393948$ \\~\\~
                    \item $12390$ \\~\\~
                    \item $19401$ \\~\\~
                \end{itemize}}
            \end{exampleblock}}
        \end{frame}

        \begin{frame}
            \only<2->{\begin{exampleblock}{Naloga}
                Določite vse možnosti za števko $a$, da je število $\overline{65833a}$:
                \only<3->{\begin{itemize}
                    \item deljivo s $3$, \\~
                    \item deljivo s $4$, \\~
                    \item deljivo s $5$, \\~
                    \item deljivo s $6$. \\~
                \end{itemize}}
            \end{exampleblock}}
        \end{frame}

        \begin{frame}
            \only<2->{\begin{exampleblock}{Naloga}
                Določite vse možnosti za števko $b$, da je število $\overline{65b90b}$:
                \only<3->{\begin{itemize}
                    \item deljivo z $2$, \\~
                    \item deljivo s $3$, \\~
                    \item deljivo s $6$, \\~
                    \item deljivo z $9$, \\~
                    \item deljivo z $10$. \\~
                \end{itemize}}
            \end{exampleblock}}
        \end{frame}

        \begin{frame}
            \only<2->{\begin{exampleblock}{Naloga}
                Določite vse možnosti za števki $c$ in $d$, da je število $\overline{115c1d}$ deljivo s $6$.
                \\~\\~\\~\\~
            \end{exampleblock}}

            \only<3->{\begin{exampleblock}{Naloga}
                Določite vse možnosti za števki $e$ in $f$, da je število $\overline{115e1f}$ deljivo z $8$.
                \\~\\~\\~\\~\\~
            \end{exampleblock}}

        \end{frame}


        \begin{frame}
            \only<2->{\begin{exampleblock}{Naloga}
                Pokažite, da za vsako naravno število $n$ $12$ deli $n^4-n^2$.
                \\~\\~\\~\\~
            \end{exampleblock}}

            \only<3->{\begin{exampleblock}{Naloga}
                Preverite, ali je število $8641 969$ deljivo s $7$.
                \\~\\~\\~\\~
            \end{exampleblock}}

        \end{frame}

% \begin{frame}
%     \only<2->{\begin{exampleblock}{Naloga}
%         Poenostavite.
%         \only<3->{\begin{itemize}
%             \item a \\~
%         \end{itemize}}
%     \end{exampleblock}}
% \end{frame}

% \begin{frame}
%     \only<2->{\begin{exampleblock}{Naloga}
%         Poenostavite.
%         \only<3->{\begin{itemize}
%             \item a \\~
%         \end{itemize}}
%     \end{exampleblock}}
% \end{frame}

% \begin{frame}
%     \only<2->{\begin{exampleblock}{Naloga}
%         Poenostavite.
%         \only<3->{\begin{itemize}
%             \item a \\~
%         \end{itemize}}
%     \end{exampleblock}}
% \end{frame}






    \subsection{Osnovni izrek o deljenju}

        \begin{frame}
            \frametitle{Osnovni izrek o deljenju}
        \end{frame}

    \subsection{Praštevila in sestavljena števila}

        \begin{frame}
            \frametitle{Praštevila in sestavljena števila}
        \end{frame}

    \subsection{Osnovni izrek aritmetike}

        \begin{frame}
            \frametitle{Osnovni izrek aritmetike}
        \end{frame}

    \subsection{Največji skupni delitelj in najmanjši skupni večkratnik}

        \begin{frame}
            \frametitle{Največji skupni delitelj in najmanjši skupni večkratnik}
        \end{frame}


    \subsection{Evklidov algoritem in zveza $D v=ab$}

        \begin{frame}
            \frametitle{Evklidov algoritem in zveza $Dv=ab$}
        \end{frame}

    \subsection{Številski sestavi}

        \begin{frame}
            \frametitle{Številski sestavi}
        \end{frame}

