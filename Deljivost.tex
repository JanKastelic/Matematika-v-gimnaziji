\section{Deljivost}

\begin{frame}
    \sectionpage
\end{frame}

\begin{frame}
    \tableofcontents[currentsection, hideothersubsections]
\end{frame}

    \subsection{Relacija deljivosti}

        \begin{frame}
            \frametitle{Relacija deljivosti}

            \begin{alertblock}{}
                Naravno število $n$ je \textbf{delitelj} naravnega števila $n$ (\textbf{deljenec}), če obstaja naravno število $k$ (\textbf{kvocient}), da velja: $$\mathbf{n=k\cdot m}.$$
            \end{alertblock}

            \begin{alertblock}{}
                Naravno število $m$ deli naravno število $n$, ko je število $n$ večkratnik števila $m$. $$m\mid n \Leftrightarrow n=k\cdot m;\quad m,n,k\in\mathbb{N}$$
            \end{alertblock}

            \begin{block}{}
                    Število $m$ je delitelj samega sebe in vseh svojih večkratnikov.
            \end{block}

            \begin{block}{}
                $1$ je delitelj vsakega naravnega števila.
            \end{block}

            \begin{block}{}
                Če $d$ deli naravni števili $m$ in $n$, $n>m$, potem $d$ deli tudi vsoto in razliko števil $m$ in $n$.
            \end{block}

        \end{frame}

        \begin{frame}
            \begin{block}{}
                Pri deljenju poljubnega naravnega števila $n$ z naravnim številom $m$ imamo dve možnosti: $n$ je deljivo z $m$ ali $n$ ni deljivo z $m$.
            \end{block}

            \begin{block}{}
                Relacija deljivosti je:
                \begin{enumerate}
                    \item \textbf{refleksivna}: $$a\mid a;$$
                    \item \textbf{antisimetrična}: $$a\mid b \wedge b\mid a \Rightarrow a=b;$$
                    \item \textbf{tranzitivna}:  $$a\mid b \wedge b\mid c \Rightarrow a\mid c.$$
                \end{enumerate}
            \end{block}

            \begin{block}{}
                Relacija s temi lastnostmi je relacija \textbf{delne urejenosti}, zato relacija deljivosti delno ureja množico $\mathbb{N}$.
            \end{block}
        \end{frame}

    \subsection{Kriteriji deljivost}

        \begin{frame}
            \frametitle{Kriteriji deljivosti}

            \begin{alertblock}{Deljivost z $2$}
                Število je deljivo z $2$ natanko takrat, ko so enice števila deljive z $2$.
            \end{alertblock}

            \begin{alertblock}{Deljivost z $3$}
                Število je deljivo s $3$ natanko takrat, ko je vsota števk števila deljiva s $3$.
            \end{alertblock}

            \begin{alertblock}{Deljivost z $4$}
                Število je deljivo s $4$ natanko takrat, ko je dvomestni konec števila deljivs s $4$.
            \end{alertblock}

            \begin{alertblock}{Deljivost z $5$}
                Število je deljivo s $5$ natanko takrat, ko so enice števila enake $0$ ali $5$.
            \end{alertblock}

        \end{frame}

        \begin{frame}
            \begin{alertblock}{Deljivost z $6$}
                Število je deljivo s $6$ natanko takrat, ko je deljivo z $2$ in s $3$ hkrati.
            \end{alertblock}

            \begin{alertblock}{Deljivost z $8$}
                Število je deljivo z $8$ natanko takrat, ko je trimestni konec števila deljivo z $8$.
            \end{alertblock}

            \begin{alertblock}{Deljivost z $9$}
                Število je deljivo z $9$ natanko takrat, ko je vsota števk števila deljiva z $9$.
            \end{alertblock}

            \begin{alertblock}{Deljivost z $10$}
                Število je deljivo z $10$ natanko takrat, ko so enice števila enake $0$.
            \end{alertblock}

        \end{frame}




    \subsection{Osnovni izrek o deljenju}

        \begin{frame}
            \frametitle{Osnovni izrek o deljenju}
        \end{frame}

    \subsection{Praštevila in sestavljena števila}

        \begin{frame}
            \frametitle{Praštevila in sestavljena števila}
        \end{frame}

    \subsection{Osnovni izrek aritmetike}

        \begin{frame}
            \frametitle{Osnovni izrek aritmetike}
        \end{frame}

    \subsection{Največji skupni delitelj in najmanjši skupni večkratnik}

        \begin{frame}
            \frametitle{Največji skupni delitelj in najmanjši skupni večkratnik}
        \end{frame}


    \subsection{Evklidov algoritem in zveza $D v=ab$}

        \begin{frame}
            \frametitle{Evklidov algoritem in zveza $Dv=ab$}
        \end{frame}

    \subsection{Številski sestavi}

        \begin{frame}
            \frametitle{Številski sestavi}
        \end{frame}

